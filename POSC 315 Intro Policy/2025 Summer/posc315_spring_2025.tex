% POSC 315 - Spring 2025

\documentclass[11pt, letterpaper]{article}
\usepackage[margin=1.25in]{geometry}
\usepackage{xcolor}
\usepackage[colorlinks=true, linkcolor=blue, urlcolor=blue, citecolor=blue]{hyperref}
\usepackage{url}
\usepackage{graphicx}
\usepackage{bookmark}
\usepackage{enumitem}
\usepackage{caption}
\usepackage{longtable}
\usepackage{booktabs}
\frenchspacing
\usepackage{titlesec}
\titleformat{\section}{\normalfont\Large\bfseries}{\thesection.}{1em}{}
\usepackage[sfdefault]{roboto}


\begin{document}

\title{POSC 315: Introduction to Public Policy \\ Summer 2025 \\ \smallskip \small Asynchronous Online \\ May 27 - June 27, 2025}
\author{Professor: David P. Adams, Ph.D. \\ \smallskip \small Division of Politics, Administration, and Justice \\ California State University, Fullerton}
\date{}
    \maketitle

\subsubsection*{Contact Information:}

\begin{itemize}
    \item Office: Gordon Hall 516
    \item Phone/Text: \href{tel:+16572784770}{(657) 278-4770}
    \item Zoom: \href{https://fullerton.zoom.us/j/3347502369}{\texttt{https://fullerton.zoom.us/j/3347502369}}
    \item website: \href{https://dadams.io}{\texttt{dadams.io}}
    \item email: \href{mailto:dpadams@fullerton.edu}{\texttt{dpadams@fullerton.edu}}
    \item Office Hours:
        \begin{itemize}
            \item Tuesdays 9:30 -- 10:30 a.m. and 7:00 -- 8:00 p.m. on \href{https://discord.com/channels/1128747433636135113/1154048074172354600}{Discord Office Hours Channel}
            \item Feel free to schedule meetings throughout the week: \href{https://dadams.io/appointments}{\texttt{dadams.io/appointments}}
        \end{itemize}  
\end{itemize}


\section{Catalog Description}

	Federal domestic policymaking. Structure, functions, and relationships among American national institutions, including executive, legislative and judicial branches, media, political parties, and pressure groups.

\section{Course Description}

	In this course, students will explore and engage in thoughtful discussions on the processes and key players in creating public policy in the United States. The curriculum focuses on the structure, functions, and relationships among American national institutions, including the executive, legislative, and judicial branches of government, the media, political parties, and interest groups. We will examine the various official and unofficial influences on the policy process and the limitations imposed by institutional and structural factors.

    \vspace{1ex}

	\noindent This course delves into the historical and constitutional development of the policy process, as well as the distinct characteristics of public policy within a federal system. By understanding policy-making in the context of a constitutional republic and a federalized governance system, students will gain a deeper appreciation for the complexities and uncertainties surrounding agenda setting, policy making, policy implementation, and policy evaluation in the American political system.

\section{Student Learning Objectives}

By the end of this course, students will be able to

\begin{enumerate}
	\item discuss and explain the key features of the public policy-making process in the United States;
	\item recognize and describe the distinct stages of the public policy process;
	\item describe the various internal and external actors that influence public policy, their interactions, and their impact on the policy process;
	\item articulate the historical and contemporary structures and institutions that facilitate, expand, or constrain the public policy process;
	\item differentiate and describe the various theories that attempt to explain the drivers and influences leading to public policy change or maintaining the status quo; and
	\item apply their knowledge of the policy process to analyze specific policy domains impacted by multiple policy actors and diverse elements of the policy process within the context of public policy-making in the United States.
\end{enumerate}

\section{General Education Information}

\subsection*{Requirements Satisfied}

	This course satisfies General Education Explorations in Social Sciences subarea D.5. The writing assignments in this course, including the policy memo papers and current event summaries described below, meet the requirement of UPS 411.201: 
	\begin{quote}Writing assignments in General Education courses shall involve the organization and expression of complex data or ideas and careful and timely evaluations of writing so that deficiencies are identified, and suggestions for improvement and/or for means of remediation are offered. Evaluations of the student's writing competence shall determine the final course grade\ldots .\end{quote}
    A grade of “D” (1.0) or higher is required to meet this General Education requirement. A grade of “D-“ (0.7) or below will not satisfy this General Education requirement.

\subsection*{General Education Student Learning Goals}

	Students completing courses in this subarea shall encounter the following learning goals:

\begin{enumerate}
	\item Examine problems, issues, and themes in the social sciences in greater depth; in a variety of cultural, historical, and geographical contexts; and from different disciplinary and interdisciplinary perspectives.
	\item Analyze and critically evaluate the application of social science concepts and theories to particular historical, contemporary, and future problems or themes, such as economic and environmental sustainability, globalization, poverty, and social justice.
	\item Analyze and critically evaluate constructs of cultural differentiation, including ethnicity, gender, race, class, and sexual orientation, and their effects on the individual and society.
	\item Apply theories and concepts from the social sciences to address historical, contemporary, and future problems confronting communities at different geographical scales, from local to global.
\end{enumerate}

\section{Text Book}

\paragraph{}\noindent \textbf{Required}: Birkland, T. A. (2020). \emph{An Introduction to the Policy Process: Theories, Concepts, and Models of Public Policy Making} (5th ed.). Routledge. ISBN: 978-0367333286

\section{Technical Competencies}

Students are expected to have the following technical competencies to succeed in this course:
\begin{itemize}
    \item Basic computer skills, including the ability to navigate the internet, use email, and create and save documents.
    \item Proficiency in using \emph{Canvas}, including submitting assignments, participating in discussions, and accessing course materials.
    \item Ability to use word processing software, such as Microsoft Word or Google Docs, to create and format documents.
    \item Access to a reliable computer and internet connection.
    \item Ability to use Zoom for virtual meetings and discussions.
    \item Basic knowledge of online communication tools, such as email and discussion boards.
    \item Ability to use online research tools and databases to find academic sources.
\end{itemize}

\section{Technical Problems}

\subsection*{University IT Help Desk}

Contact the instructor immediately to document the problem if you encounter any technical difficulties. Then contact the \href{http://www.fullerton.edu/it/students/helpdesk/index.php}{Student IT Help Desk} for assistance. You can also call the Student IT Help Desk at \href{tel:+16572788888}{(657) 278-8888}, \href{mailto:StudentITHelpDesk@fullerton.edu}{email}, visit them at the Pollak Library North \href{http://www.fullerton.edu/it/students/sgc/index.php}{Student Genius Center}, or log on to the \href{http://my.fullerton.edu/}{my.fullerton.edu} portal and click ``Online IT Help'' followed by ``Live Chat''.

\subsection*{Canvas Support}

If you encounter any technical difficulties with Canvas, call the Canvas Support Hotline at \href{tel:+18553027528}{855-302-7528}, visit the \href{https://community.canvaslms.com/docs/DOC-10720-67952720329}{Canvas Community}, or click the ``Help'' button in the lower left corner of Canvas and select ``Report a Problem''. The \href{https://cases.canvaslms.com/liveagentchat?chattype=student&sfid=001A000000YzcwQIAR}{Student Support Live Chat} is available 24 hours a day, 7 days a week.

\section{Response Time} I will strive to respond to all student emails, Discord posts, and \emph{Canvas} messages within 24 hours, except on weekends and holidays. If you are still awaiting a response within 24 hours, please send a follow-up message. If you are still waiting to receive a response within 48 hours, please send another follow-up message and contact me via phone or text at \href{tel:+16572784770}{(657) 278-4770}.


\section{University Student Policies}

In accordance with UPS 300.00, students must be familiar with certain policies applicable to all courses. Please review these policies as needed and visit this Cal State Fullerton website \texttt{\href{https://fdc.fullerton.edu/teaching/student-info-syllabi.html}{https://fdc.fullerton.edu/teaching/student-info-syllabi.html}} for links to the following information:

\begin{enumerate}
    \item   University learning goals and program learning outcomes.
    \item	Learning objectives for each General Education (GE) category.
    \item	Guidelines for appropriate online behavior (netiquette).
    \item	Students' rights to accommodations for documented special needs.
    \item   Campus student support measures, including Counseling \& Psychological Services, Title IV and Gender Equity, Diversity Initiatives and Resource Centers, and Basic Needs Services.
    \item   Disability Support Services (DSS) information.
    \item	Academic integrity (refer to UPS 300.021).
    \item	Actions to take during an emergency.
    \item	Library services information.
    \item	Student Information Technology Services, including details on technical competencies and resources required for all students.
    \item	Software privacy and accessibility statements.
\end{enumerate}

\section{Course Student Policies}

\subsection*{Course Communication}
All course announcements and communications will be sent via \emph{Canvas} and university email. Students are responsible for regularly checking their \emph{Canvas} notifications and email. Students are also responsible for ensuring that their \emph{Canvas} notifications are set to receive messages from the course. Students are expected to check \emph{Canvas} and their email at least once daily.

\subsection*{Due Dates}
All assignments are due by 11:59 p.m. on the specified due date. Save for extenuating circumstances, late assignments will not be accepted unless prior arrangements have been made with the professor. Students are responsible for submitting assignments on time. Failure to submit an assignment on time may result in a failing grade for the assignment. 

\paragraph{}Students are expected to plan ahead and manage their time effectively to ensure that assignments are submitted on time. If you are having trouble meeting a deadline, please contact the professor as soon as possible to discuss the situation.

\subsection*{Alternative Procedures for Submitting Work}
Students are expected to submit all assignments via \emph{Canvas}. If you cannot submit an assignment via \emph{Canvas}, please get in touch with the professor to discuss alternative submission procedures.

\subsection*{Retention of Student Work}
Students are responsible for retaining copies of all assignments submitted for this course. Students are also responsible for retaining copies of all graded assignments returned by the professor.

\subsection*{Extra Credit}
There is no extra credit available in this course. Students are expected to complete all assigned work to the best of their ability. Failure to complete assigned work may result in a failing grade for the course.

\subsection*{Academic Integrity}
Students are expected to adhere to the highest standards of academic integrity. Any student found to have engaged in academic dishonesty will be subject to the sanctions described in the \href{https://www.fullerton.edu/senate/publications_policies_resolutions/ups/UPS%20300/UPS%20300.021.pdf}{Academic Dishonesty Policy} (UPS 300.021). Academic dishonesty includes, but is not limited to, cheating, plagiarism, fabrication, facilitating academic dishonesty, and submitting previously graded work without prior authorization. Students are expected to be familiar with the university's policy on academic dishonesty and to adhere to this policy in all aspects of this course. Any student who has questions about the policy should ask the professor for clarification.

\subsection*{Written Work}
All written work must be submitted in a professional format, including proper grammar, spelling, and punctuation. Written work must also be properly cited using the appropriate citation style. Students are expected to follow the guidelines for written work provided by the professor and to seek clarification if they have questions about the requirements. Assignments that do not meet these standards may be subject to point deductions or resubmission requirements.

\subsection*{Plagiarism}
Plagiarism is a serious violation of academic integrity and will not be tolerated in this course. Plagiarism includes, but is not limited to, copying and pasting text from sources without proper citation, paraphrasing text from sources without proper citation, and submitting work that is not your own. Students are expected to properly cite all sources used in their work and to submit original work. Failure to do so may result in a failing grade for the assignment and further disciplinary action. Written work will be checked for plagiarism using plagiarism detection software. If you are unsure whether your work constitutes plagiarism, consult the professor before submitting.

\subsection*{AI-Generated Text and Tool Usage Policy}

\subsubsection*{Permissible Use of AI Tools}
\textbf{Definition of AI Tools:} In this course, AI tools refer to any software or platform that generates or assists in generating text, ideas, research references, or content creation. This includes, but is not limited to, large language models like OpenAI’s ChatGPT-4, Anthropic’s Claude, AI-based research tools like RefWorks or EndNote, and writing assistants like Grammarly.

\vspace{1ex}

\noindent\textbf{Permissible Use Cases:} Students are allowed to use AI tools for brainstorming, generating ideas, checking grammar, and summarizing content. However, AI tools should not be used to produce final drafts of assignments or significant portions of text that are submitted as original work. If AI tools are used, their contributions must be appropriately cited and disclosed.

\subsubsection*{Ethical Considerations}
\textbf{Academic Integrity:} The use of AI tools must adhere to the highest standards of academic integrity. This means that while these tools can support your work, they cannot replace your own analysis, critical thinking, and original writing. Misuse of AI-generated content is a form of academic dishonesty and will be treated as such under the university's policies.

\vspace{1ex}

\noindent\textbf{Required Disclosures:} Students must disclose their use of AI tools in any assignments. This disclosure should include a brief explanation of how the tool was used and how the student ensured the integrity of their work. For example, a note might state, “I used ChatGPT to generate initial ideas for this essay, but all writing and analysis are my own.”

\subsubsection*{Plagiarism and Originality}
\textbf{AI Detection Tools:} Assignments will be checked for originality using advanced AI detection software. Any submission found to contain unoriginal content generated by AI without proper citation will be subject to the university's academic dishonesty policy. Submitting AI-generated text as your own work, without attribution, is considered plagiarism.

\vspace{1ex}

\subsubsection*{Guidance and Resources}
\textbf{Ethical AI Usage Resources:} Students are encouraged to take advantage of available resources on responsible AI usage. This includes online tutorials, ethical guidelines, and citation practices. Workshops or additional materials on how to use AI tools responsibly may be provided throughout the course.

\vspace{1ex}

\noindent\textbf{Continuous Policy Review:} This AI policy will be reviewed and updated regularly to keep pace with technological advancements. Students are encouraged to provide feedback on this policy to ensure it aligns with their educational goals and the course's academic standards.

\subsubsection*{Example Scenarios}
\textbf{Acceptable Use:} A student uses Grammarly to check the grammar and clarity of their essay. The student does not need to disclose this use unless Grammarly significantly altered the content. Another student uses ChatGPT to brainstorm ideas but writes the essay independently, only citing the AI where directly quoted or paraphrased.

\vspace{1ex}

\noindent\textbf{Unacceptable Use:} A student uses ChatGPT to generate an entire essay and submits it as their own work without attribution. This would be considered a violation of the academic integrity policy and could result in a failing grade or further disciplinary action.

\subsection*{Participation}

Students are expected to participate in all course activities. This includes completing all assigned readings, watching all assigned videos, and participating in all discussions. Students are expected to participate in discussions in a professional and respectful manner. Students are expected to be familiar with the university policy on netiquette and to adhere to this policy in all aspects of this course. Any student who has questions about the policy should ask the professor for clarification. 

\subsection*{Netiquette}
Students are expected to adhere to the university's policy on netiquette. The university's policy on netiquette is as follows:
\begin{quote}Netiquette refers to a set of behaviors that are appropriate for online activity (e.g., social media, email, discussions, presentations). All personnel at Cal State Fullerton are expected to demonstrate appropriate online behavior at all times. A good summary of netiquette can be found in the \href{https://canvashelp.fullerton.edu/m/Student/l/1336786-student-what-is-netiquette}{CSUF Canvas self-help guides}, which adapts ten rules to the online course situation from the website for the book \href{http://www.albion.com/netiquette/corerules.html}{Netiquette by Virginia Shea} and other sources referenced at the bottom of the guide.\end{quote}

\section{Course Assignments}

\subsection*{Reading Quizzes (20\% of Final Grade)}
Weekly reading quizzes assess your understanding of the assigned material and help you stay engaged with key concepts. Quizzes consist of multiple-choice questions delivered through \emph{Canvas}. Each quiz opens on Monday (Tuesday during Week 1) and closes Friday at 11:59 p.m. Late quizzes are not accepted due to the compressed schedule.

\subsection*{Policy Writing Project (80\% of Final Grade)}
This course features a scaffolded policy writing project designed to build your analytical and writing skills in applied policy analysis. You will complete a series of short, focused assignments that culminate in a final policy memo. This process ensures steady progress and allows for instructor feedback at each stage.

\subsubsection*{Project Stages and Deadlines}
\begin{enumerate}
    \item \textbf{Policy Problem Proposal (10\% of Final Grade) - Due Friday, May 31} 
    \begin{itemize}
        \item Submit a 1-2 page memo identifying a policy problem, its significance, and the policy actors involved.
    \end{itemize}

    \item \textbf{Problem Definition Memo (15\% of Final Grade) - Due Friday, June 7}
    \begin{itemize}
        \item Write a 2-3 page memo clearly defining the chosen policy problem, including relevant policy contexts and key stakeholders.
    \end{itemize}

    \item \textbf{Alternatives and Evaluation Matrix (15\% of Final Grade) - Due Friday, June 14}
    \begin{itemize}
        \item Create a policy alternatives matrix summarizing at least three possible solutions to the problem. Include a brief (1-page) narrative explaining your evaluation criteria.
    \end{itemize}

    \item \textbf{Draft Policy Memo (15\% of Final Grade) - Due Friday, June 21}
    \begin{itemize}
        \item Submit a 4-5 page draft policy memo applying concepts from the course. Instructor feedback will help guide your revisions.
    \end{itemize}

    \item \textbf{Final Policy Memo and Reflection (Final Memo 20\%, Reflection 5\%) - Due Friday, June 27}
    \begin{itemize}
        \item Submit the final 5-7 page policy memo incorporating feedback and additional analysis. 
        \item Also submit a 1-2 page personal reflection on what you learned about policy analysis and the writing process.
    \end{itemize}
\end{enumerate}

\paragraph{} This project develops your critical thinking, writing, and applied policy analysis skills. All assignments must be submitted via \emph{Canvas} by 11:59 p.m. on the due date.


\section{Grading Breakdown}
\begin{itemize}
  \item Reading Quizzes (Canvas): 20\%
  \item Policy Problem Proposal: 10\%
  \item Problem Definition Memo: 15\%
  \item Alternatives and Evaluation Matrix: 15\%
  \item Draft Policy Memo: 15\%
  \item Final Policy Memo: 20\%
  \item Final Reflection: 5\%
\end{itemize}

\subsection*{Grading Scale}

\begin{itemize}
    \item A+: 97.00 -- 100.00
    \item A: 93.00 -- 96.99
    \item A-: 90.00 -- 92.99
    \item B+: 87.00 -- 89.99
    \item B: 83.00 -- 86.99
    \item B-: 80.00 -- 82.99
    \item C+: 77.00 -- 79.99
    \item C: 73.00 -- 76.99
    \item C-: 70.00 -- 72.99
    \item D+: 67.00 -- 69.99
    \item D: 63.00 -- 66.99
    \item D-: 60.00 -- 62.99
    \item F: 0.00 -- 59.99
\end{itemize}

\section*{Weekly Schedule}

\subsection*{Week 1: May 27 - May 31}
\textbf{Topics:} \newline
Chapters 1-3: Introducing the Policy Process; Elements of the Policy-Making System; Contexts of Public Policy Making.\newline
\textbf{Assignments:} \begin{itemize}
  \item Reading Quiz 1 (Tuesday-Friday)
  \item Policy Problem Proposal (1-2 pages) \textbf{Due Friday, May 31}
\end{itemize}

\subsection*{Week 2: June 3 - June 7}
\textbf{Topics:} \newline
Chapters 4-6: Official Actors; Unofficial Actors; Agenda Setting, Groups, and Power.\newline
\textbf{Assignments:} \begin{itemize}
  \item Reading Quiz 2 (Monday-Friday)
  \item Problem Definition Memo (2-3 pages) \textbf{Due Friday, June 7}
\end{itemize}

\subsection*{Week 3: June 10 - June 14}
\textbf{Topics:} \newline
Chapters 7-8: Policies and Policy Types; Decision Making and Policy Analysis.\newline
\textbf{Assignments:} \begin{itemize}
  \item Reading Quiz 3 (Monday-Friday)
  \item Alternatives and Evaluation Matrix \textbf{Due Friday, June 14}
\end{itemize}

\subsection*{Week 4: June 17 - June 21}
\textbf{Topics:} \newline
Chapters 9-10: Policy Design and Tools; Policy Implementation, Failure, and Learning.\newline
\textbf{Assignments:} \begin{itemize}
  \item Reading Quiz 4 (Monday-Friday)
  \item Draft Policy Memo (4-5 pages) \textbf{Due Friday, June 21}
\end{itemize}

\subsection*{Week 5: June 24 - June 27}
\textbf{Topics:} \newline
Chapter 11: Science and Theory in the Study of Public Policy; Course Wrap-Up.\newline
\textbf{Assignments:} \begin{itemize}
  \item Reading Quiz 5 (Monday-Friday)
  \item Final Policy Memo (5-7 pages) \textbf{Due Friday, June 27}
  \item Final Reflection (1-2 pages) \textbf{Due Friday, June 27}
\end{itemize}


\end{document}
