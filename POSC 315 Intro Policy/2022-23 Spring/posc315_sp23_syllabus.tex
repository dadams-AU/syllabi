\documentclass[11pt]{article}
\usepackage[margin=1in]{geometry}
\usepackage{url}
\usepackage{graphicx} 

\usepackage{xcolor}
\usepackage[colorlinks]{hyperref}
\hypersetup{
	colorlinks,
	citecolor=violet,
	linkcolor=red,
	urlcolor=blue}
\usepackage{enumitem}
\usepackage{longtable}
\usepackage{hanging}
\frenchspacing
\usepackage{multicol}
\usepackage{eso-pic} 

\begin{document}
		
\AddToShipoutPicture*
    {\put(500,700){\includegraphics[width=2.5cm,height=2.5cm]{meeting_qr}}}
	
	\title{\textbf{Introduction to Public Policy} \\
			POSC 315-01 ~\textbullet~ Spring 2023}
	
	\author{	
			\Large{Instructor: \textsc{David P. Adams, PhD }} \\ \\
			\textbf{Office:} 516 Gordon Hall \\ 
			\textbf{Phone:} (657) 278-4770 ~\textbullet~ \textbf{Email:} \href{mailto:dpadams@fullerton.edu}{\texttt{dpadams@fullerton.edu}} \\\\
			\textbf{Office Hours:}  Tuesdays and Thursdays 9:30 - 11:30, \\  Thursdays 5:30 - 6:30, and by appointment at \url{https://t.ly/4dpa}}
			

\date{ \textbf{Dates:} January 24 -- May 18 ~\textbullet~ \textbf{Time:} Tuesdays and Thursdays 2:30 - 3:45 \textsc{p.m.} \\\vspace{1ex} \textbf{Place:} GH 248}

\maketitle



\begin{description}
	
	\item \underline{Course Description:} \\
	In this class, students should be prepared to learn and thoughtfully discuss the processes and actors involved in making public policy in the United States. The focus is on the structure, functions, and relationships among American national institutions, including the executive, legislative, and judicial branches of government, the media, political parties, and interest groups. The policy process is influenced in many ways, both officially and unofficially, and is restricted by various institutional and structural aspects. This course explores the historical and constitutional development of the policy process, as well as the unique nature of public policy in a federal system.  
	
	Understanding how policy is made within a constitutional republic with a federalized system of governance will allow students to appreciate better the nature of the American political system and the complexity and uncertainty of agenda setting, policy making, policy implementation, and policy evaluation.
		
	\item \underline{Student Learning Objectives:} 
	    \begin{enumerate}
	   		 \item Students will be able to discuss and explain general qualities of the policy process as it occurs in the United States.
		 
	   		 \item Students will be able to identify specific characteristics of the several ``stages'' of the policy process. 
		 
	   		 \item Students will be able to explain the numerous endogenous and exogenous actors who influence public policy, how they interact, and how they shape the policy process. 
		 
	   		 \item Students will be able to articulate the historical and contemporary structures and institutions which promote, expand, and/or constrict the policy process. 
	     
	   	     \item Students will be able to distinguish and articulate the differences between several theories that attempt to explain the mechanisms and influences which lead to policy change or ensure the status quo.
		 
	   		 \item Students will be able to apply their understanding of the policy process to specific policy domains that are affected by numerous policy actors and  various aspects of the policy process within the context United States public policy making. 
	   		\end{enumerate}

\item \underline{General Education}
	\begin{description}
		\item[]General Education Requirement Satisfied by this Course: \\ This course satisfies General Education Explorations in Social Sciences subarea D.5 for those using Catalog Years 2017 or earlier and subarea D.4 for those using Catalog Years 2018 and later. The writing assignments in this course, including the policy memo papers and current event summaries, described below, meet the requirement of UPS 411.201: ``Writing assignments in General Education courses shall involve the organization and expression of complex data or ideas and careful and timely evaluations of writing so that deficiencies are identified, and suggestions for improvement and/or for means of remediation are offered. Evaluations of the student’s writing competence shall determine the final course grade\ldots.''
		
		\item[] General Education Student Learning Goals: \\ Students completing courses in subarea D.5 shall encounter the following learning goals.
			\begin{enumerate}
				\item Examine problems, issues, and themes in the social sciences in greater depth; in a variety of cultural, historical, and geographical contexts; and from different disciplinary and interdisciplinary perspectives.
				\item Analyze and critically evaluate the application of social science concepts and theories to particular historical, contemporary, and future problems or themes, such as economic and environmental sustainability, globalization, poverty, and social justice.
				\item Analyze and critically evaluate constructs of cultural differentiation, including ethnicity, gender, race, class, and sexual orientation, and their effects on the individual and society. 
				\item Apply theories and concepts from the social sciences to address historical, contemporary, and future problems confronting communities at different geographical scales, from local to global.
			\end{enumerate}
	\end{description}
			
\item \underline{Text Book:}
	\begin{itemize}
			\item Birkland, Thomas A. 2020. \emph{An Introduction to the Policy	Process: Theories, Concepts, and Models of Public Policy Making.} 5th edition. New York: Routledge.  
	\end{itemize}

\item \underline{Graded Items:}
	\begin{multicols}{2}
		\begin{enumerate}
			\item 3 Tests: 45\%
			\item Policy Issue Analyses 15\%
			\item 2 Documentary Response Papers: 10\%
			\item Final Term Paper 25\%
			\item Attendance: 5\%
		\end{enumerate} 
	\end{multicols}

\item \underline{Graded Assignments and Examinations:} 

\begin{itemize}
	\item \underline{Three Examinations:} Three non-cumulative exams will assess students knowledge of course learning objectives and key concepts of American politics, public policy making and the policy process. Questions consist of 50 multiple choice questions. 
	
	\item \underline{Policy Term Paper:}Each student is expected to research and write on a specific public policy. The general idea of this paper is to discuss real-world public policy in terms of the concepts learned in class and through course readings. Additional details, instructions, and a grading rubric for this assignment will be provided during the second week of class. The final paper is due as indicated in the course schedule below. Students may choose any specific public policy, including court decisions, executive regulations, and legislative acts. Feel free to discuss the policy with Dr. Adams for feedback and suggestions during office hours. Begin thinking about your policy at the beginning of the semester so that you can relate course concepts to it throughout the term.

	\item \underline{Documentary Response Papers:} Two short documentary response papers will allow students to further explore concepts learned in class throughout the semester by discussing them in terms of a real-world policy domain. Additional details about this assignment will be provided during the second week of class. These papers should be \textit{no more} than 500 words of text and submitted as a \emph{Canvas} assignment prompt. 
	
	\item \underline{Policy Issue Analysis:} Throughout the semester we will discuss several recent public policy issues. These will come from Congressional Quarterly (CQ) Researcher's \href{https://library.cqpress.com/cqresearcher/toc.php?mode=cqres-date&level=2&values=2022}{\emph{Issues for Debate in American Public Policy} 2022 edition}. Links to the issues will be provided on Canvas. These are accessible from CSUF computers or while on the campus wi-fi. You can download .pdf's while on campus to read while off campus. Students are expected about 250 words for each analysis.  
	
	\item \underline{Attendance and Participation:} Students are expected to attend each class meeting and participate in class discussions about relevant policy topics. Attendance will be taken each day the class meets and students are expected to participate in and contribute to discussions. Students may take two absences this semester without penalty. 
	
	\item \underline{Current Event Days:} On two days of class, as indicated in the schedule below, we will have \underline{no in-person class} session. These will be days specifically set aside to meet with the professor to discuss your policy paper topic. \href{https://outlook.office.com/bookwithme/user/48008ba4b4ee4cb7bf1a0ba4fb62ae1a@Fullerton.edu?anonymous&ep=plink}{You can sign up here}. As well, a few current event topics will be linked to a canvas discussion post. Students can choose one or more current events to briefly discuss in terms of the course concepts in about a paragraph. Meaningful participation in the current event discussions will result in extra credit on the policy paper.

	
\end{itemize}

\item \underline{Course Policies:}

	\begin{description}
	
	\item[Course Communication and Response Time:] All course announcements and individual emails are sent through \emph{Canvas} or CSUF email, using only CSUF email accounts. Therefore, students MUST check their CSUF email regularly (i.e., several times a week) for the course duration. The instructor will respond to emails and phone calls within 48 hours, including weekends. Students are also encouraged to ask the instructor for clarification on any issue, whether it be for course materials, assignments, or any other question related to the course. Just as the student expects the instructor to respond promptly to his or her communication, the instructor likewise expects a timely response from the student. Students should feel free to contact the instructor via email with any inquiry concerning matters of the course.
	
	\item[Participation:] As shown in the course schedule below, there are several days during which a specific issue will be discussed and placed in the context of policy processes thus far explored in class lectures and readings; it is the student's responsibility to be actively engaged in these discussions. Be sure to help out your fellow classmates and actively participate to incorporate course concepts into current policy issues. 
	
	\item[Reading:] To succeed in this course, the student must read course materials \emph{prior to coming to class}. Readings are indicated on the course schedule below for the day during which they are discussed in class. 
	
	\item[Assignment Deadlines and Missed Exams:] Firm assignment due dates are outlined in the course schedule at the end of this document. These deadlines exist for two examinations, three current event policy summaries, two documentary response papers, and the final exam; \emph{a 10\% grade penalty is assessed for each day these items are late (up to 50\%)}, though exceptions are made for illness, emergency, or crisis situation, or with an excused absence. If applicable, students must make arrangements to make up missed exams. Late assignments will be accepted up to one week after the original due date, no more. 
	
	\item[Grading Policy:] All graded items, including the final course grade,
	are assigned according to the following scale:
	\begin{center}

		A+ = 98--100\%, A = 93--97\%, A- = 90--92\%, B+ = 87--89\%, B = 83--86\%, \\ B- = 80--82\%, C+ = 77--79\%, C = 73--76\%, C- = 70--72\%, D+ = 67--69\%, \\ D = 63--66\%, D- = 60--62\%, F = 0--59\%

	\end{center}
	Grades shall be recorded on \emph{Canvas} and items will be returned to the
	the student as soon as possible.
	
	\item[Technical Requirements:] Students are expected to:
	\begin{enumerate}
		\item Have basic computer competency, which includes:
		\begin{enumerate}
			\item the ability to use a personal computer to locate, create, move, copy, delete, name, rename, and save files and folders on hard drives, secondary storage devices, and clouds such as Dropbox and Google Drive;
			\item the ability to use a word processing program to create, edit, format, store, retrieve, and print documents;
			\item the ability to use their CSUF email accounts to receive, create, edit, print, save, and send an email message with and without an attached file(s);
			\item the ability to use an Internet browser such as Chrome, Safari, Firefox, or Internet Explorer to search and access web sites.
		\end{enumerate}
		\item Have ongoing reliable access to a computer with Internet connectivity for regular course assignments.
		\item Utilize a word processing program such as Microsoft Word and a pdf viewer such as Adobe Acrobat; and have the ability to regularly print assignments.
		\item Maintain and access three times weekly their CSUF student email account.
		\item Utilize \emph{Canvas} to access course materials and complete assignments.
	\end{enumerate}
	
	\end{description}


\item \underline{University Information:}

\begin{description}
	\item[Academic Honesty:] Academic dishonesty includes such things as cheating, inventing false information or citations, plagiarism, and helping someone else commit an act of academic dishonesty. It usually involves an attempt by a student to show a possession of a level of knowledge, which, in fact, the student does not possess. Cheating is defined as the act of obtaining or attempting to obtain credit for work by the use of any dishonest, deceptive, fraudulent, or unauthorized means. Plagiarism is defined as the act of taking the work of another and offering it as one's own without giving credit to that source. An instructor who believes that an act of academic dishonesty has occurred is (1) obliged to discuss the matter with the student(s) involved; (2) should possess reasonable evidence such as documents or personal observation; and (3) may take whatever action deemed appropriate, ranging from an oral reprimand to an F in the course. Additional information on this policy is available from the \href{http://www.fullerton.edu/senate/publications_policies_resolutions/ups/UPS 300/UPS 300.021.pdf}{University Policy Statement 300.021}.
	
		\item[Accommodations for students with special needs:] Please inform the instructor during the first week of classes about any disability or special needs that you may have that may require specific arrangements related to attending class sessions, carrying out class assignments, or writing papers or examinations. Please do so by emailing the instructor to make an appointment to discuss your specific needs. According to California State University policy, students with disabilities must document their disabilities at the Disability Support Services (DSS) Office to be accommodated in their courses. Additional information can be found at the \href{http://www.fullerton.edu/dss}{DSS Website}, by calling 657-278-3112, or by emailing \href{mailto: dsservices@fullerton.edu}{dsservices@fullerton.edu}.  
	
		\item[Emergency Preparedness:] To be able to respond effectively in an emergency, be sure to note (1) fire alarm pull station locations, (2) evacuation map including the class's outside meeting area, (3) emergency procedures for fire, medical emergency, hazardous materials release, earthquake and dangerous situations, and (4) the location of the nearest emergency phone. Any person with special needs is encouraged to privately speak with the instructor. All campus personnel are required to participate in all campus-wide emergency drills. Emergency preparedness information can be found at the \href{http://prepare.fullerton.edu/campuspreparedness/ClassroomPreparedness.asp}{Classroom Preparedness website}.
	
		If an emergency disrupts normal campus operations or causes the University to close for a prolonged period of time (such as more than three days), students are expected to complete the course assignments listed on this syllabus as soon as reasonably possible to do so. At the instructors discretion, the syllabus may be amended or updated to reflect changing circumstances related to an emergency.
		
		\begin{itemize}
			\item[] Before and Emergency Occurs:
				\begin{enumerate}
					\item Know the safe evacuation routs for your specific building and floor.
					\item Know the evacuation assembly areas for your building.
				\end{enumerate} 
			\item[] If an Emergency Occurs:
				\begin{enumerate}
					\item Keep calm; do not run or panic. It is best to maintain a clear head in an emergency situation.
					\item Evacuation is not always the safest course of action. If directed to evacuate, take your belongings and proceed safely to the nearest evacuation route. 
					\item Do not leave the area. Remember that faculty and other staff members need to account for your whereabouts.
					\item Do not re-enter the building until informed it is save by a building marshal or other campus authority.
					\item If directed to evacuate the campus, please follow the evacuation routes established by either parking or police officers. 
				\end{enumerate}
		\end{itemize}
		
		\item[University Learning Center:] The goal of the University Learning Center (ULC) is to provide all CSUF students with academic support in an inviting and contemporary environment. The staff of the ULC will assist students with all their academic assignments, general study skills, and computer user needs. The ULC staff work with all students from diverse backgrounds in most undergraduate general education courses including those in science and mathematics, humanities and social sciences, and other subjects. They offer one-to-one peer tutoring, online writing review, and many more services. More information can be found on the \href{http://www.fullerton.edu/ulc/}{University Learning Center website}.
		
		\item[Writing Center:] The Writing Center offers 30-minute one-on-one peer tutoring sessions and workshops, aimed at providing assistance for all written assignments and student writing concerns. Writing Center services are available to students from all academic disciplines. Registration and appointment schedules are available at the \href{http://fullerton.mywconline.com/}{Writing Center Appointment Scheduling System}. Walk-in appointments are also available on a first come, first serve basis to students who have registered online. More information can be found at the \href{http://www.fullerton.edu/learningassistance/tutoring_centers/writing.asp}{Writing Center's webpage}. The Writing Center is located on the first floor of the Pollak Library; their phone number is 657-278-3650.
\end{description}

%\newpage
\begin{large}
	\item \underline{Course Schedule by Week}
\end{large}
\end{description}
\begin{center}
\begin{longtable}{p{1.5cm} | p{7.8cm} | p{7.5cm}}	
	\large{\textbf{Week}} & \large{\textbf{Tuesday}} 						& \large{\textbf{Thursday}} 						\\ \hline \hline
	
	\emph{Week 1} 	& \underline{1/24:} \textbf{Introduction}				& \underline{1/26:} \textbf{Overview of Policy} 		\\
	\emph{Readings}	& Syllabus 												& \textit{Birkland ch. 1}								\\
					& 														&	American Regime Values								\\ \hline
	
	\emph{Week 2} 	& \underline{1/31:} \textbf{Policy Process Elements}	&  \underline{2/2:} \textbf{History \& Structure} 		\\
	\emph{Readings} & \textit{Birkland ch. 2}								&  \textit{Birkland ch. 3}									\\
					& 														& 														\\ \hline
	
	\emph{Week 3}	& 	\underline{2/7:} \textbf{Documentary 1}				& \underline{2/9:} \textbf{Issue Discussion 1}				\\
	\emph{Readings}	& 														& 															\\
					& 														& 														\\ \hline
	
	\emph{Week 4}	& \underline{2/14:} \textbf{Official Policy Actors}		& \underline{2/16:} \textbf{Unofficial Policy Actors}		\\
	\emph{Readings}	& \textit{Birkland ch. 4 }								& \textit{Birkland ch. 5}								\\
					&														& Review for Test 1										\\ \hline
				
	\emph{Week 5}	& \underline{2/21:} \textbf{TEST 1}		 				& \underline{2/23:} \textbf{Current Event Day} 				\\
	\emph{Readings}	& 														& \textit{No In-Person Class}							\\
					& 														& \href{https://t.ly/4dpa}{Office Hours to Discuss Policy Paper}			\\ \hline	

	
	\emph{Week 6}  	& \underline{2/28:} \textbf{Issue Discussion 2}			& \underline{3/2:} \textbf{Agenda Setting}			\\
	\emph{Readings}	& 														& \textit{Birkland ch. 6} 									\\
					& 														& 													\\ \hline	
	
	\emph{Week 7}	& \underline{3/7:} \textbf{Policy Types}				& \underline{3/9:} \textbf{Decision Making I}		\\
	\emph{Readings}	& \textit{Birkland ch. 7}								& \textit{Birkland ch. 7}						 	\\
					& 														& 													\\ \hline	
	
	\emph{Week 8}	& \underline{3/14:} \textbf{Decision Making II} 		& \underline{3/16:} \textbf{Policy Analysis}			\\
	\emph{Readings}	& \textit{Birkland ch. 8}								& \textit{Birkland ch. 9}							\\
					& 														& 													\\ \hline	
	
	\emph{Week 9}	& \underline{3/21:} \textbf{Issue Discussion 3}			& \underline{3/23:} \textbf\textbf{Test 2}				\\
	\emph{Readings}	& Review for Test 2										& 	  													\\
					& 														& 													\\ \hline	
	
	\emph{Week 10} 	& \underline{4/4:} \textbf{Documentary 2}				& \underline{4/6:} \textbf{Policy Design \& Tools} 			\\
	\emph{Readings}	& \textbf{No In-Person Class}							& \textit{Canvas Reading}							\\ 
					&														&													\\\hline	
	
	\emph{Week 11}  & \underline{4/10:} \textbf{Implementation} 			& \underline{4/12:} \textbf{Issue Discussion 4} 			\\
	\emph{Readings}	& \textit{Birkland ch. 10}								& 													\\
					& 														& 													\\ \hline	
	
	\emph{Week 12} 	& \underline{4/18:} \textbf{Failure \& Learning}	 	& \underline{4/20:} \textbf{Evaluation \& Science}		\\
	\emph{Readings}	& \textit{Birkland ch. 10}								& \textit{Birkland ch. 11}							\\
					&														& 													\\\hline	
		
	\emph{Week 13} 	& \underline{4/25:} \textbf{Issue Discussion 5}			& \underline{4/27:} \textbf{Current Event Day}		\\
	\emph{Readings}	& 														& \href{https://t.ly/4dpa}{Office Hours to Discuss Policy Paper}		\\
					& 														& \textit{No In-Person Class}						\\ \hline	
	
	\emph{Week 14} 	& \underline{5/2:} \textbf{Bonus Documentary} 			& \underline{5/4:} \textbf{Policy Science}		\\
	\emph{Readings}	& 														& \textit{Birkland ch. 11}							\\
					& \textbf{Policy Paper Due}								& 													\\ \hline	
	
	\emph{Week 15}	& \underline{5/9:} \textbf{Collaboration}				& \underline{5/11:} \textbf{Final Exam Review}		\\
	\emph{Readings}	& \textit{Canvas Reading} 								& 					 								\\
					& Wrap-up remaining topics								&													\\ \hline	
					
	
\end{longtable}
\end{center}
\begin{center}
	\begin{Large} \textbf{TEST 3: Online May 18th}
\end{Large}
\end{center}

\end{document}