% POSC 315 Fall 2025 - Asynchronous Online
% Document metadata for PDF accessibility and compliance
\DocumentMetadata{
  pdfstandard=UA-2,               % PDF/UA-2 standard for accessibility
  pdfversion=2.0,                 % PDF version 2.0
  lang=en-US                      % Document language set to US English
}

% Document class and font size
\documentclass[12pt]{article}     % Standard article class with 12pt font size

% Language settings
\usepackage[american]{babel}      % Set language to American English

% ================================
% FONT SETTINGS
% ================================

\usepackage{fontspec}

% === Recommended (Sans-serif body text) ===
\IfFontExistsTF{TeX Gyre Heros}{
  \setmainfont{TeX Gyre Heros} % Default to sans-serif
}{
  \setmainfont{Latin Modern Sans} % Fallback sans
}

% Monospaced font (for code, etc.)
\IfFontExistsTF{TeX Gyre Cursor}{
  \setmonofont{TeX Gyre Cursor}
}{
  \setmonofont{Latin Modern Mono}
}

% Page layout and spacing
\usepackage{geometry}             % Adjust page margins
\geometry{margin=1in}             % Set 1-inch margins
\usepackage{setspace}             % Control line spacing
\onehalfspacing                   % Set line spacing to 1.5

% List, graphics, table, and caption settings
\usepackage{enumitem}             % Enhanced list customization
\setlist[itemize]{                % Default settings for itemized lists
  itemsep=0pt,                    % No extra space between items
  parsep=0pt,                     % No extra space between paragraphs
  topsep=0.25\baselineskip        % Small space before the list
}
\newlist{flatlist}{itemize}{1}    % Define a bulletless, flush-left list
\setlist[flatlist]{               % Settings for flatlist
  label={},                       % No bullet or label
  leftmargin=0pt,                 % No left margin
  itemsep=0pt, parsep=0pt,        % No extra spacing
  topsep=0pt                      % No space before the list
}
\usepackage{graphicx}             % Include graphics
\usepackage{array, booktabs, longtable} % Enhanced table formatting
\usepackage{caption}              % Customize captions
\usepackage{tabularray}           % Modern table package
\UseTblrLibrary{booktabs}         % Use booktabs for better table rules
\usepackage{colortbl}
\usepackage{xcolor}               % Color support


% Tagging for accessibility
\IfFileExists{tagpdf-base.sty}{
  \usepackage{tagpdf}
  \tagpdfsetup{activate-all=true, interwordspace=true}
}{
  \newcommand\tagpdfsetup[1]{}
}

% Hyperlinks and PDF metadata
\usepackage[
  pdfusetitle,                    % Use \title and \author for PDF metadata
  pdflang=en-US,                  % Set PDF language to US English
  pdfstartview=FitH,              % Open PDF with horizontal fit
  pdfdisplaydoctitle=true         % Display document title in PDF viewer
]{hyperref}

\hypersetup{
  unicode=true,                   % Enable Unicode support
  colorlinks=true,                % Use colored links
  linkcolor=blue,                 % Color for internal links
  urlcolor=blue,                  % Color for URLs
  citecolor=blue,                 % Color for citations
  pdftitle={POSC 315: Introduction to Public Policy}, % PDF title
  pdfauthor={David P. Adams, Ph.D.},          % PDF author metadata
  pdfsubject={California State University, Fullerton Course Syllabus}, % PDF subject
  pdfkeywords={CSUF, Syllabus, Political Science, Public Policy}, % PDF keywords
  pdfborderstyle={/S/U/W 1}       % Underline links in PDF
}
\urlstyle{same}                   % Use the same font style for URLs

% Document title and metadata
\title{\textbf{POSC 315-50} \\ \textsc{Introduction to Public Policy}} % Course title
\author{}                         % Leave author blank (info in body)
\date{Fall 2025}                  % Term and year of the course

% Accessibility enhancements
\setlistdepth{3}                  % Ensure consistent list tagging depth

% Begin the document
\begin{document}

\maketitle

\begin{center}
    \includegraphics[width=2.75in, alt={Cal State Fullerton wordmark}]{csuf_logo.png}
\end{center}


% ========== SECTION 1: FACULTY INFORMATION ==========
\section{Faculty Information}
% Replace placeholders with your information
\noindent \textbf{Instructor:} David P. Adams, Ph.D. \\
\noindent \textbf{Pronouns:} He/Him \\
\noindent \textbf{Office:} Gordon Hall 521 \\
\noindent \textbf{Phone/Text:} (657) 278-4770 \\
\noindent \textbf{Website:} \href{https://dadams.io}{dadams.io} \\
\noindent \textbf{Email:} dpadams@fullerton.edu \\
\noindent \textbf{Zoom ID:} 334 750 2639 \\
\noindent \textbf{Office hours:} Mondays 7:00--8:00 p.m. on \href{https://discord.gg/yxCXU9yC}{Discord Office Hours Channel}; Tuesdays at 10:00--11:00 a.m. in person; schedule individual in-person or Zoom appointments at \href{https://dadams.io/appointments}{dadams.io/appointments}. 

% ========== SECTION 2: COURSE COMMUNICATION ==========
\section{Course Communication}
All course announcements and communications will be sent via \emph{Canvas} and university email. Students are responsible for regularly checking their \emph{Canvas} notifications and email at least once daily. Please ensure your \emph{Canvas} notifications are set to receive messages from the course.

For questions and support, use the course Discord server for the fastest response. For private matters, email me at dpadams@fullerton.edu. I strive to respond within 24 hours on weekdays.

% ========== SECTION 3: TECHNICAL PROBLEMS ==========
\section{Technical Problems}
If you encounter any technical difficulties, contact the instructor immediately to document the problem. Then, contact: \href{http://www.fullerton.edu/it/students/helpdesk/index.php}{student IT help desk}, \href{mailto:StudentITHelpDesk@fullerton.edu}{email}, phone (657) 278-8888, walk-in \href{http://www.fullerton.edu/it/students/sgc/index.php}{student genius center}, online chat - log into \href{http://my.fullerton.edu}{portal}; click ``Online IT Help''; click ``Live Chat.''

\vspace{0.5em}
\noindent \textbf{\underline{For issues with Canvas}}: Canvas Support Hotline = (657) 278-8888, \href{https://canvashelp.fullerton.edu/}{search the CSUF Canvas Guides with AI Assistant}, or \href{https://titans.service-now.com/sp?id=sc_cat_item&sys_id=f88efe80ebea6a10fb7cfcffcad0cdc6&subject=Canvas}{report a problem.}

\vspace{0.5em}
\noindent \textbf{\underline{For issues with Kritik}}: Contact \href{mailto:support@kritik.io}{support@kritik.io} or use the help resources at \href{https://us.kritik.io/help}{us.kritik.io/help}. Document any Kritik issues immediately with screenshots and notify the instructor.

\vspace{0.5em}
\noindent \textbf{Alternative plan for submitting work:} If you cannot submit an assignment via Canvas or Kritik, email it to me immediately at dpadams@fullerton.edu with documentation of the technical issue.

\vspace{0.5em}
\noindent \textbf{Response time:} I will strive to respond to all student emails, Discord posts, and Canvas messages within 24 hours, except on weekends and holidays. If you have not received a response within 24 hours, please send a follow-up message. If still no response within 48 hours, please text or call (657) 278-4770.
% ========== SECTION 4: COURSE INFORMATION ==========
\section{Course Information}
\noindent \textbf{Prefix, number, title:} POSC 315, \textit{Introduction to Public Policy} \\
\noindent \textbf{Meeting times with modality:} Fully online (100\% online), asynchronous

\vspace{0.5em}
\begin{itemize}
  \item \textbf{Course requisite(s):} POSC 100, GE Area 4A 

  \item \textbf{Catalog description:} Federal domestic policymaking. Structure, functions, and relationships among American national institutions, including executive, legislative and judicial branches, media, political parties, and pressure groups.

  \item \textbf{Additional description:} This course explores the processes and key players in creating public policy in the United States. We examine the various official and unofficial influences on the policy process and the limitations imposed by institutional and structural factors. Students will gain a deeper appreciation for the complexities surrounding agenda setting, policy making, implementation, and evaluation in the American political system.

  \item \textbf{Workload expectation:} This course requires consistent engagement throughout the semester. The scaffolded paper project (60\% of grade) means you'll be working on some aspect of your policy analysis nearly every week. Plan accordingly and avoid procrastination, as each stage builds on previous work. Expect to spend 8--10 hours per week on course materials, assignments, and peer reviews.

  \item \textbf{Course materials and equipment:}
  \begin{itemize}
    \item \textbf{Required text(s):} Birkland, T. A. (2020). \emph{An Introduction to the Policy Process: Theories, Concepts, and Models of Public Policy Making} (5th ed.). Routledge. ISBN: 978-0367333286
    \item \textbf{Other materials:} Reliable computer and internet connection, access to Canvas, Discord, and Kritik platforms
  \end{itemize}
\end{itemize}

\vspace{1em}
\noindent \textbf{Student Learning Outcomes:}
\begin{enumerate}
\item Discuss and explain the key features of the public policy-making process in the United States
\item Recognize and describe the distinct stages of the public policy process
\item Describe the various internal and external actors that influence public policy, their interactions, and their impact on the policy process
\item Articulate the historical and contemporary structures and institutions that facilitate, expand, or constrain the public policy process
\item Differentiate and describe the various theories that attempt to explain the drivers and influences leading to public policy change or maintaining the status quo
\item Apply knowledge of the policy process to analyze specific policy domains impacted by multiple policy actors and diverse elements of the policy process
\end{enumerate}

% ========== SECTION 5: GRADING POLICIES AND STANDARDS ==========
\section{Grading Policies and Standards}

% Part a: Grading Scale
\noindent \textbf{a. Grading scale:}

The grading scale for this course is outlined in Table~\ref{tab:grade-scale}. Letter grades correspond to the percentage ranges shown. Note that grades are not rounded up.

\begin{center}
\begin{table}[h]
  \caption{Grade scale}
  \label{tab:grade-scale}
  \centering
  \begin{tblr}{
    colspec = {l c l c},
    rowhead = 1,
    row{1} = {font=\bfseries, bg=gray!20},
  }
  Grade & Percent    & Grade & Percent \\
  A+    & 97.0--100.0& C+    & 77.0--79.9 \\
  A     & 93.0--96.9 & C     & 73.0--76.9 \\
  A-    & 90.0--92.9 & C-    & 70.0--72.9 \\
  B+    & 87.0--89.9 & D+    & 67.0--69.9 \\
  B     & 83.0--86.9 & D     & 63.0--66.9 \\
  B-    & 80.0--82.9 & D-    & 60.0--62.9 \\
        &            & F     & 0.0--59.9 \\
  \end{tblr}
\end{table}
\end{center}

% Part b: Required Course Assignments
\vspace{1em}
\noindent \textbf{b. Required Course Assignments:}

The breakdown of required course assignments and their respective weightings is provided in Table~\ref{tab:assignment-weighting}. These assignments are designed to assess your understanding of the course material and your ability to apply it effectively.

\begin{center}
\begin{table}[h]
  \caption{Assignment weighting}
  \label{tab:assignment-weighting}
  \centering
  \begin{tblr}{
    colspec = {l c},
    rowhead = 1,
    row{1} = {font=\bfseries, bg=gray!20},
  }
  Assignment                          & Percentage \\
  Module Quizzes (weekly)             & 25\% \\
  Policy Process Analysis Paper       & 60\% \\
  \quad Topic Selection \& Proposal   & \quad 5\% \\
  \quad Annotated Bibliography        & \quad 10\% \\
  \quad Literature Review              & \quad 15\% \\
  \quad Draft Analysis                 & \quad 15\% \\
  \quad Final Paper                    & \quad 15\% \\
  Discussion Participation             & 10\% \\
  Attendance and Participation         & 5\%  \\
  Total                               & 100\% \\
  \end{tblr}
\end{table}
\end{center}


\noindent \textbf{Module Quizzes (25\% of Final Grade):} Each module includes a quiz following the video content to assess understanding of key concepts. Quizzes consist of 10--20 multiple-choice questions and must be completed by Sunday at 11:59 p.m. of each week. Lowest quiz score dropped.

\vspace{0.5em}
\noindent \textbf{Policy Process Analysis Paper (60\% of Final Grade):} This scaffolded writing project develops your ability to analyze a specific policy issue through the lens of policy process theories. Each stage builds on the previous work and includes peer review via Kritik (integrated into each assignment's grade):

\begin{itemize}
\item \textbf{Topic Selection \& Proposal (5\%):} 1--2 page proposal defining your policy issue and research question
\item \textbf{Annotated Bibliography (10\%):} Compile and evaluate 8--10 scholarly sources with 150--200 word annotations
\item \textbf{Literature Review (15\%):} 4--5 page synthesis of existing research, identifying gaps and theoretical connections
\item \textbf{Draft Analysis (15\%):} 7--9 page application of a theoretical framework to analyze your policy issue
\item \textbf{Final Paper (15\%):} Complete 10--12 page analysis incorporating all feedback received
\end{itemize}

\textit{Note: Each assignment grade includes both your submission quality and peer review participation via Kritik. Missing peer reviews will significantly impact that assignment's grade.}

\vspace{0.5em}
\noindent \textbf{Discussion Participation (10\% of Final Grade):} Bi-weekly participation in Canvas discussions and Discord conversations demonstrates engagement with course material and peers. Quality matters more than quantity---thoughtful, substantive contributions that advance the conversation earn full credit.

% Part c: Attendance and Participation
\vspace{1em}
\noindent \textbf{c. Attendance and Participation policy (5\% of Final Grade):}
As an asynchronous online course, ``attendance'' means regular engagement with course materials, timely submission of assignments, and active participation in discussions and peer reviews. Students should log into Canvas at least 3--4 times per week and complete all module activities by the weekly deadlines. Consistent engagement is essential for success in the scaffolded paper project.

% Part d: Examination dates
\vspace{1em}
\noindent \textbf{d. Examination dates:}
No traditional exams in this course. Assessment occurs through weekly module quizzes and the scaffolded writing project with integrated peer review.

% Part e: Make-up and late submission policy
\vspace{1em}
\noindent \textbf{e. Make-up and late submission policy:}
\begin{itemize}
\item \textbf{Written assignments:} 5\% penalty per day for late submissions unless prior arrangements are made
\item \textbf{Module quizzes:} Cannot be made up due to weekly release schedule (lowest score dropped to account for emergencies)
\item \textbf{Kritik peer reviews:} Have firm deadlines and cannot be made up---missing peer reviews results in significant grade reduction for that assignment
\item \textbf{Final paper:} No late submissions accepted without documented emergency
\end{itemize}

% Part f: Authentication of student work
\vspace{1em}
\noindent \textbf{f. Authentication of student work:}
All written work will be checked for originality using Turnitin (integrated with Canvas). Students may be asked to discuss their work via Zoom to verify authorship. The Kritik peer review process helps establish familiarity with each student's writing style throughout the semester.

% Part g: Extra credit
\vspace{1em}
\noindent \textbf{g. Extra credit:}
No extra credit offered. Focus your efforts on the assigned work, particularly the scaffolded paper project which comprises the majority of your grade.

% Part h: Retention of student work
\vspace{1em}
\noindent \textbf{h. Retention of student work:}
Keep copies of all submitted work. I may retain exemplary student work (anonymized) for future teaching purposes with student permission.

% ========== SECTION 6: ACADEMIC INTEGRITY ==========
\section{Academic Integrity}
Students are expected to adhere to the highest standards of academic integrity. Any student found to have engaged in academic dishonesty will be subject to the sanctions described in the \href{https://www.fullerton.edu/senate/publications_policies_resolutions/ups/UPS%20300/UPS%20300.021.pdf}{Academic Dishonesty Policy} (UPS 300.021). Academic dishonesty includes, but is not limited to, cheating, plagiarism, fabrication, facilitating academic dishonesty, and submitting previously graded work without prior authorization. Students are expected to be familiar with the university's policy on academic dishonesty and to adhere to this policy in all aspects of this course. Any student who has questions about the policy should ask the professor for clarification.

\subsection*{Plagiarism}
Plagiarism is a serious violation of academic integrity and will not be tolerated in this course. Plagiarism includes, but is not limited to, copying and pasting text from sources without proper citation, paraphrasing text from sources without proper citation, and submitting work that is not your own. Students are expected to properly cite all sources used in their work and to submit original work. Failure to do so may result in a failing grade for the assignment and further disciplinary action.

\subsection*{Written Work}
All written work must be submitted in a professional format, including proper grammar, spelling, and punctuation. Written work must also be properly cited using APA 7th edition format. Students are expected to follow the guidelines for written work provided by the professor and to seek clarification if they have questions about the requirements.

\subsection*{Artificial Intelligence (AI) Policy}

\subsubsection*{Definition of Generative AI}

\noindent For this course, generative AI refers to systems capable of producing human-like text, images, data analysis, or other content. Examples include:

\begin{itemize}
    \item Large Language Models (e.g., GPT-4, GPT-5, Claude, Gemini, TitanGPT)
    \item Text-to-image or multimodal generators (e.g., DALL-E, Midjourney)
    \item AI writing assistants and summarizers
    \item Automated coding, data, or content generators
\end{itemize}

\subsubsection*{AI Use Policy for the Policy Process Analysis Paper}

AI is permitted and encouraged as a learning tool, under the following guidelines:

\textbf{Allowed uses for the paper project:}
\begin{itemize}
    \item Brainstorming research questions and identifying potential sources
    \item Getting feedback on draft sections before submission
    \item Checking grammar, clarity, and APA formatting
    \item Generating questions to deepen your analysis
    \item Identifying potential gaps or weaknesses in your argument
\end{itemize}

\textbf{Prohibited uses:}
\begin{itemize}
    \item Having AI write any portion of your submitted work
    \item Using AI to paraphrase or rewrite sections of your paper
    \item Submitting AI-generated text as your own analysis
    \item Using AI to synthesize sources or create your literature review
\end{itemize}

\textit{All analysis, synthesis, and argumentation must be your own original work.}

\subsubsection*{Rationale for AI Policy}

This policy is designed to ensure AI use strengthens---not substitutes---your academic work:

\begin{enumerate}
    \item Promotes critical engagement with public policy theory by using AI as a feedback partner
    \item Enhances literature review and writing by highlighting missing connections or blind spots
    \item Builds professional literacy with tools already common in public service organizations
    \item Develops ethical judgment by practicing responsible use of emerging technologies
    \item Prepares you for professional environments where AI tools are increasingly prevalent
\end{enumerate}

\subsubsection*{Ethics and Responsible Use}

\noindent Students are expected to engage with AI responsibly:

\begin{itemize}
    \item \textbf{Authorship}: All substantive writing must be your own. AI may provide critique, but not draft or rewrite.
    \item \textbf{Citation}: When AI meaningfully informs your work, cite it (e.g., ``ChatGPT (GPT-5), personal communication, [date]'').
    \item \textbf{Bias Awareness}: AI outputs reflect biases. Evaluate them critically for fairness and accuracy.
    \item \textbf{Sustainability}: Be mindful of AI's environmental footprint and use tools thoughtfully.
\end{itemize}

\subsubsection*{Repercussions for Misuse}

\begin{itemize}
    \item Misuse includes submitting AI-generated work as your own, failing to cite AI contributions, or relying on AI instead of demonstrating your own analysis.
    \item Consequences may include revision requirements, grade penalties, or formal academic integrity proceedings.
    \item Kritik's AI detection tools will be used to identify potential misuse.
\end{itemize}

% ========== SECTION 7: TECHNICAL COMPETENCIES ==========
\section{Technical Competencies}
Students need:
\begin{itemize}
\item Proficiency with Canvas, including submitting assignments, participating in discussions, and accessing course materials
\item Ability to use Discord for class discussions and office hours
\item Experience with Kritik platform for peer review assignments (training provided in Week 1)
\item Skills in word processing, online research, and proper APA citation formatting
\item Ability to stream video content for module lectures and documentaries
\item Access to CSUF Writing Center for paper development support (optional but strongly recommended)
\item Time management skills for multi-stage writing project with firm peer review deadlines
\end{itemize}

% ========== SECTION 8: STUDENT RESOURCES WEBSITE ==========
\section{Student Resources Website}
It is the student's responsibility to read and understand the required and important \href{https://fdc.fullerton.edu/teaching/student-info-syllabi.html}{student information for course syllabi}. Included is information about:

\begin{itemize}
\item University learning goals
\item General Education learning objectives
\item Netiquette/appropriate online behavior
\item Students' rights to accommodations
\item Campus student support resources
\item Academic integrity
\item Emergency preparedness
\item Library services (including research consultations)
\item Student IT services and competencies
\item Software privacy and accessibility
\item Accessibility statement
\item Diversity statement
\item Land acknowledgement
\item Final exam schedule
\item Semester calendar
\end{itemize}

% ========== SECTION 9: COURSE POLICIES ==========
\section{Course Policies}

\noindent \textbf{Netiquette:} Maintain professional and respectful communication in all course interactions. Review CSUF's netiquette guidelines on the student resources website. This applies to Canvas discussions, Discord conversations, and Kritik peer reviews.

\vspace{0.5em}
\noindent \textbf{Module Schedule:} New modules release each Monday at 12:01 a.m. All module activities (videos, readings, quizzes, discussions) are due by Sunday at 11:59 p.m. Stay current with modules to succeed in the scaffolded paper project.

\vspace{0.5em}
\noindent \textbf{Kritik Peer Review Process:} 
\begin{itemize}
\item Each paper stage requires reviewing 3 peer submissions via Kritik
\item Reviews must include substantive feedback (minimum 150 words per review)
\item Use provided rubrics to evaluate specific criteria
\item Complete reviews within one week of submission deadline
\item Your peer review quality affects your assignment grade
\item Rate the helpfulness of feedback you receive to improve the process
\end{itemize}

\vspace{0.5em}
\noindent \textbf{Writing Support:} The CSUF Writing Center offers free consultations for all stages of the writing process. Schedule appointments at \href{http://www.fullerton.edu/writingcenter/}{fullerton.edu/writingcenter}. I strongly encourage using this resource, especially for the literature review and analysis stages.

\vspace{0.5em}
\noindent \textbf{Academic Support:} If you're struggling with course content or the paper project, please reach out early. Options include:
\begin{itemize}
\item Office hours (Tuesday mornings and evenings on Discord)
\item Individual appointments via \href{https://dadams.io/appointments}{dadams.io/appointments}
\item Discord \#qustions or \#315-public-policy channel for peer support
\item Canvas discussion forums for content questions
\end{itemize}

% ========== SECTION 10: GE REQUIREMENTS ==========
\section{General Education Requirements}
\noindent \textbf{GE requirement(s) that this course meets:} General Education Explorations in Social Sciences subarea 4U.

\vspace{0.5em}
\noindent \textbf{How the GE writing requirement will be met and assessed:}
The scaffolded Policy Process Analysis Paper with integrated peer reviews meets the writing requirement of UPS 411.201. Writing assignments involve organization and expression of complex data and ideas with careful evaluation and feedback at each stage. Writing competence determines 60\% of the final grade through the multi-stage paper project.

\vspace{0.5em}
\noindent \textbf{GE grading standard:}
A grade of ``D'' (1.0) or higher is required to meet this General Education requirement. A grade of ``D-'' (0.7) or below will not satisfy this General Education requirement.

\vspace{0.5em}
\noindent \textbf{GE Student Learning Goals:}
Students completing courses in this subarea shall:
\begin{enumerate}
\item Examine problems, issues, and themes in the social sciences in greater depth; in a variety of cultural, historical, and geographical contexts; and from different disciplinary and interdisciplinary perspectives
\item Analyze and critically evaluate the application of social science concepts and theories to particular historical, contemporary, and future problems or themes, such as economic and environmental sustainability, globalization, poverty, and social justice
\item Analyze and critically evaluate constructs of cultural differentiation, including ethnicity, gender, race, class, and sexual orientation, and their effects on the individual and society
\item Apply theories and concepts from the social sciences to address historical, contemporary, and future problems confronting communities at different geographical scales, from local to global
\end{enumerate}

% ========== SECTION 11: CALENDAR/SCHEDULE ==========
\section{Calendar of Topics / Schedule of Classes}

\noindent \textbf{Week 1, 8/25--8/31}\\
Topic(s): Introduction to the Policy Process\\
Reading(s): Syllabus; Birkland Ch. 1; American Regime Values (Canvas)\\
Assignment(s) Due: Introductions on Discord; Kritik platform setup and initial assignment; Module quiz\

\noindent \textbf{Week 2, 9/1--9/7}\\
Topic(s): Elements of the Policy-Making System; Policy Process Theories\\
Reading(s): Birkland Ch. 2; Theory readings on Canvas\\
Assignment(s) Due: Module quiz; Discussion post\\

\noindent \textbf{Week 3, 9/8--9/14}\\
Topic(s): Historical Context \& Structure of US Policy Making\\
Reading(s): Birkland Ch. 3\\
Assignment(s) Due: Module quiz; Discussion post\\

\noindent \textbf{Week 4, 9/15--9/21}\\
Topic(s): Official Policy Actors\\
Reading(s): Birkland Ch. 4\\
Assignment(s) Due: Module quiz; \textbf{Topic Selection \& Proposal (submit to Kritik by 11:59 PM Sunday)}\\

\noindent \textbf{Week 5, 9/22--9/28}\\
Topic(s): Unofficial Policy Actors I\\
Reading(s): Birkland Ch. 5\\
Assignment(s) Due: Module quiz; \textbf{Complete peer reviews of proposals on Kritik (by 11:59 PM Sunday)}\\

\noindent \textbf{Week 6, 9/29--10/5}\\
Topic(s): Unofficial Policy Actors II; Social Movements\\
Reading(s): Letter from Birmingham Jail (Canvas)\\
Assignment(s) Due: Module quiz; Discussion post\\

\noindent \textbf{Week 7, 10/6--10/12}\\
Topic(s): Documentary 1 \& Discussion\\
Reading(s): Documentary viewing guide (Canvas)\\
Assignment(s) Due: Documentary reflection; \textbf{Annotated Bibliography (submit to Kritik by 11:59 PM Sunday)}\\

\noindent \textbf{Week 8, 10/13--10/19}\\
Topic(s): Agenda Setting, Power, and Problem Definition\\
Reading(s): Birkland Ch. 6\\
Assignment(s) Due: Module quiz; \textbf{Complete peer reviews of bibliographies on Kritik (by 11:59 PM Sunday)}\\

\noindent \textbf{Week 9, 10/20--10/26}\\
Topic(s): Policy Typologies and Frameworks\\
Reading(s): Birkland Ch. 7\\
Assignment(s) Due: Module quiz; Discussion post\\

\noindent \textbf{Week 10, 10/27--11/2}\\
Topic(s): Policy Decision Making and Analysis\\
Reading(s): Birkland Ch. 8; Analysis readings (Canvas)\\
Assignment(s) Due: Module quiz; \textbf{Literature Review (submit to Kritik by 11:59 PM Sunday)}\\

\noindent \textbf{Week 11, 11/3--11/9}\\
Topic(s): Policy Design and Tools\\
Reading(s): Birkland Ch. 9\\
Assignment(s) Due: Module quiz; \textbf{Complete peer reviews of literature reviews on Kritik (by 11:59 PM Sunday)}\\

\noindent \textbf{Week 12, 11/10--11/16}\\
Topic(s): Policy Implementation\\
Reading(s): Birkland Ch. 10 (first half)\\
Assignment(s) Due: Module quiz; Discussion post\\

\noindent \textbf{Week 13, 11/17--11/23}\\
Topic(s): Policy Failure and Learning\\
Reading(s): Birkland Ch. 10 (second half)\\
Assignment(s) Due: Module quiz; \textbf{Draft Analysis Paper (submit to Kritik by 11:59 PM Sunday)}\\

\noindent \textbf{Thanksgiving Break, 11/24--11/30}\\
No new content---use this time to complete peer reviews and begin final paper revisions\\

\noindent \textbf{Week 14, 12/1--12/7}\\
Topic(s): Documentary 2 \& Discussion\\
Reading(s): Documentary viewing guide (Canvas)\\
Assignment(s) Due: Documentary reflection; \textbf{Complete peer reviews of draft papers on Kritik (by 11:59 PM Sunday)}\\

\noindent \textbf{Week 15, 12/8--12/14}\\
Topic(s): Policy Evaluation and Science\\
Reading(s): Birkland Ch. 11\\
Assignment(s) Due: Module quiz; Final paper preparation\\

\noindent \textbf{Week 16, 12/15--12/19}\\
Topic(s): Course Wrap-up and Synthesis\\
Reading(s): None\\
Assignment(s) Due: \textbf{Final Policy Process Analysis Paper due Thursday 12/18 at 11:59 PM (submit to Canvas)}\\

\section{Why This Course Matters}

  We live in an era when deep divisions shape not just our politics but our perceptions of reality itself. Nearly two-thirds of Americans (65\%) say they ``always or often feel exhausted'' by politics (Pew Research Center 2023). Self-identified moderates, once the largest group, have dropped to a record low of just 34\% (Gallup 2024). Meanwhile, unfavorable views of the opposing party have more than doubled since the 1990s, reaching record highs of “very unfavorable” sentiment (Pew Research Center; Gallup). James Madison warned in Federalist 10 of the “violence of faction,” and Montesquieu's vision of checks and balances was meant to prevent any single group from dominating. Yet today, partisan animosity runs so deep that many Americans see political opponents not merely as wrong, but as a threat to the republic itself. Alexis de Tocqueville once marveled at Americans' capacity to build associations and bridge differences—a civic muscle that now appears strained. And yet, even amid this polarization, the machinery of governance continues to turn: problems still become laws, programs are implemented, and occasionally, genuine solutions emerge from the gridlock.

  This course equips you with the analytical tools to see beyond the doom-scrolling and cable news shouting matches. You'll discover that polarization isn't destiny; it's a policy problem that can be understood through frameworks like path dependence and punctuated equilibrium. FDR faced a nation literally coming apart during the Depression, yet managed to fundamentally reshape the relationship between citizens and government through savvy use of policy windows and coalition building. The same processes he mastered---agenda setting, problem definition, policy entrepreneurship---still operate today, just waiting for those who understand them. Through your research project, you'll trace how real policies navigate this polarized landscape, learning to spot the moments when change becomes possible. You'll understand why some issues break through the gridlock while others languish, how policy entrepreneurs exploit crises to push long-dormant solutions, and why implementation often matters more than legislation. This isn't just academic theory; it's a toolkit for citizenship in troubled times. Whether you're headed for public service or simply want to be an informed participant in democracy, this course teaches you to read the hidden grammar of American politics---to see the republic not as hopelessly broken, but as a complex system that citizens who understand it can still influence.


\end{document}