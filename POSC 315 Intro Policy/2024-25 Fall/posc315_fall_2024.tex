% POSC 315 - Syllabus - Online Asynchronous - Fall 2024

\documentclass[12pt, letterpaper]{article}
\usepackage[margin=1in]{geometry}
\usepackage{xcolor}
\usepackage[colorlinks=true, linkcolor=blue, urlcolor=blue, citecolor=blue]{hyperref}
\usepackage{url}
\usepackage{graphicx}
\usepackage{bookmark}
\usepackage{enumitem}
\frenchspacing
\usepackage{titlesec}
\titleformat{\section}{\normalfont\Large\bfseries}{\thesection.}{1em}{}
\usepackage[scaled]{helvet}
\renewcommand\familydefault{\sfdefault}

\begin{document}
\title{\includegraphics[width=6cm]{/Users/dadams/repositories/syllabi/Images/stacked.png} \\\vspace{1ex} \textbf{Introduction to Public Policy}}

\author{POSC 315-02 -- Fall 2024}
\date{Tuesdays and Thursdays at 11:30 in MH 554 \\\vspace{2ex} \textsc{August 26 -- December 13, 2024}}

    \maketitle


\subsection*{Professor: David P. Adams, Ph.D.}

\subsubsection*{Contact Information:}

\begin{itemize}
	\item Office: 516 Gordon Hall
	\item Phone/Text: \href{tel:+16572784770}{(657) 278-4770}
	\item Zoom: \href{https://fullerton.zoom.us/j/3347502369}{\texttt{https://fullerton.zoom.us/j/3347502369}}
	\item website: \href{https://dadams.io}{\texttt{dadams.io}}
	\item email: \href{dpadams@fullerton.edu}{\texttt{dpadams@fullerton.edu}}
	\item Office Hours:
        \begin{itemize}
            \item Tuesdays and Thursdays: 9:30 -- 11:00 a.m.; Thursdays: 5:00pm -- 6:00 p.m.
            \item Schedule meetings throughout the week: \href{https://dadams.io/appt}{\texttt{dadams.io/appt}}
        \end{itemize}  
\end{itemize}


\section{Catalog Description}

	Federal domestic policymaking. Structure, functions, and relationships among American national institutions, including executive, legislative and judicial branches, media, political parties, and pressure groups.

\section{Course Description}

	In this course, students will explore and engage in thoughtful discussions on the processes and key players in creating public policy in the United States. The curriculum focuses on the structure, functions, and relationships among American national institutions, including the executive, legislative, and judicial branches of government, the media, political parties, and interest groups. We will examine the various official and unofficial influences on the policy process and the limitations imposed by institutional and structural factors.

    \vspace{1ex}

	\noindent This course delves into the historical and constitutional development of the policy process, as well as the distinct characteristics of public policy within a federal system. By understanding policy-making in the context of a constitutional republic and a federalized governance system, students will gain a deeper appreciation for the complexities and uncertainties surrounding agenda setting, policy making, policy implementation, and policy evaluation in the American political system.

\section{Student Learning Objectives}

By the end of this course, students will be able to

\begin{enumerate}
	\item discuss and explain the key features of the public policy-making process in the United States;
	\item recognize and describe the distinct stages of the public policy process;
	\item describe the various internal and external actors that influence public policy, their interactions, and their impact on the policy process;
	\item articulate the historical and contemporary structures and institutions that facilitate, expand, or constrain the public policy process;
	\item differentiate and describe the various theories that attempt to explain the drivers and influences leading to public policy change or maintaining the status quo; and
	\item apply their knowledge of the policy process to analyze specific policy domains impacted by multiple policy actors and diverse elements of the policy process within the context of public policy-making in the United States.
\end{enumerate}


\section{General Education Information}

\subsection*{Requirements Satisfied}

	This course satisfies General Education Explorations in Social Sciences subarea D.5. The writing assignments in this course, including the policy memo papers and current event summaries described below, meet the requirement of UPS 411.201: 
	\begin{quote}Writing assignments in General Education courses shall involve the organization and expression of complex data or ideas and careful and timely evaluations of writing so that deficiencies are identified, and suggestions for improvement and/or for means of remediation are offered. Evaluations of the student's writing competence shall determine the final course grade\ldots .\end{quote}
    A grade of “D” (1.0) or higher is required to meet this General Education requirement. A grade of “D-“ (0.7) or below will not satisfy this General Education requirement.

\subsection*{General Education Student Learning Goals}

	Students completing courses in this subarea shall encounter the following learning goals:

\begin{enumerate}
	\item Examine problems, issues, and themes in the social sciences in greater depth; in a variety of cultural, historical, and geographical contexts; and from different disciplinary and interdisciplinary perspectives.
	\item Analyze and critically evaluate the application of social science concepts and theories to particular historical, contemporary, and future problems or themes, such as economic and environmental sustainability, globalization, poverty, and social justice.
	\item Analyze and critically evaluate constructs of cultural differentiation, including ethnicity, gender, race, class, and sexual orientation, and their effects on the individual and society.
	\item Apply theories and concepts from the social sciences to address historical, contemporary, and future problems confronting communities at different geographical scales, from local to global.
\end{enumerate}

\section{Text Books}


\paragraph{}\noindent \textbf{Required}: Kraft, Michael E. and Scott R. Furlong. 2025. \emph{Public Policy: Politics, Analysis, and Alternatives}. 8th ed. CQ Press: Thousand Oaks, Calif.

\paragraph{}\noindent \emph{Recommended}: Smith, Kevin B. and Christopher W. Larimer. 2017. \emph{The Public Policy Theory Primer}. 3rd ed. Westview Press: Boulder, Colo.

\section{Technical Competencies}

Students are expected to have the following technical competencies to succeed in this course:
\begin{itemize}
    \item Basic computer skills, including the ability to navigate the internet, use email, and create and save documents.
    \item Proficiency in using \emph{Canvas}, including submitting assignments, participating in discussions, and accessing course materials.
    \item Ability to use word processing software, such as Microsoft Word or Google Docs, to create and format documents.
    \item Access to a reliable computer and internet connection.
    \item Ability to use Zoom for virtual meetings and discussions.
    \item Basic knowledge of online communication tools, such as email and discussion boards.
    \item Ability to use online research tools and databases to find academic sources.
\end{itemize}

\section{Technical Problems}

\subsection*{University IT Help Desk}

Contact the instructor immediately to document the problem if you encounter any technical difficulties. Then contact the \href{http://www.fullerton.edu/it/students/helpdesk/index.php}{Student IT Help Desk} for assistance. You can also call the Student IT Help Desk at \href{tel:+16572788888}{(657) 278-8888}, \href{mailto:StudentITHelpDesk@fullerton.edu}{email}, visit them at the Pollak Library North \href{http://www.fullerton.edu/it/students/sgc/index.php}{Student Genius Center}, or log on to the \href{http://my.fullerton.edu/}{my.fullerton.edu} portal and click ``Online IT Help'' followed by ``Live Chat''.

\subsection*{Canvas Support}

If you encounter any technical difficulties with Canvas, call the Canvas Support Hotline at \href{tel:+18553027528}{855-302-7528}, visit the \href{https://community.canvaslms.com/docs/DOC-10720-67952720329}{Canvas Community}, or click the ``Help'' button in the lower left corner of Canvas and select ``Report a Problem''. The \href{https://cases.canvaslms.com/liveagentchat?chattype=student&sfid=001A000000YzcwQIAR}{Student Support Live Chat} is available 24 hours a day, 7 days a week.

\section{Response Time} I will strive to respond to all student emails, Discord posts, and \emph{Canvas} messages within 24 hours, except on weekends and holidays. If you are still awaiting a response within 24 hours, please send a follow-up message. If you are still waiting to receive a response within 48 hours, please send another follow-up message and contact me via phone or text at \href{tel:+16572784770}{(657) 278-4770}.


\section{University Student Policies}

In accordance with UPS 300.00, students must be familiar with certain policies applicable to all courses. Please review these policies as needed and visit this Cal State Fullerton website \texttt{\href{https://fdc.fullerton.edu/teaching/student-info-syllabi.html}{https://fdc.fullerton.edu/teaching/student-info-syllabi.html}} for links to the following information:

\begin{enumerate}
    \item   University learning goals and program learning outcomes.
    \item	Learning objectives for each General Education (GE) category.
    \item	Guidelines for appropriate online behavior (netiquette).
    \item	Students' rights to accommodations for documented special needs.
    \item   Campus student support measures, including Counseling \& Psychological Services, Title IV and Gender Equity, Diversity Initiatives and Resource Centers, and Basic Needs Services.
    \item   Disability Support Services (DSS) information.
    \item	Academic integrity (refer to UPS 300.021).
    \item	Actions to take during an emergency.
    \item	Library services information.
    \item	Student Information Technology Services, including details on technical competencies and resources required for all students.
    \item	Software privacy and accessibility statements.
\end{enumerate}

\section{Course Student Policies}

\subsection*{Course Communication}
All course announcements and communications will be sent via \emph{Canvas} and university email. Students are responsible for regularly checking their \emph{Canvas} notifications and email. Students are also responsible for ensuring that their \emph{Canvas} notifications are set to receive messages from the course. Students are expected to check \emph{Canvas} and their email at least once daily.

\subsection*{Due Dates}
All assignments are due by 11:59 p.m. on the specified due date. Save for extenuating circumstances, late assignments will not be accepted unless prior arrangements have been made with the professor. Students are responsible for submitting assignments on time. Failure to submit an assignment on time may result in a failing grade for the assignment. 

\subsection*{Alternative Procedures for Submitting Work}
Students are expected to submit all assignments via \emph{Canvas}. If you cannot submit an assignment via \emph{Canvas}, please get in touch with the professor to discuss alternative submission procedures.

\subsection*{Retention of Student Work}
Students are responsible for retaining copies of all assignments submitted for this course. Students are also responsible for retaining copies of all graded assignments returned by the professor.

\subsection*{Extra Credit}
There is one bonus assignment available in this course. A documentary analysis assignment will be available in Week 15. The assignment is optional and will provide extra credit points on the final exam. The assignment will be graded based on the quality of the analysis and the depth of engagement with the course material.

\subsection*{Academic Integrity}
Students are expected to adhere to the highest standards of academic integrity. Any student found to have engaged in academic dishonesty will be subject to the sanctions described in the \href{https://www.fullerton.edu/senate/publications_policies_resolutions/ups/UPS%20300/UPS%20300.021.pdf}{Academic Dishonesty Policy} (UPS 300.021). Academic dishonesty includes, but is not limited to, cheating, plagiarism, fabrication, facilitating academic dishonesty, and submitting previously graded work without prior authorization. Students are expected to be familiar with the university's policy on academic dishonesty and to adhere to this policy in all aspects of this course. Any student who has questions about the policy should ask the professor for clarification.

\subsection*{Written Work}
All written work must be submitted in a professional format, including proper grammar, spelling, and punctuation. Written work must also be properly cited using the appropriate citation style. Students are expected to follow the guidelines for written work provided by the professor and to seek clarification if they have questions about the requirements.

\subsection*{Plagiarism}
Plagiarism is a serious violation of academic integrity and will not be tolerated in this course. Plagiarism includes, but is not limited to, copying and pasting text from sources without proper citation, paraphrasing text from sources without proper citation, and submitting work that is not your own. Students are expected to properly cite all sources used in their work and to submit original work. Failure to do so may result in a failing grade for the assignment and further disciplinary action. Written work will be checked for plagiarism using plagiarism detection software.

\subsection*{AI-Generated Text and Tool Usage Policy}

\subsubsection*{Permissible Use of AI Tools}
\textbf{Definition of AI Tools:} In this course, AI tools refer to any software or platform that generates or assists in generating text, ideas, research references, or content creation. This includes, but is not limited to, large language models like OpenAI’s ChatGPT-4, Anthropic’s Claude, AI-based research tools like RefWorks or EndNote, and writing assistants like Grammarly.

\vspace{1ex}

\noindent\textbf{Permissible Use Cases:} Students are allowed to use AI tools for brainstorming, generating ideas, checking grammar, and summarizing content. However, AI tools should not be used to produce final drafts of assignments or significant portions of text that are submitted as original work. If AI tools are used, their contributions must be appropriately cited and disclosed.

\subsubsection*{Ethical Considerations}
\textbf{Academic Integrity:} The use of AI tools must adhere to the highest standards of academic integrity. This means that while these tools can support your work, they cannot replace your own analysis, critical thinking, and original writing. Misuse of AI-generated content is a form of academic dishonesty and will be treated as such under the university's policies.

\vspace{1ex}

\noindent\textbf{Required Disclosures:} Students must disclose their use of AI tools in any assignments. This disclosure should include a brief explanation of how the tool was used and how the student ensured the integrity of their work. For example, a note might state, “I used ChatGPT to generate initial ideas for this essay, but all writing and analysis are my own.”

\subsubsection*{Plagiarism and Originality}
\textbf{AI Detection Tools:} Assignments will be checked for originality using advanced AI detection software. Any submission found to contain unoriginal content generated by AI without proper citation will be subject to the university's academic dishonesty policy. Submitting AI-generated text as your own work, without attribution, is considered plagiarism.

\vspace{1ex}

\noindent\textbf{Collaboration and AI Use:} While collaboration is encouraged in this course, any use of AI tools must be clearly documented, and all collaborators must agree on the use and citation of AI-generated content. Collaborative work that involves AI tools must be transparently disclosed in the submission.

\subsubsection*{Guidance and Resources}
\textbf{Ethical AI Usage Resources:} Students are encouraged to take advantage of available resources on responsible AI usage. This includes online tutorials, ethical guidelines, and citation practices. Workshops or additional materials on how to use AI tools responsibly may be provided throughout the course.

\vspace{1ex}

\noindent\textbf{Continuous Policy Review:} This AI policy will be reviewed and updated regularly to keep pace with technological advancements. Students are encouraged to provide feedback on this policy to ensure it aligns with their educational goals and the course's academic standards.

\subsubsection*{Example Scenarios}
\textbf{Acceptable Use:} A student uses Grammarly to check the grammar and clarity of their essay. The student does not need to disclose this use unless Grammarly significantly altered the content. Another student uses ChatGPT to brainstorm ideas but writes the essay independently, only citing the AI where directly quoted or paraphrased.

\vspace{1ex}

\noindent\textbf{Unacceptable Use:} A student uses ChatGPT to generate an entire essay and submits it as their own work without attribution. This would be considered a violation of the academic integrity policy and could result in a failing grade or further disciplinary action.

\subsection*{Participation}

Students are expected to participate in all course activities. This includes completing all assigned readings, watching all assigned videos, and participating in all discussions. Students are expected to participate in discussions in a professional and respectful manner. Students are expected to be familiar with the university policy on netiquette and to adhere to this policy in all aspects of this course. Any student who has questions about the policy should ask the professor for clarification. 

\subsection*{Netiquette}
Students are expected to adhere to the university's policy on netiquette. The university's policy on netiquette is as follows:
\begin{quote}Netiquette refers to a set of behaviors that are appropriate for online activity (e.g., social media, email, discussions, presentations). All personnel at Cal State Fullerton are expected to demonstrate appropriate online behavior at all times. A good summary of netiquette can be found in the \href{https://canvashelp.fullerton.edu/m/Student/l/1336786-student-what-is-netiquette}{CSUF Canvas self-help guides}, which adapts ten rules to the online course situation from the website for the book \href{http://www.albion.com/netiquette/corerules.html}{Netiquette by Virginia Shea} and other sources referenced at the bottom of the guide.\end{quote}


\section{Kritik: Peer Review Platform}

\noindent This course will use the peer review platform \href{https://kritik.io/}{Kritik} for peer review assignments. Kritik is an online platform that allows students to provide feedback on their classmates' work. Students will be assigned to review the work of their peers and provide constructive feedback. Students will also receive feedback from their peers on their own work. The professor will use Kritik to monitor the peer review process and provide guidance as needed.

\subsection*{Kritik Assignments}
\noindent The Kritik platform will be used for the weekly syntheses and the concentration area literature review assignments. Students are expected to familiarize themselves with the platform and use it to complete the peer review assignments. The professor will provide guidance and support as needed to ensure that students can use Kritik effectively.

\subsection*{Kritik Overview}

\begin{itemize}

\item \textbf{A Three-Stage Writing Process:}

\begin{enumerate}
    \item Craft Your Analysis: Follow the provided rubric and delve into a public policy challenge. This could involve, for example, evaluating the ethical implications of a proposed environmental regulation or assessing the effectiveness of a social welfare program.
    \item Provide Constructive Critique: Anonymously evaluate your peers' work using the rubric. Offer actionable feedback that focuses on the strengths and weaknesses of their analysis, drawing connections to relevant public administration concepts.
    \item Reflect and Improve: Receive anonymous feedback on the quality and impact of your comments. Learn to deliver clear, concise, and impactful feedback—a crucial skill for public servants collaborating on complex issues.
\end{enumerate}

\item \textbf{Grading and Participation:} You'll earn four scores for each Kritik activity: Creation, Evaluation, Feedback, and Overall. These scores, along with active participation, will contribute to your course grade. Participating thoughtfully in Kritik activities will not only improve your own skills but also enrich the learning experience for your peers.

\item \textbf{Registration and Support:} We'll thoroughly introduce Kritik in class, and a dedicated email invitation will guide you through registration and course enrollment. The Kritik Help Center offers additional resources, and I'm always available to address any questions or concerns.
\end{itemize}
\section{Course Assignment Descriptions}
\subsection*{Discussion Posts (20\% of Final Grade)} 
\begin{itemize}
    \item \textbf{Description}: \emph{Students will participate in \underline{six of the eight} online discussion boards}, responding to prompts related to the current week's readings and lectures. Prompts may involve analyzing case studies, applying theoretical concepts to current events, or debating policy issues.
    \item \textbf{Expectations}: Students are expected to post a response to the prompt and engage with their classmates' posts. Discussion posts are due by 11:59 p.m. on Saturdays for the week they are assigned. Late posts will not be accepted.
    \item \textbf{Objectives}: Engage with course material, apply theoretical concepts to real-world scenarios, and develop critical thinking skills.
    \item \textbf{Grading Criteria}: Posts will be graded based on the quality of the response, depth of analysis, timeliness of contributions, and engagement with classmates' posts.
    \item \textbf{Skipped Posts}: Students are expected to complete six of the eight discussion posts. Indicate a skipped post by submitting a blank post with the subject line "Skipped Post."
\end{itemize}

\subsection*{Policy Process Analysis Paper (35\% of Final Grade)}

This course includes a scaffolded and peer-reviewed writing project where students will analyze a specific policy issue through the lens of policy process theories. The project will be completed in stages, with opportunities for peer review and feedback at each stage. The final paper will be a 10-12 page policy process analysis paper that applies policy process theories to understand the selected policy issue's development, implementation, and/or impact. Kritik will be used for out to submit drafts and provide feedback to peers.

\subsubsection*{Project Stages}
\begin{enumerate}
    \item \textbf{Topic Selection and Proposal (Week 3)}
    \begin{itemize}
        \item Submit a one-page proposal outlining the policy issue, its significance, and a preliminary research question focusing on the policy process (e.g., agenda-setting, formulation, implementation, evaluation).
        \item Participate in peer reviews of proposals.
    \end{itemize}
    \item \textbf{Annotated Bibliography (Week 6)}
    \begin{itemize}
        \item Compile an annotated bibliography of 8-10 sources, summarizing and evaluating each source's relevance to your research question and its relation to policy process theories.
        \item Exchange and review peers' bibliographies.
    \end{itemize}
    \item \textbf{Literature Review (Week 9)}
    \begin{itemize}
        \item Write a 3-4 page literature review synthesizing key findings, identifying gaps in the research, and situating your analysis within the broader context of policy process theories.
        \item Review and comment on peers' literature reviews.
    \end{itemize}
    \item \textbf{Draft Analysis (Week 12)}
    \begin{itemize}
        \item Submit a 6-8 page draft analysis, including an introduction, literature review, theoretical framework, and preliminary conclusions on how policy process theories explain the issue.
        \item Conduct detailed peer reviews using a provided rubric.
    \end{itemize}
    \item \textbf{Final Paper (Week 15)}
    \begin{itemize}
        \item Submit a final 10-12 page policy process analysis paper, incorporating peer feedback and additional research or revisions. The paper should apply policy process theories to understand the selected policy issue's development, implementation, and/or impact.
    \end{itemize}
\end{enumerate}

\paragraph{} This project will help you develop your research, analytical, evaluative, and writing skills while providing a deep understanding of policy theory and the policy process. The project will also provide opportunities for peer review and feedback, enhancing your ability to provide and receive constructive criticism. 

\subsection*{Midterm Exam (20\% of Final Grade)}
An in-class midterm exam will be held in Week 8. The exam will cover material from the first half of the course, including readings, lectures, and discussions. The exam will consist of multiple-choice, short answer, and essay questions.

\subsection*{Final Exam (25\% of Final Grade)}
An in-class final exam will be held during the scheduled final exam period. The final exam will be comprehensive and cover all material from the course, including readings, lectures, discussions, and assignments. The final exam will consist of multiple-choice, short answer, and essay questions.

\section{Grading}
Grades will be allocated based on the following components:
\begin{itemize}
    \item \textbf{Discussion Posts}: 20\%
    \item \textbf{Policy Analysis Paper}: 35\%
    \item \textbf{Midterm Exam}: 20\%
    \item \textbf{Final Exam}: 25\%    
\end{itemize}

\subsection*{Grading Scale}

\begin{itemize}
    \item A+: 97.00 -- 100.00
    \item A: 93.00 -- 96.99
    \item A-: 90.00 -- 92.99
    \item B+: 87.00 -- 89.99
    \item B: 83.00 -- 86.99
    \item B-: 80.00 -- 82.99
    \item C+: 77.00 -- 79.99
    \item C: 73.00 -- 76.99
    \item C-: 70.00 -- 72.99
    \item D+: 67.00 -- 69.99
    \item D: 63.00 -- 66.99
    \item D-: 60.00 -- 62.99
    \item F: 0.00 -- 59.99
\end{itemize}

\section{Course Schedule}

\begin{itemize}
    \item \textbf{Week 1: August 27 and August 29}
    \begin{itemize}
        \item \textbf{Topic:} Introduction to Public Policy
        \item \textbf{Readings:} Smith \& Larimer, Chapter 1; Kraft \& Furlong, Chapter 1
        \item \textbf{Discussion:} Why Study Public Policy?
    \end{itemize}

    \item \textbf{Week 2: September 3 and September 5}
    \begin{itemize}
        \item \textbf{Topic:} Government Institutions and Policy Actors
        \item \textbf{Readings:} Kraft \& Furlong, Chapter 2; Smith \& Larimer, Chapter 3
        \item \textbf{Discussion:} Decision-Making Models and Institutional Rational Choice
    \end{itemize}

    \item \textbf{Week 3: September 10 and September 12}
    \begin{itemize}
        \item \textbf{Topic:} Understanding Public Policymaking
        \item \textbf{Readings:} Kraft \& Furlong, Chapter 3; Smith \& Larimer, Chapter 5
        \item \textbf{Assignment:} Topic Selection and Proposal Due
    \end{itemize}

    \item \textbf{Week 4: September 17 and September 19}
    \begin{itemize}
        \item \textbf{Topic:} Policy Design and Typologies
        \item \textbf{Readings:} Smith \& Larimer, Chapter 4; Kraft \& Furlong, Chapter 3
        \item \textbf{Discussion:} Policy Typologies as Analytic Tools
    \end{itemize}

    \item \textbf{Week 5: September 24 and September 26}
    \begin{itemize}
        \item \textbf{Topic:} Policy Analysis Introduction
        \item \textbf{Readings:} Kraft \& Furlong, Chapter 4; Smith \& Larimer, Chapter 6
        \item \textbf{Discussion:} Rationalist and Post-Positivist Approaches
    \end{itemize}

    \item \textbf{Week 6: October 1 and October 3}
    \begin{itemize}
        \item \textbf{Topic:} Impact Analysis and Program Evaluation
        \item \textbf{Readings:} Smith \& Larimer, Chapter 7; Kraft \& Furlong, Chapter 6
        \item \textbf{Assignment Due:} Annotated Bibliography 
    \end{itemize}

    \item \textbf{Week 7: October 8 and October 10}
    \begin{itemize}
        \item \textbf{Topic:} Policy Implementation
        \item \textbf{Readings:} Smith \& Larimer, Chapter 8; Kraft \& Furlong, Chapter 6
        \item \textbf{Discussion:} Implementation Analysis and Challenges
    \end{itemize}

    \item \textbf{Week 8: October 15 and October 17}
    \begin{itemize}
        \item \textbf{Topic:} Midterm Review
        \item \textbf{Exam:} Midterm Exam
    \end{itemize}

    \item \textbf{Week 9: October 22 and October 24}
    \begin{itemize}
        \item \textbf{Topic:} Economic and Budgetary Policy
        \item \textbf{Readings:} Kraft \& Furlong, Chapter 7
        \item \textbf{Assignment:} Literature Review Due
    \end{itemize}

    \item \textbf{Week 10: October 29 and October 31}
    \begin{itemize}
        \item \textbf{Topic:} Social and Environmental Policy
        \item \textbf{Readings:} Kraft \& Furlong, Chapters 8 and 11
    \end{itemize}

    \item \textbf{Week 11: November 5 and November 7}
    \begin{itemize}
        \item \textbf{Topic:} Policy Change and Innovation
        \item \textbf{Readings:} Smith \& Larimer, Chapter 9
        \item \textbf{Discussion:} Policy Change and Innovation
    \end{itemize}

    \item \textbf{Week 12: November 12 and November 14}
    \begin{itemize}
        \item \textbf{Topic:} Foreign Policy and Homeland Security
        \item \textbf{Readings:} Kraft \& Furlong, Chapter 12
        \item \textbf{Assignment:} Draft Analysis Due
    \end{itemize}

    \item \textbf{Week 13: November 19 and November 21}
    \begin{itemize}
        \item \textbf{Topic:} Policy Conflicts and Strategies
        \item \textbf{Readings:} Kraft \& Furlong, Chapter 13
        \item \textbf{Discussion:} Policy Conflicts and Strategies
    \end{itemize}

    \item \textbf{Break}: November 26 and November 28
    \begin{itemize}
        \item \textbf{Note:} Thanksgiving Break (No Class on November 26 and 28)
    \end{itemize}

    \item \textbf{Week 14: December 3 and December 5}
    \begin{itemize}
        \item \textbf{Topic:} Class Wrap-Up and Bonus Documentary Screening 
        \item \textbf{Discussion:} Course Reflections and Final Thoughts
    \end{itemize}

    \item \textbf{Week 15: December 10 and December 12}
    \begin{itemize}
        \item \textbf{Topic:} Final Exam Review
        \item \textbf{Assignment:} Final Paper Due
    \end{itemize}

    \item \textbf{Week 16: December 17}
    \begin{itemize}
        \item \textbf{Final Exam: 11:00 - 12:50}
        \item \textbf{Bonus Assignment:} Optional Documentary Analysis Due

    \end{itemize}
\end{itemize}


\end{document}
