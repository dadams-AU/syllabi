% POSC 315 - Syllabus - Online Asynchronous - Fall 2024

\documentclass[12pt, letterpaper]{article}
\usepackage[margin=1in]{geometry}
\usepackage{xcolor}
\usepackage[colorlinks=true, linkcolor=blue, urlcolor=blue, citecolor=blue]{hyperref}
\usepackage{url}
\usepackage{graphicx}
\usepackage{bookmark}
\usepackage{enumitem}
\frenchspacing

\usepackage[scaled]{helvet}
\renewcommand\familydefault{\sfdefault}

\begin{document}
\title{\textbf{Introduction to Public Policy}}

\author{POSC 315 -- Fall 2024}
\date{Asynchronous Online Course \\ \textsc{August 26 -- December 13, 2024}}

    \maketitle


\subsection*{Professor: David P. Adams, Ph.D.}

\subsubsection*{Contact Information:}

\begin{itemize}
	\item Office: 516 Gordon Hall
	\item Phone/Text: \href{tel:+16572784770}{(657) 278-4770}
	\item Zoom: \href{https://fullerton.zoom.us/j/3347502369}{\texttt{https://fullerton.zoom.us/j/3347502369}}
	\item website: \href{https://dadams.io}{\texttt{https://dadams.io}}
	\item email: \href{dpadams@fullerton.edu}{\texttt{dpadams@fullerton.edu}}
	\item Office Hours:
        \begin{itemize}
            \item Monday: 10:00am-12:00pm on Zoom
            \item Schedule meetings throughout the week: \href{https://dadams.io/appt}{\texttt{https://dadams.io/appt}}
        \end{itemize}  
\end{itemize}


\section*{Catalog Description}

	Federal domestic policymaking. Structure, functions, and relationships among American national institutions, including executive, legislative and judicial branches, media, political parties, and pressure groups.

\section*{Course Description}

	In this course, students will explore and engage in thoughtful discussions on the processes and key players in creating public policy in the United States. The curriculum focuses on the structure, functions, and relationships among American national institutions, including the executive, legislative, and judicial branches of government, the media, political parties, and interest groups. We will examine the various official and unofficial influences on the policy process and the limitations imposed by institutional and structural factors.

	This course delves into the historical and constitutional development of the policy process, as well as the distinct characteristics of public policy within a federal system. By understanding policy-making in the context of a constitutional republic and a federalized governance system, students will gain a deeper appreciation for the complexities and uncertainties surrounding agenda setting, policy making, policy implementation, and policy evaluation in the American political system.

\section*{Student Learning Objectives}

By the end of this course, students will be able to

\begin{enumerate}
	\item discuss and explain the key features of the public policy-making process in the United States;
	\item recognize and describe the distinct stages of the public policy process;
	\item describe the various internal and external actors that influence public policy, their interactions, and their impact on the policy process;
	\item articulate the historical and contemporary structures and institutions that facilitate, expand, or constrain the public policy process;
	\item differentiate and describe the various theories that attempt to explain the drivers and influences leading to public policy change or maintaining the status quo; and
	\item apply their knowledge of the policy process to analyze specific policy domains impacted by multiple policy actors and diverse elements of the policy process within the context of public policy-making in the United States.
\end{enumerate}


\section*{General Education Information}

\subsection*{Requirements Satisfied}

	This course satisfies General Education Explorations in Social Sciences subarea D.5. The writing assignments in this course, including the policy memo papers and current event summaries described below, meet the requirement of UPS 411.201: 
	\begin{quote}Writing assignments in General Education courses shall involve the organization and expression of complex data or ideas and careful and timely evaluations of writing so that deficiencies are identified, and suggestions for improvement and/or for means of remediation are offered. Evaluations of the student's writing competence shall determine the final course grade\ldots .\end{quote}
    A grade of “D” (1.0) or higher is required to meet this General Education requirement. A grade of “D-“ (0.7) or below will not satisfy this General Education requirement.

\subsection*{General Education Student Learning Goals}

	Students completing courses in this subarea shall encounter the following learning goals:

\begin{enumerate}
	\item Examine problems, issues, and themes in the social sciences in greater depth; in a variety of cultural, historical, and geographical contexts; and from different disciplinary and interdisciplinary perspectives.
	\item Analyze and critically evaluate the application of social science concepts and theories to particular historical, contemporary, and future problems or themes, such as economic and environmental sustainability, globalization, poverty, and social justice.
	\item Analyze and critically evaluate constructs of cultural differentiation, including ethnicity, gender, race, class, and sexual orientation, and their effects on the individual and society.
	\item Apply theories and concepts from the social sciences to address historical, contemporary, and future problems confronting communities at different geographical scales, from local to global.
\end{enumerate}

\section*{Text Book}

Kraft, Michael E. and Scott R. Furlong. 2025. \emph{Public Policy: Politics, Analysis, and Alternatives}. 8th ed. CQ Press.

\section*{Technical Competencies}

Students are expected to have the following technical competencies to succeed in this course:
\begin{itemize}
    \item Basic computer skills, including the ability to navigate the internet, use email, and create and save documents.
    \item Proficiency in using \emph{Canvas}, including submitting assignments, participating in discussions, and accessing course materials.
    \item Ability to use word processing software, such as Microsoft Word or Google Docs, to create and format documents.
    \item Access to a reliable computer and internet connection.
    \item Ability to use Zoom for virtual meetings and discussions.
    \item Basic knowledge of online communication tools, such as email and discussion boards.
    \item Ability to use online research tools and databases to find academic sources.
\end{itemize}

\section*{Technical Problems}

\subsection*{University IT Help Desk}

Contact the instructor immediately to document the problem if you encounter any technical difficulties. Then contact the \href{http://www.fullerton.edu/it/students/helpdesk/index.php}{Student IT Help Desk} for assistance. You can also call the Student IT Help Desk at \href{tel:+16572788888}{(657) 278-8888}, \href{mailto:StudentITHelpDesk@fullerton.edu}{email}, visit them at the Pollak Library North \href{http://www.fullerton.edu/it/students/sgc/index.php}{Student Genius Center}, or log on to the \href{http://my.fullerton.edu/}{my.fullerton.edu} portal and click ``Online IT Help'' followed by ``Live Chat''.

\subsection*{Canvas Support}

If you encounter any technical difficulties with Canvas, call the Canvas Support Hotline at \href{tel:+18553027528}{855-302-7528}, visit the \href{https://community.canvaslms.com/docs/DOC-10720-67952720329}{Canvas Community}, or click the ``Help'' button in the lower left corner of Canvas and select ``Report a Problem''. The \href{https://cases.canvaslms.com/liveagentchat?chattype=student&sfid=001A000000YzcwQIAR}{Student Support Live Chat} is available 24 hours a day, 7 days a week.

\section*{Response Time} I will strive to respond to all student emails, Discord posts, and \emph{Canvas} messages within 24 hours, except on weekends and holidays. If you are still awaiting a response within 24 hours, please send a follow-up message. If you are still waiting to receive a response within 48 hours, please send another follow-up message and contact me via phone or text at \href{tel:+16572784770}{(657) 278-4770}.


\section{University Student Policies}

In accordance with UPS 300.00, students must be familiar with certain policies applicable to all courses. Please review these policies as needed and visit this Cal State Fullerton website \texttt{\href{https://fdc.fullerton.edu/teaching/student-info-syllabi.html}{https://fdc.fullerton.edu/teaching/student-info-syllabi.html}} for links to the following information:

\begin{enumerate}
    \item   University learning goals and program learning outcomes.
    \item	Learning objectives for each General Education (GE) category.
    \item	Guidelines for appropriate online behavior (netiquette).
    \item	Students’ rights to accommodations for documented special needs.
    \item   Campus student support measures, including Counseling \& Psychological Services, Title IV and Gender Equity, Diversity Initiatives and Resource Centers, and Basic Needs Services.
    \item	Academic integrity (refer to UPS 300.021).
    \item	Actions to take during an emergency.
    \item	Library services information.
    \item	Student Information Technology Services, including details on technical competencies and resources required for all students.
    \item	Software privacy and accessibility statements.
\end{enumerate}

\section*{Course Student Policies}

\subsection*{Course Communication}
All course announcements and communications will be sent via \emph{Canvas} and university email. Students are responsible for regularly checking their \emph{Canvas} notifications and email. Students are also responsible for ensuring that their \emph{Canvas} notifications are set to receive messages from the course. Students are expected to check \emph{Canvas} and their email at least once daily.

\subsection*{Due Dates}
All assignments are due by 11:59pm on the specified due date. Late assignments will not be accepted unless prior arrangements have been made with the professor. Students are responsible for submitting assignments on time.

\subsection*{Alternative Procedures for Submitting Work}
Students are expected to submit all assignments via \emph{Canvas}. If you cannot submit an assignment via \emph{Canvas}, please get in touch with the professor to discuss alternative submission procedures.

\subsection*{Retention of Student Work}
Students are responsible for retaining copies of all assignments submitted for this course. Students are also responsible for retaining copies of all graded assignments returned by the professor.

\subsection*{Extra Credit}
There are no extra credit assignments in this course. 

\subsection*{Academic Integrity}
Students are expected to adhere to the highest standards of academic integrity. Any student found to have engaged in academic dishonesty will be subject to the sanctions described in the \href{https://www.fullerton.edu/senate/publications_policies_resolutions/ups/UPS%20300/UPS%20300.021.pdf}{Academic Dishonesty Policy} (UPS 300.021). Academic dishonesty includes, but is not limited to, cheating, plagiarism, fabrication, facilitating academic dishonesty, and submitting previously graded work without prior authorization. Students are expected to be familiar with the university's policy on academic dishonesty and to adhere to this policy in all aspects of this course. Any student who has questions about the policy should ask the professor for clarification.

\section*{Course Delivery and Technology}

This course is divided into 16 modules. Each module is one week long and begins on a Monday and ends on a Friday. Each module contains a variety of learning activities, including readings, videos, discussions, and assignments. The course schedule is available on \emph{Canvas} and is subject to change with advance notice.

\section*{Canvas}

This course will be delivered via \emph{Canvas}, the primary platform for all course materials, announcements, and communications. Students are expected to:
\begin{itemize}
    \item Log on to \emph{Canvas} at least once daily to check for updates.
    \item Ensure that \emph{Canvas} notifications are set to receive messages.
    \item Check their university email at least once daily.
\end{itemize}

\section*{Course Structure Overview}

The course is structured around weekly modules, each including the following components:
\begin{itemize}
    \item \textbf{Video Modules}: One videos per module, complementing the readings.
    \item \textbf{Readings}: Each module corresponds to a textbook chapter along with supplementary readings.
    \item \textbf{Discussions}: Weekly discussion prompts to be responded to by Friday 11:59pm.
    \item \textbf{Writing Assignments}: Four policy memo assignments spread throughout the course, due by Friday 11:59pm.
    \item \textbf{Final Project}: A comprehensive policy analysis report, due at the end of the last module.
    \item \textbf{Participation and Engagement}: Active participation in discussions and engagement with all course materials.
\end{itemize}

\section*{Course Assignment Descriptions}

\subsection*{Discussion Posts}
Students are required to submit weekly discussion posts in response to assigned prompts. These should:
\begin{itemize}
    \item Be at least 250 words in length.
    \item Demonstrate critical engagement with the course material.
    \item Be submitted by Friday 11:59pm each week.
\end{itemize}

\subsection*{Policy Memo Assignments}
There are four policy memos throughout the course, requiring students to:
\begin{itemize}
    \item Analyze specific environmental policy issues.
    \item Submit memos of at least 500 words by the due date.
    \item Show critical analysis and engagement with the topic.
\end{itemize}

\subsection*{Final Project}
The final project will allow students to creatively engage with and present their analysis of a selected environmental issue. Options for the final project include:

\begin{itemize}
    \item \textbf{Recorded Presentation}: A video presentation, lasting 15-20 minutes, where students present their policy analysis, incorporating visuals such as slides, graphs, and other relevant media.
    \item \textbf{Podcast}: A podcast episode, 15-20 minutes long, discussing the policy analysis in a structured format. Students should include interviews with experts or role-playing to enhance the discussion (if possible).
    \item \textbf{Interactive Web Content}: Creation of an interactive website or digital presentation that includes text, visuals, and other media to explain the policy analysis. This should be equivalent in content to a 2,000-word report.
\end{itemize}

Students are required to:
\begin{itemize}
    \item Submit a proposal early in the semester outlining their chosen format and topic.
    \item Provide a draft or outline mid-semester for feedback.
    \item Ensure that their final submission effectively communicates their analysis and understanding of the chosen environmental issue.
    \item Submit their final project by Friday 11:59pm of the final module.
\end{itemize}

\textbf{Evaluation Criteria:} Projects will be evaluated based on clarity of communication, depth of analysis, creativity in presentation, technical quality, and engagement with the topic.

\section*{Grading}
Grades will be allocated based on the following components:
\begin{itemize}
    \item \textbf{Discussion Posts}: 20\%
    \item \textbf{Policy Memos}: 40\%
    \item \textbf{Final Project}: 30\%
    \item \textbf{Participation and Engagement}: 10\%
\end{itemize}

\subsection*{Grading Scale}

\begin{itemize}
    \item A: 90-100
    \item A-: 85-89
    \item B+: 80-84
    \item B: 75-79
    \item B-: 70-74
    \item C+: 65-69
    \item C: 60-64
    \item C-: 55-59
    \item D+: 50-54
    \item D: 45-49
    \item D-: 40-44
    \item F: 0-39
\end{itemize}

\section*{Course Schedule}
\subsection*{Week 1 (8/26): Introduction to Public Policy}
\begin{itemize}
    \item \textbf{Readings:} Chapter 1 - Public Policy and Politics
    \item \textbf{Video:} Overview of Public Policy (provided on Canvas)
    \item \textbf{Discussion Post:} Discuss why it's important to study public policy and its impacts on society.
    \item \textbf{Assignment:} None this week.
\end{itemize}

\subsection*{Week 2 (9/2): Government Institutions and Policy Actors}
\begin{itemize}
    \item \textbf{Readings:} Chapter 2 - Government Institutions and Policy Actors
    \item \textbf{Video:} The Role of Federalism and Separation of Powers in Policy Making
    \item \textbf{Discussion Post:} Evaluate the impact of federalism on environmental policies.
    \item \textbf{Assignment:} Policy Memo 1 on the role of federal and state governments in policy formation.
\end{itemize}

\subsection*{Week 3 (9/9): The Policy Process}
\begin{itemize}
    \item \textbf{Readings:} Chapter 3 - Understanding Public Policymaking
    \item \textbf{Video:} The Policy Making Process Explained
    \item \textbf{Discussion Post:} How do different public policy instruments affect environmental policy?
\end{itemize}

\subsection*{Week 4 (9/16): Policy Analysis Introduction}
\begin{itemize}
    \item \textbf{Readings:} Chapter 4 - Policy Analysis: An Introduction
    \item \textbf{Video:} Fundamentals of Policy Analysis
    \item \textbf{Discussion Post:} Discuss the importance of policy analysis in shaping effective environmental policies.
\end{itemize}

\subsection*{Week 5 (9/23): Problem Analysis and Policy Alternatives}
\begin{itemize}
    \item \textbf{Readings:} Chapter 5 - Public Problems and Policy Alternatives
    \item \textbf{Video:} Crafting Effective Policy Alternatives
    \item \textbf{Discussion Post:} Analyze a recent environmental issue and discuss potential policy alternatives.
\end{itemize}

\subsection*{Week 6 (9/30): Evaluating Policy Alternatives}
\begin{itemize}
    \item \textbf{Readings:} Chapter 6 - Assessing Policy Alternatives
    \item \textbf{Video:} Methods of Policy Evaluation
    \item \textbf{Discussion Post:} Evaluate the effectiveness of a recent controversial environmental policy.
\end{itemize}

\subsection*{Week 7 (10/7): Economic and Budgetary Policy}
\begin{itemize}
    \item \textbf{Readings:} Chapter 7 - Economic and Budgetary Policy
    \item \textbf{Video:} Economic Policies and Their Impact on the Environment
    \item \textbf{Discussion Post:} Discuss how economic policies influence environmental regulation.
\end{itemize}

\subsection*{Week 8 (10/14): Midterm Review}
\begin{itemize}
    \item \textbf{Assignment:} Policy Memo 2 analyzing the impact of economic policy on environmental regulation.
\end{itemize}

\subsection*{Week 9 (10/21): Health Care Policy}
\begin{itemize}
    \item \textbf{Readings:} Chapter 8 - Health Care Policy
    \item \textbf{Video:} Public Health and Environmental Concerns
    \item \textbf{Discussion Post:} Discuss the intersection of health care policies and environmental health.
\end{itemize}

\subsection*{Week 10 (10/28): Welfare and Social Security Policy}
\begin{itemize}
    \item \textbf{Readings:} Chapter 9 - Welfare and Social Security Policy
    \item \textbf{Video:} Social Policies and Environmental Impacts
    \item \textbf{Discussion Post:} Analyze the implications of welfare policies on environmental sustainability.
\end{itemize}

\subsection*{Week 11 (11/4): Education Policy}
\begin{itemize}
    \item \textbf{Readings:} Chapter 10 - Education Policy
    \item \textbf{Video:} Education Reforms and Environmental Awareness
    \item \textbf{Discussion Post:} Discuss the role of education policies in promoting environmental consciousness.
\end{itemize}

\subsection*{Week 12 (11/11): Environmental and Energy Policy}
\begin{itemize}
    \item \textbf{Readings:} Chapter 11 - Environmental and Energy Policy
    \item \textbf{Video:} Challenges and Opportunities in Environmental and Energy Policies
    \item \textbf{Discussion Post:} Evaluate the effectiveness of recent environmental or energy policies.
\end{itemize}

\subsection*{Week 13 (11/18): Foreign Policy and Homeland Security}
\begin{itemize}
    \item \textbf{Readings:} Chapter 12 - Foreign Policy and Homeland Security
    \item \textbf{Video:} Environmental Security in Foreign Policy
    \item \textbf{Discussion Post:} Discuss the impact of homeland security measures on environmental policies.
\end{itemize}

\subsection*{Fall Break (11/25): No Class}

\subsection*{Week 14 (12/2): Policy Analysis and Choice}
\begin{itemize}
    \item \textbf{Readings:} Chapter 13 - Politics, Analysis, and Policy Choice
    \item \textbf{Video:} Decision Making in Public Policy
    \item \textbf{Discussion Post:} Reflect on how political analysis and public choices impact policy decisions.
\end{itemize}

\subsection*{Week 15 (12/9): Course Review and Final Exam Prep}
\begin{itemize}
    \item \textbf{Assignment:} Policy Memo 3 reviewing a major environmental issue and proposing policy solutions.
\end{itemize}

\subsection*{Week 16 (12/16): Final Project Submissions and Course Reflections}
\begin{itemize}
    \item \textbf{Final Project:} Comprehensive public policy analysis report due.
    \item \textbf{Discussion Post:} Reflect on the course learnings and your final project experience.
\end{itemize}

This revised schedule aligns the course structure with the chapters from the new textbook while maintaining the focus on environmental policy topics wherever relevant.

\end{document}
