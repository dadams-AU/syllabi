\documentclass[12pt, letterpaper]{article}
\usepackage[english]{babel}
\usepackage[T1]{fontenc}
%\usepackage[light]{ubuntu}
\usepackage[margin=1.25in]{geometry}
\usepackage{xcolor}
\usepackage{url}
\usepackage[utf8]{inputenc}
\usepackage[sfdefault]{roboto}
\usepackage{tabularx}
\usepackage{booktabs}
\frenchspacing
\usepackage{multicol}
\usepackage{eso-pic}
\usepackage[longnamesfirst]{natbib}
\bibpunct{(}{)}{;}{a}{}{,}
\usepackage{caption}
\usepackage{subcaption}
\usepackage{setspace}
\usepackage{paralist}
\usepackage{quoting}
\usepackage{comment}
\usepackage{enumitem}
\usepackage{graphicx}
\usepackage{float}
\usepackage{bookmark}
\renewcommand{\thesection}{\arabic{section}.}
\renewcommand{\thesubsection}{\thesection\arabic{subsection}}
\renewcommand{\thesubsubsection}{\thesubsection.\arabic{subsubsection}}
\usepackage{hyperref}
\hypersetup{
    colorlinks=true,
    linkcolor=blue,
    citecolor=black,      
    urlcolor=blue
}

\usepackage{fancyhdr}
\usepackage{graphicx}
\pagestyle{fancy}
\renewcommand{\headrulewidth}{0pt}
\fancyhead{}
\fancyfoot{}
\fancyfoot[C]{\thepage}

\fancypagestyle{plain}{
  \fancyhead{}
  \fancyhead[C]{\includegraphics[width=8cm]{csuf_logo.png}}
  \fancyfoot{}
  \renewcommand{\headrulewidth}{0pt}
}


\usepackage{titlesec}
\titleformat{\section}{\normalfont\fontsize{20}{15}\bfseries}{\thesection}{1em}{}
\titleformat{\subsection}{\normalfont\fontsize{16}{15}\bfseries}{\thesubsection}{1em}{}
\titleformat{\subsubsection}{\normalfont\fontsize{12}{15}\bfseries}{\thesubsubsection}{1em}{}
\titleformat{\paragraph}{\normalfont\fontsize{12}{15}\bfseries}{\theparagraph}{1em}{}
\titleformat{\subparagraph}{\normalfont\fontsize{12}{15}\bfseries}{\thesubparagraph}{1em}{}
\titlespacing*{\section}{0pt}{0.5\baselineskip}{0.5\baselineskip}
\titlespacing*{\subsection}{0pt}{0.5\baselineskip}{0.5\baselineskip}
\titlespacing*{\subsubsection}{0pt}{0.5\baselineskip}{0.5\baselineskip}
\titlespacing*{\paragraph}{0pt}{0.5\baselineskip}{0.5\baselineskip}
\titlespacing*{\subparagraph}{0pt}{0.5\baselineskip}{0.5\baselineskip}
 
    
    
    \begin{document}
    \title{Collaborative Governance}
    
    \author{POSC 588 -- Spring 2024}
    \date{Tuesdays at 7:00 in McCarthy Hall 213}
    
        \maketitle

\subsection*{Professor: David P. Adams, Ph.D.}

\subsubsection*{Contact Information:}

\begin{itemize}
	\item Office: 516 Gordon Hall
	\item Phone/Text: (657) 278-4770
	\item website: \href{https://dadams.io}{\texttt{dadams.io}}
	\item email: \href{dpadams@fullerton.edu}{\texttt{dpadams@fullerton.edu}}
	\item Office hours: Tuesdays \& Thursdays from 9:30 to 11:00, Thursdays from 5:30 to 6:30, and by \href{https://ly.dadams.io/appt}{appointment}.
	\item Schedule meetings throughout the week: \href{https://dadams.io/appt}{\texttt{dadams.io/appt}}
\end{itemize}


\section*{Catalog Description}

Topics include federalism, intersectoral public administration, intergovernmental relations, public-private partnerships, public contract management, interlocal agreements, and network governance.

\section*{Course Description}

Organizations across all sectors increasingly respond to complex problems through involvement in networks that offer innovative and flexible responses. Managing networks is different from managing a single organization. Knowing ways of working within and across organizations is essential to effective performance in a networked system. This course focuses on collaborative governance as interactions across nonprofit, for-profit, and public sectors, with analyses and applications. The course also focuses on federalism, intergovernmental relations, public-private partnerships, contract management, interlocal service provision and production, and networked governance.

\section*{Course Objectives}

While collaborative governance can help generate and implement enduring and meaningful public policy, it can also be challenging. This course explores the management issues raised by collaborative governance. It seeks to provide a theoretical and practical foundation so that you can become a better producer and consumer of the processes, tools, and approaches to collaborative governance. By the end of the course, students should be able to

\begin{enumerate}
    \item Identify fundamental changes in public management that have led to the increasing usage of intergovernmental, interagency, and intersectoral networks;
    \item Understand the difference between managing hierarchies and managing networks;
    \item Practice and apply various techniques and tools for improving collaborative governance;
    \item Suggest courses of action for improving the performance of collaborative governance;
    \item Describe key concepts, principles, tools, and problems associated with collaborative governance;
    \item Demonstrate how collaborative governance is being used to address contemporary issues and assess the potential of collaborative governance for modern policy problems. 
\end{enumerate}


\section*{Required Texts}

There are three books for this course:

\begin{enumerate}[leftmargin=!,labelindent=5pt,itemindent=-15pt]
    \item Agranoff, Robert. 2012. \emph{Collaborating to Manage: A Primer for the Public Sector}. Washington, DC: Georgetown University Press. 

    \item Agranoff, Robert and Aleksey Kolpakov. 2023. \emph{The Politics of Collaborative Public Management: A Primer}. New York: Routledge.
    
    \item  Henderson, Alexander C. 2015. \emph{Municipal Shared Services and Consolidation: A Public Solutions Handbook}. New York: Routledge. 
\end{enumerate}

\subsection*{Additional Readings}

In addition to the above texts, several additional readings, including articles, book chapters, and case studies, are posted on Canvas and are noted in the course schedule at the end of this document. 


\section*{University Student Policies}

In accordance with UPS 300.00, students must be familiar with certain policies applicable to all courses. Please review these policies as needed and visit this Cal State Fullerton website \href{https://t.ly/csuf-syllabus}{\texttt{https://t.ly/csuf-syllabus}} for links to the following information:

\begin{enumerate}
    \item   University learning goals and program learning outcomes.
    \item	Learning objectives for each General Education (GE) category.
    \item	Guidelines for appropriate online behavior (netiquette).
    \item	Students’ rights to accommodations for documented special needs.
    \item   Campus student support measures, including Counseling \& Psychological Services, Title IV and Gender Equity, Diversity Initiatives and Resource Centers, and Basic Needs Services.
    \item	Academic integrity (refer to UPS 300.021).
    \item	Actions to take during an emergency.
    \item	Library services information.
    \item	Student Information Technology Services, including details on technical competencies and resources required for all students.
    \item	Software privacy and accessibility statements.
\end{enumerate}

\section*{Course Student Policies}

\subsection*{Course Communication}
All course announcements and communications will be sent via \emph{Canvas} and university email. Students are responsible for regularly checking their \emph{Canvas} notifications and email. Students are also responsible for ensuring that their \emph{Canvas} notifications are set to receive messages from the course. Students are expected to check \emph{Canvas} and their email at least once daily.

\subsection*{Due Dates}
If you have concerns about meeting assignment deadlines, please get in touch with the professor in advance to discuss potential accommodation. Late work is not accepted without prior approval from the professor.

\subsection*{Alternative Procedures for Submitting Work}
Students are expected to submit all assignments via \emph{Canvas}. If you cannot submit an assignment via \emph{Canvas}, please get in touch with the professor to discuss alternative submission procedures.

\subsection*{Extra Credit}
There are no extra credit assignments in this course. 

\subsection*{Academic Integrity}
Students are expected to adhere to the highest standards of academic integrity. Any student found to have engaged in academic dishonesty will be subject to the sanctions described in the \href{https://www.fullerton.edu/senate/publications_policies_resolutions/ups/UPS%20300/UPS%20300.021.pdf}{Academic Dishonesty Policy} (UPS 300.021). Academic dishonesty includes, but is not limited to, cheating, plagiarism, fabrication, facilitating academic dishonesty, and submitting previously graded work without prior authorization. Students are expected to be familiar with the university's policy on academic dishonesty and to adhere to this policy in all aspects of this course. Any student who has questions about the policy should ask the professor for clarification.


\subsection*{Response Time} I will strive to respond to all student emails and \emph{Canvas} messages within 24 hours, except on weekends and holidays. If you are still awaiting a response within 24 hours, please send a follow-up message. If you are still waiting to receive a response within 48 hours, please send another follow-up message and contact me via phone or SMS at (657) 278-4770.


\section*{Kritik: Sharpening Your Peer-Review Skills}

This term, we'll leverage Kritik, a dynamic peer learning platform, to hone your critical thinking and communication skills—essential tools for any aspiring public administrator. Through Kritik, you'll analyze real-world policy scenarios, provide constructive feedback to peers, and receive valuable insights on your own work.

\begin{itemize}

\item \textbf{A Three-Stage Learning Journey:}

\begin{enumerate}
    \item Craft Your Analysis: Follow the provided rubric and delve into a public policy challenge. This could involve, for example, evaluating the ethical implications of a proposed environmental regulation or assessing the effectiveness of a social welfare program.
    \item Provide Constructive Critique: Anonymously evaluate your peers' work using the rubric. Offer actionable feedback that focuses on the strengths and weaknesses of their analysis, drawing connections to relevant public administration concepts.
    \item Reflect and Improve: Receive anonymous feedback on the quality and impact of your comments. Learn to deliver clear, concise, and impactful feedback—a crucial skill for public servants collaborating on complex issues.
\end{enumerate}

\item \textbf{Grading and Participation:} You'll earn four scores for each Kritik activity: Creation, Evaluation, Feedback, and Overall. These scores, along with active participation, will contribute to your course grade. Participating thoughtfully in Kritik activities will not only improve your own skills but also enrich the learning experience for your peers.

\item \textbf{Registration and Support:} We'll thoroughly introduce Kritik in class, and a dedicated email invitation will guide you through registration and course enrollment. The Kritik Help Center offers additional resources, and I'm always available to address any questions or concerns.
\end{itemize}

\section*{Course Requirements}

Assignment due dates are listed in the course schedule at the end of this document. Written assignments are submitted through \emph{Kritik} according to the times, days, and additional instructions provided on the platform. Kritik assignments are due in three stages: (1) the initial submission, (2) the peer review, and (3) the reflection. Initial submissions are due as indicated below. Peer reviews are due 72 hours after the initial submission deadline. Reflections are due 48 hours after the peer review deadline.

\subsection*{Big Dig Podcast}

This 2-point assignment involves listening to a podcast and contributing to a Canvas discussion. The assignment is worth 2\% of your total grade.

\subsection*{Kritik Orientation}

This 5-point assignment is designed to familiarize you with Kritik. The assignment is worth 3\% of your total grade. This assignment will open in the 1st week of the semester and close in the 2nd week of the semester.
 
\subsection*{Intellectual Autobiography}

This short essay should describe your intellectual evolution and its impact on your viewpoint regarding public administration. The assignment is worth 5\% of your total grade. This assignment will open in the 2nd week of the semester and close in the 3rd week of the semester.
 
\subsection*{Stakeholder Mapping}

This 5-point assignment is designed to help you identify the stakeholders involved in a collaborative governance process. The assignment is worth 5\% of your total grade. This assignment will open in the 3rd week of the semester and close in the 4th week of the semester.

\subsection*{Case Study Analysis}

This 20-point assignment is designed to help you apply the concepts and theories discussed in class to a real-world case study. The assignment is worth 20\% of your total grade. This assignment will open in the 3rd week of the semester and close in the 6th week of the semester.

\subsection*{Group Paper}

This 40-point assignment is designed to help you apply the concepts and theories discussed in class to a real-world case study. The assignment is worth 40\% of your total grade. This assignment will open in the 6th week of the semester and close in the 13th week of the semester. A mid-semester symposium will be held in the 10th week of the semester to present progress on the group project.

\subsection*{Group Presentation}

This 10-point assignment is designed to help you apply the concepts and theories discussed in class to a real-world case study. Presentations will be on the last day of class --- the 15th week. The assignment is worth 10\% of your total grade.

\subsection*{Simulation Reflection}

This 10-point assignment requires you to reflect on the collaboration simulation event you participated in. The purpose is to critically connect and compare your hands-on simulation experiences with the theoretical concepts, literature, and case studies we have explored in our course. The assignment is due one week after the simulation. 

\subsection*{Discussion Facilitation}

Two or three students will be assigned to facilitate discussion for each class session. The assignment is worth 5\% of your total grade.

\subsection*{Course Reflection}

This 5-point assignment is designed to help you reflect on your learning in this course. The assignment is worth 5\% of your total grade. It is due on the day of the final exam --- the 16th week.


\section*{Grades}

\subsection*{Grading Scale and Grade Weights}  
The grading scale is shown in Table~\ref{tab:grading-scale}. Grades will be given based on Table~\ref{tab:grade-weights} weights.

\begin{table}[ht]
\centering
\caption{Grading Scale}
\begin{tabular}{llll}
\toprule
\textbf{Grade} & \textbf{Percentage} & \textbf{Grade} & \textbf{Percentage} \\
\midrule
A+ & 98.0 – 100 & B- & 80.0 – 81.9\\
A & 92.0 – 97.9 & C+ & 78.0 – 79.9\\
A- & 90.0 – 91.9 & C & 72.0 – 77.9\\
B+ & 88.0 – 89.9 & C- & 70.0 – 71.9\\
B & 82.0 – 87.9 & & \\
\bottomrule
\end{tabular}
\label{tab:grading-scale}
\end{table}

\begin{table}[ht]
    \centering
    \caption{Graded Items and Points}
    \begin{tabular}{lll}
        \toprule
    \textbf{Assignment} & \textbf{Points} & \textbf{Due Date}\\
    \midrule
    1. Big Dig Podcast & 2 & 1/30 \\
    2. Kritik Orientation & 3 & 1/30 \\
    3. Intellectual Autobiography & 5 & 2/6 \\
    4. Stakeholder Mapping & 5 & 2/13 \\
    5. Case Study Analysis & 20 & 2/27\\
    6. Simulation Reflection & 10 & 4/16 \\
    7. Group Paper & 40 & 4/23 \\
    8. Group Presentation & 10 & 5/7\\
    9. Course Reflection & 5 & 5/14 \\ 
    10. Discussion Facilitation & 5 & Rolling \\\bottomrule
    \emph{Total} & \emph{100} \\
    \end{tabular}
    \label{tab:grade-weights}
    \end{table}

\newpage


\section*{Course Schedule}

The course schedule is subject to change with advance notice. Changes will be announced in class and posted on \emph{Canvas}.

\subsection*{1/23 -- Week 1: Flexible Asynchronous Course}
    \subsubsection*{Activities}
        \begin{itemize}
            \item "The Big Dig" podcast
            \item Introduction to Kritik Video
        \end{itemize}   
    \subsubsection*{Readings - Get an early start!}
        \begin{itemize}
            \item \citet[chapter 1]{Agranoff2012}
            \item \citet[chapter 1]{Agranoff2023}
            \item \citet[chapter 1]{Henderson2015}
        \end{itemize}
    \subsubsection*{Assignments}
        \begin{itemize}
            \item Big Dig Podcast and Discussion
            \item Kritik Orientation Activity
        \end{itemize}

\subsection*{1/30 -- Week 2: Introduction to Collaborative Governance}
    \subsubsection*{Activities}
        \begin{itemize}
            \item Overview of the course structure and expectations
            \item Review of the syllabus
            \item Introduction to Kritik
        \end{itemize}
    \subsubsection*{Readings}
        \begin{itemize}
            \item \citet[chapter 1]{Agranoff2012}
            \item \citet[chapter 1]{Agranoff2023}
            \item \citet[chapter 1]{Henderson2015}
        \end{itemize}
    \subsubsection*{Assignments}
        \begin{itemize}
            \item Kritik Orientation
            \item Intellectual Autobiography
        \end{itemize}

\subsection*{2/6 -- Week 3: Foundations of Collaborative Governance}
    \subsubsection*{Activities}
        \begin{itemize}
            \item Discussion of the readings
            \item Introduction to stakeholder mapping
        \end{itemize}
    \subsubsection*{Readings}
        \begin{itemize}
            \item \citet[chapter 2]{Agranoff2023}
            \item \citet[chapter 1]{Bingham2008}
            \item \citet[chapter 2]{Henderson2015}
        \end{itemize}
    \subsubsection*{Assignments}
        \begin{itemize}
            \item Stakeholder Mapping
        \end{itemize}

\subsection*{2/13 -- Week 4: Boundaries, Federalism, and Intergovernmental Relations}
    \subsubsection*{Activities}
        \begin{itemize}
            \item Discussion of the readings
            \item Introduction to case study analysis
        \end{itemize}
    \subsubsection*{Readings}
        \begin{itemize}
           \item \href{https://www.maxwell.syr.edu/docs/default-source/research/parcc/e-parcc/building-a-healthy-community-victoria-lowerson-and-martha-s-feldman-case.pdf?sfvrsn=c38fa74_2}{Case Study: Building Healthy Communities}
            \item \citet[chapter 2]{Agranoff2012}
            \item \citet[chapter 3]{Agranoff2023}
            \item \citet{SCHNEIDER2009a}
        \end{itemize}
    \subsubsection*{Assignments}
        \begin{itemize}
            \item Case Study Analysis
        \end{itemize}         

\subsection*{2/20 -- Week 5: Networks and Network Management}
    \subsubsection*{Activities}
        \begin{itemize}
            \item Discussion of the readings
            \item Presentation of Stakeholder Mapping
        \end{itemize}
    \subsubsection*{Readings}
        \begin{itemize}
            \item Case Study
            \item \citet[chapters 3, 6]{Agranoff2012}
            \item \citet[chapter 8]{Agranoff2017}
            \item \citet[chapter 3]{Kickert1997}
        \end{itemize}


\subsection*{2/27 -- Week 6: Communities and Culture}
    \subsubsection*{Activities}
        \begin{itemize}
            \item Discussion of the readings
            \item Introduction to group paper
        \end{itemize}
    \subsubsection*{Readings}
        \begin{itemize}
            \item Case Study
            \item \citet[chapter 6]{Agranoff2023}
            \item \citet[chapter 3]{Henderson2015}
            \item \citet{Lubell2007}
            \item \citet[chapter 4]{Wondolleck2000}
        \end{itemize}   
    \subsubsection*{Assignments}
        \begin{itemize}
            \item Group Paper
        \end{itemize}     

\subsection*{3/5 -- Week 7: Barriers to Collaborating}
    \subsubsection*{Activities}
        \begin{itemize}
            \item Discussion of the readings
        \end{itemize}
    \subsubsection*{Readings}
        \begin{itemize}
            \item \citet[chapter 7]{Agranoff2012}
            \item \citet[chapter 9]{Agranoff2023}
            \item \citet[chapter 8]{Henderson2015}
            \item \citet[chapters 3--4]{OLeary2009}
        \end{itemize}
 
\subsection*{3/12 -- Week 8: Group Paper Virtual Workshop}

    \subsubsection*{Activities}
        \begin{itemize}
            \item Group Paper Workshop
        \end{itemize}
    \subsubsection*{Readings}
        \begin{itemize}
            \item \citet{OSTROM2010a}
        \end{itemize}

\subsection*{3/19 -- Week 9: Consolidation, Contracts, and External Agreements}
    \subsubsection*{Activities}
        \begin{itemize}
            \item Discussion of the readings
            \item Group Paper Workshop
        \end{itemize}
    \subsubsection*{Readings}
        \begin{itemize}
            \item \citet[chapter 4--5]{Agranoff2012}
            \item \citet[chapter 4--6]{Henderson2015}
        \end{itemize}

\subsection*{3/26 -- Week 10: Collaborative Governance in Action}
    \subsubsection*{Activities}
        \begin{itemize}
            \item Mid-Semester Symposium: Presentation of progress on group projects
            \item Simulation Preparation
            \item Pot Luck Dinner
        \end{itemize}
    \subsection*{Readings}
        \begin{itemize}
            \item \citet[chapter 5]{Agranoff2023}
            \item \citet[chapter 7]{Henderson2015}
        \end{itemize}


\subsection*{4/9 -- Week 11: Simulation}
    \subsubsection*{Activities}
        \begin{itemize}
            \item Simulation
        \end{itemize}
    \subsubsection*{Readings}
        \begin{itemize}
            \item TBD
        \end{itemize}
    \subsubsection*{Assignments}
        \begin{itemize}
            \item Simulation Reflection
        \end{itemize}

\subsection*{4/16 -- Week 12: New Organizations and Local Public Management}
    \subsubsection*{Activities}
        \begin{itemize}
            \item Simulation Debrief
            \item Discussion of the readings
        \end{itemize}
    \subsubsection*{Readings}
        \begin{itemize}
            \item Case Study
            \item \citet[chapter 6]{Agranoff2023}
            \item \citet[chapter 9--10]{Henderson2015}
        \end{itemize}
    \subsubsection*{Assignment}
        \begin{itemize}
            \item Group Paper
        \end{itemize}

\subsection*{4/23 -- Week 13: No Class} 
    \subsubsection*{Activities}
        \begin{itemize}
            \item Group Paper Submission
        \end{itemize}


\subsection*{4/30 -- Week 14: Performance and Collaborative Governance in the 21st Century}
    \subsubsection*{Activities}
        \begin{itemize}
            \item Discussion of the readings
            \item Group Paper Presentation Workshop
        \end{itemize}
    \subsubsection*{Readings}
        \begin{itemize}
            \item \citet[chapters 10--12]{Agranoff2023}
            \item \citet[chapter 10]{Henderson2015}
            \item TBD
        \end{itemize}


\subsection*{5/7 -- Week 15: Group Presentations}
    \subsubsection*{Activities}
        \begin{itemize}
            \item Group Presentations
        \end{itemize}

\subsection*{5/14 -- Week 16: No Class}
    \subsubsection*{Assignment}
        \begin{itemize}
            \item Course Reflection Paper
        \end{itemize}

\section*{Technical Problems}

\subsection*{University IT Help Desk}

Contact the instructor immediately to document the problem if you encounter any technical difficulties. Then contact the \href{http://www.fullerton.edu/it/students/helpdesk/index.php}{Student IT Help Desk} for assistance. You can also call the Student IT Help Desk at (657) 278-8888, \href{mailto:StudentITHelpDesk@fullerton.edu}{email}, visit them at the Pollak Library North \href{http://www.fullerton.edu/it/students/sgc/index.php}{Student Genius Center}, or log on to the \href{http://my.fullerton.edu/}{\texttt{my.fullerton.edu}} portal and click ``Online IT Help'' followed by ``Live Chat''.

\subsection*{Canvas Support}

If you encounter any technical difficulties with Canvas, call the Canvas Support Hotline at 855-302-7528, visit the \href{https://community.canvaslms.com/docs/DOC-10720-67952720329}{Canvas Community}, or click the ``Help'' button in the lower left corner of Canvas and select ``Report a Problem''. The \href{https://cases.canvaslms.com/liveagentchat?chattype=student&sfid=001A000000YzcwQIAR}{Student Support Live Chat} is available 24 hours a day, 7 days a week.

\subsection*{Kritik Support}

If you have any questions about Kritik, please use their live chat. A human agent will respond promptly within a few minutes from 9am-5pm eastern time Monday to Friday. Outside of these hours, you’ll receive a reply that Kritik will be back the next business day. They monitor around the clock and will still respond to urgent requests within a few hours. You can also visit \href{https://help.kritik.io}{\texttt{https://help.kritik.io}} to view their help articles.

\section*{Important Scheduling Note}

Please be aware the California Faculty Association---the labor union of Lecturers, Professors, Coaches, Counselors, and Librarians across the 23 CSU campuses---is navigating challenging contract negotiations with CSU management, and a strike or work stoppage may occur this term. Our working conditions are your learning conditions; we seek to protect both. For updates, visit \href{www.CFAbargaining.org}{\texttt{www.CFAbargaining.org}}.
    

\singlespace
\bibliographystyle{apsr}
\bibliography{588}



\end{document}