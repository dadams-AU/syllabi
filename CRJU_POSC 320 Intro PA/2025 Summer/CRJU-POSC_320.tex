% Simplified accessibility setup - compatible with all LaTeX distributions
\documentclass[11pt]{scrartcl} % KOMA-script for better structure

% Language and Encoding
\usepackage[english]{babel}
\usepackage[T1]{fontenc}

% Font and Typography - Using Roboto for better readability
\usepackage[sfdefault]{roboto}
\renewcommand{\familydefault}{\sfdefault}

% Enhanced hyperlinks with accessibility features
\usepackage[colorlinks=true, linkcolor=blue, urlcolor=blue, citecolor=blue]{hyperref}
\hypersetup{
  pdftitle={CRJU/POSC 320: Introduction to Public Administration - Summer 2025},
  pdfauthor={David P. Adams, Ph.D.},
  pdfsubject={Course Syllabus - Introduction to Public Administration},
  pdfkeywords={Public Administration, Criminal Justice, CSU Fullerton, Syllabus, CRJU 320, POSC 320, Government, Policy},
  pdfcreator={LaTeX with Accessibility Features},
  pdflang={en-US}
}

% Layout and Design
\usepackage[margin=1in]{geometry}
\usepackage{enumitem}
\usepackage{graphicx}
\usepackage{array, booktabs, longtable}
\usepackage{xcolor}
\usepackage{caption}

% Enhanced list formatting for accessibility
\setlist{leftmargin=*}
\setlist[itemize]{label=\textbullet}
\setlist[enumerate]{label=\arabic*.}

% Use KOMA-Script sectioning - no numbering for clean appearance
\setcounter{secnumdepth}{0}
\setkomafont{section}{\large\bfseries}
\setkomafont{subsection}{\normalsize\bfseries}
\setkomafont{subsubsection}{\normalsize\bfseries}

% Document metadata and structure
\title{\textbf{CRJU/POSC 320: Introduction to Public Administration}\\ 
       \large Summer Session 2025 --- Asynchronous Online\\
       \normalsize June 30 -- August 1, 2025}
\author{Professor David P. Adams, Ph.D.\\
        Division of Politics, Administration, and Justice\\
        California State University, Fullerton}
\date{}

\begin{document}

\maketitle

% Add university logo with proper alt text
\begin{center}
\includegraphics[width=0.4\textwidth]{csuf_logo.png}
\end{center}

\section{Contact Information and Office Hours}

\textbf{Professor:} David P. Adams, Ph.D.\\
\textbf{Office:} Gordon Hall 516\\
\textbf{Phone/Text:} \href{tel:+16572784770}{(657) 278-4770}\\
\textbf{Email:} \href{mailto:dpadams@fullerton.edu}{dpadams@fullerton.edu}\\
\textbf{Website:} \href{https://dadams.io}{dadams.io}\\
\textbf{Zoom Office:} \href{https://fullerton.zoom.us/j/3347502369}{Join Zoom Office Hours}

\subsection{Office Hours Schedule}
\begin{itemize}
\item \textbf{Regular Hours:} Tuesdays 9:30--10:30 AM and 7:00--8:00 PM
\item \textbf{Platform:} \href{https://discord.com/channels/1128747433636135113/1154048074172354600}{Discord Office Hours Channel}
\item \textbf{Appointments:} Schedule flexible meetings at \href{https://dadams.io/appointments}{dadams.io/appointments}
\end{itemize}

\subsection{Response Time Expectations}
I will respond to emails, Discord posts, and Canvas messages within 24 hours during weekdays. If you don't receive a response within 24 hours, please send a follow-up message. After 48 hours without response, contact me by phone or text.

\section{Course Overview}

\subsection{Catalog Description}
Introduces public administration through current trends and problems of public sector agencies in such areas as organization behavior, public budgeting, personnel, planning and policy making. Examples and cases from the Criminal Justice field. (POSC 320 and CRJU 320 are the same course.)

\subsection{Course Description}
Public administration impacts our daily lives in countless ways. This course explores what public administrators do, how decisions are made, and how political environments shape administrative actions. We examine how public administrators respond to citizen demands and international pressures while balancing core values of accountability, efficiency, and equity.

This course introduces both the science and art of public administration. Students will explore theoretical and practical aspects of public administration in the American political system, including organizational theory, decision making, systems analysis, performance evaluation, and administrative improvement. Emphasis is placed on understanding the roles and responsibilities of public administrators in a democratic society.

\section{Student Learning Objectives}
Upon successful completion of this course, students will be able to:

\begin{enumerate}
\item Display broad understanding of public administration and its role in democratic society
\item Demonstrate knowledge of core concepts and theories in public administration
\item Identify complex problems facing public organizations
\item Exhibit critical thinking by interpreting information, comparing ideas, and developing informed opinions
\item Contrast public and private administration, including their benefits and limitations
\item Demonstrate effective written communication skills
\end{enumerate}

\section{Required Materials}

\subsection{Textbook}
Kettl, Daniel F. 2023. \emph{Politics of the Administrative Process} (9th ed.). Washington, D.C.: CQ Press.

\subsection{Additional Resources}
Supplemental readings and multimedia content will be posted to Canvas throughout the course.

\section{Prerequisites and General Education}

\subsection{Prerequisites}
\begin{itemize}
\item POSC 100 (American Government)
\item Completion of G.E. Category D.1
\end{itemize}

\textbf{Important:} If you have not completed these prerequisites, you should not be enrolled in this course.

\subsection{General Education Requirements Satisfied}
This course satisfies General Education Explorations in Social Sciences subarea D.4 for students using Catalog Years 2018 and later. Writing assignments meet UPS 411.201 requirements for complex data organization, careful evaluation, and remediation feedback.

\subsection{General Education Student Learning Goals}
Students will:
\begin{enumerate}
\item Examine problems, issues, and themes in the social sciences in greater depth; in a variety of cultural, historical, and geographical contexts; and from different disciplinary and interdisciplinary perspectives.
\item Analyze and critically evaluate the application of social science concepts and theories to particular historical, contemporary, and future problems or themes, such as economic and environmental sustainability, globalization, poverty, and social justice.
\item Analyze and critically evaluate constructs of cultural differentiation, including ethnicity, gender, race, class, and sexual orientation, and their effects on the individual and society.
\item Apply theories and concepts from the social sciences to address historical, contemporary, and future problems confronting communities at different geographical scales, from local to global.
\end{enumerate}

\section{Technical Requirements and Support}

\subsection{Required Technical Competencies}
\begin{itemize}
\item Basic computer skills and internet navigation
\item Proficiency with Canvas (assignments, discussions, materials access)
\item Word processing software (Microsoft Word or Google Docs)
\item Reliable computer and internet connection
\item Zoom for virtual meetings
\item Email and online discussion board usage
\item Online research tools and academic databases
\end{itemize}

\subsection{Technical Support Resources}

\subsubsection{University IT Help Desk}
Contact me immediately to document technical problems, then contact:
\begin{itemize}
\item \textbf{Phone:} \href{tel:+16572788888}{(657) 278-8888}
\item \textbf{Email:} \href{mailto:StudentITHelpDesk@fullerton.edu}{StudentITHelpDesk@fullerton.edu}
\item \textbf{Location:} Pollak Library North Student Genius Center
\item \textbf{Online:} \href{http://my.fullerton.edu/}{my.fullerton.edu}, then click ``Online IT Help'' and then ``Live Chat''
\end{itemize}

\subsubsection{Canvas Support}
\begin{itemize}
\item \textbf{Hotline:} \href{tel:+18553027528}{(855) 302-7528} (24/7)
\item \textbf{Community:} \href{https://community.canvaslms.com/docs/DOC-10720-67952720329}{Canvas Community}
\item \textbf{In Canvas:} Click ``Help'' then ``Report a Problem''
\item \textbf{Live Chat:} \href{https://cases.canvaslms.com/liveagentchat?chattype=student&sfid=001A000000YzcwQIAR}{Student Support Live Chat} (24/7)
\end{itemize}

\section{Course Policies}

\subsection{Communication Expectations}
\begin{itemize}
\item All course announcements sent via Canvas and university email
\item Students must check Canvas and email daily
\item Ensure Canvas notifications are enabled
\item Use professional communication standards
\end{itemize}

\subsection{Assignment Submission and Due Dates}
\begin{itemize}
\item All assignments due by 11:59 PM on specified dates
\item Submit all work via Canvas unless alternative arrangements made
\item Late assignments not accepted without prior arrangement
\item Contact professor immediately if deadline challenges arise
\item Students responsible for retaining copies of all submitted work
\end{itemize}

\subsection{Participation Standards}
Students must:
\begin{itemize}
\item Complete all assigned readings and videos
\item Participate professionally in all discussions
\item Follow university netiquette policies
\item Maintain respectful, academic discourse
\end{itemize}

\subsection{Academic Integrity Policy}
This course maintains the highest academic integrity standards. Academic dishonesty includes but is not limited to:
\begin{itemize}
\item Cheating and plagiarism
\item Fabrication of information
\item Facilitating others' academic dishonesty
\item Unauthorized resubmission of previous work
\end{itemize}

Violations will be subject to sanctions under \href{https://www.fullerton.edu/senate/publications_policies_resolutions/ups/UPS%20300/UPS%20300.021.pdf}{UPS 300.021 Academic Dishonesty Policy}.

\subsection{AI Tool Usage Policy}

\subsubsection{Permitted Uses}
AI tools (ChatGPT, Claude, Grammarly, etc.) may be used for:
\begin{itemize}
\item Brainstorming and idea generation
\item Grammar and style checking
\item Content summarization for understanding
\item Research assistance
\end{itemize}

\subsubsection{Required Practices}
\begin{itemize}
\item \textbf{Disclosure Required:} Document all AI tool usage
\item \textbf{Proper Citation:} Cite AI-generated content appropriately
\item \textbf{Original Analysis:} Final work must reflect your understanding
\item \textbf{Academic Integrity:} AI cannot replace your critical thinking
\end{itemize}

\subsubsection{Unacceptable Uses}
\begin{itemize}
\item Submitting AI-generated content as original work
\item Using AI to complete entire assignments
\item Failing to disclose AI assistance
\item Copying AI output without attribution
\end{itemize}

\subsubsection{Citation Examples}
\begin{itemize}
\item \textbf{Acceptable:} ``I used ChatGPT to brainstorm initial ideas, but all analysis and writing are my own.''
\item \textbf{Proper Citation:} ``Summary generated by ChatGPT, OpenAI, [date of use].''
\item \textbf{Unacceptable:} Submitting AI-generated essays without attribution
\end{itemize}

\subsection{Written Work Standards}
All submissions must demonstrate:
\begin{itemize}
\item Professional formatting and presentation
\item Proper grammar, spelling, and punctuation
\item Appropriate citation style
\item Original analysis and critical thinking
\end{itemize}

\subsection{Writing Support Resources}
Assignments will be checked using plagiarism detection software. Additional resources for writing support are available through the university's \href{https://www.fullerton.edu/writingcenter/}{Writing Center}. Students are encouraged to utilize these resources to improve their writing skills. Course-specific writing resources will also be provided on Canvas and at \href{https://courses.dadams.io/POSC320/posc320_async_handouts.html}{CRJU/POSC 320 Handouts}.

\section{Course Structure and Delivery}

\subsection{Delivery Method}
Asynchronous online via Canvas. Students must log in daily for announcements and updates.

\subsection{Course Organization}
The course consists of 10 modules over 5 weeks, with each module including:
\begin{itemize}
\item Video lectures (available on Canvas and YouTube)
\item Required readings from textbook and supplemental materials
\item Interactive quizzes
\item Discussion forums
\end{itemize}

\section{Major Assignments}

\subsection{Policy Brief Project (45\% of grade)}
A scaffolded, professional writing project developing over five weeks:

\subsubsection{Project Components}
\begin{enumerate}
\item \textbf{Week 1:} Problem identification and research foundation
\item \textbf{Week 2:} Stakeholder analysis and contextual framework
\item \textbf{Week 3:} Organizational theory application
\item \textbf{Week 4:} Management challenges and solution development
\item \textbf{Week 5:} Final recommendations and executive summary
\end{enumerate}

\subsubsection{Project Requirements}
\begin{itemize}
\item 7--10 pages, professional policy brief format
\item Integration of Kettl textbook concepts
\item Current sources (2020 or later)
\item Google Docs with tracked changes for process monitoring
\item Evidence-based recommendations for policy makers
\end{itemize}

\subsection{Research Logs (10\% of grade)}
Weekly 2--3 sentence reflections documenting:
\begin{itemize}
\item Learning process discoveries
\item Research methodology insights
\item Connection between new information and course concepts
\item Evolution of understanding over time
\end{itemize}

These metacognitive tools develop critical thinking and self-awareness essential for effective public administration.

\section{Assessment and Grading}

\subsection{Grade Distribution}
\begin{table}[ht]
\caption{Course Grade Weights}
\centering
\begin{tabular}{>{\raggedright}p{0.6\textwidth}r}
\toprule
\textbf{Assignment Category} & \textbf{Weight} \\
\midrule
Video Lectures and Quizzes & 30\% \\
Discussion Participation & 15\% \\
Policy Brief Project & 45\% \\
Research Logs & 10\% \\
\bottomrule
\end{tabular}
\label{tab:grade-weights}
\end{table}

\subsection{Grading Scale}
\begin{table}[ht]
\caption{Letter Grade Scale}
\centering
\begin{tabular}{llll}
\toprule
\textbf{Grade} & \textbf{Range} & \textbf{Grade} & \textbf{Range} \\
\midrule
A+ & 98.0--100   & C+ & 78.0--79.9 \\
A  & 92.0--97.9  & C  & 72.0--77.9 \\
A- & 90.0--91.9  & C- & 70.0--71.9 \\
B+ & 88.0--89.9  & D+ & 68.0--69.9 \\
B  & 82.0--87.9  & D  & 62.0--67.9 \\
B- & 80.0--81.9  & D- & 60.0--61.9 \\
   &             & F  & Below 59.9 \\
\bottomrule
\end{tabular}
\label{tab:grading-scale}
\end{table}

\subsection{Grade Disputes}
Contact me via email or Canvas with detailed questions about grades. Include the assignment in question and allow up to 48 hours for response. No extra credit opportunities are available.

\section{Course Schedule}

\subsection{Week 1: Introduction and Foundations (June 30--July 6)}
\textbf{Modules 1--2}
\begin{itemize}
\item Course introduction and overview
\item What is public administration?
\item Historical foundations and basic concepts
\item \textbf{Discussion:} Importance of public administration in daily life
\item \textbf{Policy Brief:} Problem statement and research foundation
\item \textbf{Research Log:} Week 1 learning reflection
\end{itemize}

\subsection{Week 2: Government Functions and Organization (July 7--13)}
\textbf{Modules 3--4}
\begin{itemize}
\item Government functions and operations
\item Organizational theory foundations
\item Applications in public administration
\item \textbf{Discussion:} Real-world examples of government functions
\item \textbf{Policy Brief:} Stakeholder analysis and context
\item \textbf{Research Log:} Week 2 stakeholder research reflection
\end{itemize}

\subsection{Week 3: Executive Branch and Human Capital (July 14--20)}
\textbf{Modules 5--6}
\begin{itemize}
\item Executive branch structure and organization
\item Common organizational problems
\item Civil service and human capital management
\item \textbf{Discussion:} Public sector human capital challenges
\item \textbf{Policy Brief:} Organizational theory application
\item \textbf{Research Log:} Week 3 organizational analysis reflection
\end{itemize}

\subsection{Week 4: Decision Making and Implementation (July 21--27)}
\textbf{Modules 7--8}
\begin{itemize}
\item Decision making in public administration
\item Budgeting processes and challenges
\item Implementation strategies
\item Performance measurement
\item \textbf{Discussion:} Public sector budgeting challenges
\item \textbf{Policy Brief:} Management challenges and solutions
\item \textbf{Research Log:} Week 4 solution development reflection
\end{itemize}

\subsection{Week 5: Regulation and Accountability (July 28--August 1)}
\textbf{Modules 9--10}
\begin{itemize}
\item Role of regulation in public administration
\item Public administration and the judiciary
\item Accountability concepts and mechanisms
\item Oversight and democratic governance
\item \textbf{Discussion:} Balancing regulation and innovation
\item \textbf{Policy Brief:} Final recommendations and polish (Due Friday, August 1)
\item \textbf{Research Log:} Final learning reflection
\item \textbf{Course Wrap-Up:} Summary and next steps
\end{itemize}

\section{University Resources and Policies}

\subsection{Required University Policy Review}
Students must familiarize themselves with policies at: \\
\url{https://fdc.fullerton.edu/teaching/student-info-syllabi.html}

Key areas include:
\begin{itemize}
\item University learning goals and program outcomes
\item General Education category objectives
\item Online behavior guidelines (netiquette)
\item Accommodation rights for documented special needs
\item Campus support services (Counseling, Title IV, Basic Needs)
\item Disability Support Services information
\item Academic integrity policies (UPS 300.021)
\item Emergency procedures
\item Library and IT services
\item Software privacy and accessibility statements
\end{itemize}

\subsection{Support Services}
\begin{itemize}
\item \textbf{\href{https://www.fullerton.edu/caps/}{Counseling \& Psychological Services}} -- confidential mental health support
\item \textbf{\href{https://www.fullerton.edu/titleix/}{Title IX and Gender Equity Resources}} -- support for discrimination and harassment issues
\item \textbf{\href{https://www.fullerton.edu/dirc/}{Diversity Initiatives and Resource Centers}} -- multicultural support and programming
\item \textbf{\href{https://www.fullerton.edu/basicneeds/}{Basic Needs Services}} -- food, housing, and financial assistance
\item \textbf{\href{https://www.fullerton.edu/dss/}{Disability Support Services (DSS)}} -- accommodations and accessibility services
\item \textbf{\href{https://www.library.fullerton.edu/}{Library Research Support}} -- research assistance and academic resources
\end{itemize}


\section{Accessibility and Accommodations}

This syllabus and all course materials have been designed with accessibility in mind. Students with documented disabilities should contact Disability Support Services to discuss accommodations. All course videos include captions, and alternative formats for course materials are available upon request.

\vspace{1em}
\hrule
\vspace{0.5em}
\textbf{Commitment to Excellence:} This course represents my commitment to outstanding, accessible education. I welcome feedback on how we can continue improving the learning experience for all students.

\end{document}