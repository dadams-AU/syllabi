\documentclass[12pt, letterpaper]{article}
\usepackage{fancyhdr}
\usepackage[english]{babel}
\usepackage[T1]{fontenc}
\usepackage[margin=1in]{geometry}
\usepackage[sfdefault]{roboto}
\usepackage{xcolor}
\usepackage{url}
\usepackage[utf8]{inputenc}
\usepackage{tabularx}
\usepackage{booktabs}
\frenchspacing
\usepackage{multicol}
\usepackage{eso-pic}
\usepackage{caption}
\usepackage{subcaption}
\usepackage{setspace}
\usepackage{paralist}
\usepackage{quoting}
\usepackage{comment}
\usepackage{enumitem}
\usepackage{fancyhdr}
\usepackage{graphicx}
\pagestyle{fancy}
\renewcommand{\headrulewidth}{0pt}
\fancyhead{}
\fancyfoot{}
\fancyfoot[C]{\thepage}

\usepackage{float}
\usepackage{bookmark}
\renewcommand{\thesection}{\arabic{section}.}
\renewcommand{\thesubsection}{\thesection\arabic{subsection}}
\renewcommand{\thesubsubsection}{\thesubsection.\arabic{subsubsection}}
\usepackage{etoolbox}
\patchcmd{\thebibliography}{\section*{\refname}}{\section{\refname}}{}{}
\usepackage{hyperref}
\hypersetup{
    colorlinks=true,
    linkcolor=blue,
    filecolor=magenta,      
    urlcolor=blue,
}



\begin{document}
\title{\includegraphics[width=8cm]{csuf_logo.png} \\ \textbf{Introduction to Public Administration}}

\author{CRJU/POSC 320 — Winter Intersession 2023-24}
\date{Asynchronous Online Course}

    \maketitle


\subsection*{Professor: David P. Adams, Ph.D.}

\subsubsection*{Contact Information:}

\begin{itemize}
	\item Office: 516 Gordon Hall
	\item Phone/SMS: (657) 278-4770
	\item website: \href{https://dadams.io}{\texttt{https://dadams.io}}
	\item email: \href{dpadams@fullerton.edu}{\texttt{dpadams@fullerton.edu}}
	\item Office Hours:
        \begin{itemize}
            \item Schedule meetings throughout the week: \href{https://t.ly/dpa-appt}{\texttt{https://t.ly/dpa-appt}}
            \item \href{https://discord.gg/pKjbtWydxm}{Discord Server} Office Hours: Tuesdays and Thursdays 9:30--11:00
        \end{itemize}
\end{itemize}


\section*{Technical Problems}

\subsection*{University IT Help Desk}

Contact the instructor immediately to document the problem if you encounter any technical difficulties. Then contact the \href{http://www.fullerton.edu/it/students/helpdesk/index.php}{Student IT Help Desk} for assistance. You can also call the Student IT Help Desk at (657) 278-8888, \href{mailto:StudentITHelpDesk@fullerton.edu}{email}, visit them at the Pollak Library North \href{http://www.fullerton.edu/it/students/sgc/index.php}{Student Genius Center}, or log on to the \href{http://my.fullerton.edu/}{my.fullerton.edu} portal and click ``Online IT Help'' followed by ``Live Chat''.

\subsection*{Canvas Support}

If you encounter any technical difficulties with Canvas, call the Canvas Support Hotline at 855-302-7528, visit the \href{https://community.canvaslms.com/docs/DOC-10720-67952720329}{Canvas Community}, or click the ``Help'' button in the lower left corner of Canvas and select ``Report a Problem''. The \href{https://cases.canvaslms.com/liveagentchat?chattype=student&sfid=001A000000YzcwQIAR}{Student Support Live Chat} is available 24 hours a day, 7 days a week.

\section*{Response Time} I will strive to respond to all student emails and \emph{Canvas} messages within 24 hours, except on weekends and holidays. If you are still awaiting a response within 24 hours, please send a follow-up message. If you are still waiting to receive a response within 48 hours, please send another follow-up message and contact me via phone or SMS at (657) 278-4770.

\section*{Catalog Description}

Introduces public administration through current trends and problems of public sector agencies in such areas as organization behavior, public budgeting, personnel, planning and policy making. Examples and cases from the Criminal Justice field. (POSC 320 and CRJU 320 are the same course.)

\section*{Course Description}

Public administration plays an important role in our everyday lives. What do public administrators do? What makes this important field of government work? How are decisions made and how does the political environment impact those decisions? Our public administrators have to respond to various demands from United States residents and deal with situations and demands from abroad. The values we share interact and compete for the way our administrators create and implement policy. The core values of public administration include accountability, efficiency, and equity. We'll explore these topics and more as we engage in our class together. 

\vspace*{1em}

\noindent This course is an introduction to the study and practice---the science and art---of public administration. Students will be acquainted with the theoretical and practical aspects of public administration in the American political setting. Topics include organizational theory and practice, decision making, systems analysis, performance evaluation, and administrative and managerial improvement, among others. The emphasis is placed on understanding the roles and responsibilities of public administrators in a democratic political system. 
	

\section*{Student Learning Objectives}

\begin{enumerate}
    \item Display a broad understanding of public administration and its role in a democratic society. 

    \item Demonstrate knowledge of the concepts and theories in public administration. 
    
    \item Identify complex problems that face public organizations.

    \item Exhibit critical thinking by interpreting information, comparing ideas, and developing opinions. 
    
    \item Contrast public and private administration with their corresponding benefits and shortfalls. 

    \item Demonstrate effective written communication skills. 

\end{enumerate}

\section*{Required Text}

\begin{itemize}
    \item Kettl, Daniel F. 2023. \emph{Politics of the Administrative Process} (9th ed.) Washington, D.C.: CQ Press.
    \item Additional readings posted to Canvas
\end{itemize}

\section*{Prequisites}

POSC 100 and completion of G.E. Category D.1.  If you have not already taken and passed this course or its equivalent, you should not be enrolled in POSC/CRJU 320.

\section*{General Education Information}


\subsection*{Requirements Satisfied}

	This course satisfies General Education Explorations in Social Sciences subarea D.4 for those using Catalog Years 2018 and later. The writing assignments in this course, including the policy memo papers and current event summaries described below, meet the requirement of UPS 411.201: 
	\begin{quote}Writing assignments in General Education courses shall involve the organization and expression of complex data or ideas and careful and timely evaluations of writing so that deficiencies are identified, and suggestions for improvement and/or for means of remediation are offered. Evaluations of the student's writing competence shall determine the final course grade\ldots .\end{quote}

\subsection*{General Education Student Learning Goals}

	Students completing courses in this subarea shall encounter the following learning goals:

\begin{enumerate}
	\item Examine problems, issues, and themes in the social sciences in greater depth; in a variety of cultural, historical, and geographical contexts; and from different disciplinary and interdisciplinary perspectives.
	\item Analyze and critically evaluate the application of social science concepts and theories to particular historical, contemporary, and future problems or themes, such as economic and environmental sustainability, globalization, poverty, and social justice.
	\item Analyze and critically evaluate constructs of cultural differentiation, including ethnicity, gender, race, class, and sexual orientation, and their effects on the individual and society.
	\item Apply theories and concepts from the social sciences to address historical, contemporary, and future problems confronting communities at different geographical scales, from local to global.
\end{enumerate}


\section*{University Student Policies}

In accordance with UPS 300.004, students must be familiar with certain policies applicable to all courses. Please review these policies as needed and visit the Cal State Fullerton website for \texttt{\href{https://fdc.fullerton.edu/teaching/student-info-syllabi.html}{syllabus policies}} for links to the following information:

\begin{enumerate}
    \item   University learning goals and program learning outcomes.
    \item	Learning objectives for each General Education (GE) category.
    \item	Guidelines for appropriate online behavior (netiquette).
    \item	Students’ rights to accommodations for documented special needs.
    \item   Campus student support measures, including Counseling \& Psychological Services, Title IV and Gender Equity, Diversity Initiatives and Resource Centers, and Basic Needs Services.
    \item	Academic integrity (refer to UPS 300.021).
    \item	Actions to take during an emergency.
    \item	Library services information.
    \item	Student Information Technology Services, including details on technical competencies and resources required for all students.
    \item	Software privacy and accessibility statements.
\end{enumerate}


\subsection*{Academic Integrity}
Students are expected to adhere to the highest standards of academic integrity. Any student found to have engaged in academic dishonesty will be subject to the sanctions described in the \href{https://www.fullerton.edu/senate/publications_policies_resolutions/ups/UPS%20300/UPS%20300.021.pdf}{Academic Dishonesty Policy} (UPS 300.021). Academic dishonesty includes, but is not limited to, cheating, plagiarism, fabrication, facilitating academic dishonesty, and submitting previously graded work without prior authorization. Students are expected to be familiar with the university's policy on academic dishonesty and to adhere to this policy in all aspects of this course. Any student who has questions about the policy should ask the professor for clarification.


\section*{Course Student Policies}

\subsection*{Course Communication}
All course announcements and communications will be sent via \emph{Canvas} and university email. Students are responsible for regularly checking their \emph{Canvas} notifications and email. Students are also responsible for ensuring that their \emph{Canvas} notifications are set to receive messages from the course. Students are expected to check \emph{Canvas} and their email at least once daily.

\subsubsection*{Discord Server}

There is a \href{https://discord.gg/pKjbtWydxm}{Discord Server} available with text and voice channels for students to communicate with each other and the professor. Office Hours are on the server Tuesdays and Thursdays at 9:30 to 11:00 a.m. Feel free to post in the \#office-hours and \#320-public-admin channels. The Discord Server is not required for this course, but it is available for students who wish to use it.

\subsection*{Participation}

Students are expected to participate in all course activities. This includes completing all assigned readings, watching all assigned videos, and participating in all discussions. Students are expected to participate in discussions in a professional and respectful manner. Students are expected to be familiar with the university policy on netiquette and to adhere to this policy in all aspects of this course. Any student who has questions about the policy should ask the professor for clarification. 

\begin{quote}Netiquette refers to a set of behaviors that are appropriate for online activity (e.g., social media, email, discussions, presentations). All personnel at Cal State Fullerton are expected to demonstrate appropriate online behavior at all times. A good summary of netiquette can be found in the \href{https://canvashelp.fullerton.edu/m/Student/l/1336786-student-what-is-netiquette}{CSUF Canvas self-help guides}, which adapts ten rules to the online course situation from the website for the book \href{http://www.albion.com/netiquette/corerules.html}{Netiquette by Virginia Shea} and other sources referenced at the bottom of the guide.\end{quote}

\subsection*{Due Dates}
If you have concerns about meeting assignment deadlines, please get in touch with the professor in advance to discuss potential accommodation. This is a fast-paced course, and late assignments will deducted a half letter grade for each day (24 hours) they are late.

\subsection*{Alternative Procedures for Submitting Work}
Students are expected to submit all assignments via \emph{Canvas}. If you cannot submit an assignment via \emph{Canvas}, please get in touch with the professor to discuss alternative submission procedures.

\subsection*{Extra Credit}
There are no extra credit assignments in this course. 

\section*{Course Delivery}

This course will be delivered asynchronously online via \emph{Canvas}. Students are expected to log on to \emph{Canvas} at least once daily to check for announcements and updates. Students are also expected to check their university email at least once daily.

\section*{Course Structure}

This course is divided into 10 modules. Each module will include a video lecture, assigned readings, a discussion, and a writing assignment. 
\section*{Course Requirements}

\subsection*{Video Lectures and Quizzes}

Each module will include a video lecture. Students are expected to watch each video lecture and complete the corresponding quiz. Video lectures will be available on \emph{Canvas} and on YouTube. Quizzes will be administered via \emph{Canvas}.

\subsection*{Readings}

Each module will include assigned readings. Students are expected to complete all assigned readings. Readings consist of the textbook material covered in each module and additional readings posted to \emph{Canvas}.

\subsection*{Discussions}

Each module will include a discussion. Students are expected to participate in all discussions. Discussions will be administered via \emph{Canvas}.

\subsection*{Writing Assignments}

Each module will include a writing assignment. Students are expected to complete all writing assignments. Writing assignments will be submitted via \emph{Canvas}.

\section*{Grades}

\subsection*{Grading Scale and Grade Weights}  
The grading scale is shown in Table~\ref{tab:grading-scale}. Grades will be given based on Table~\ref{tab:grade-weights} weights.

\begin{table}[ht]
\centering
\caption{Grading Scale}
\begin{tabular}{llll}
\toprule
\textbf{Grade} & \textbf{Percentage} & \textbf{Grade} & \textbf{Percentage} \\ \hline
        A+ & 98.0 -- 100  & C+ & 78.0 -- 79.9 \\
        A  & 92.0 -- 97.9 & C  & 72.0 -- 77.9 \\
        A- & 90.0 -- 91.9 & C- & 70.0 -- 71.9 \\
        B+ & 88.0 -- 89.9 & D+ & 68.0 -- 69.9 \\
        B  & 82.0 -- 87.9 & D  & 62.0 -- 67.9 \\
        B- & 80.0 -- 81.9 & D- & 60.0 -- 61.9 \\
           &              & F  & Below 59.9   \\
\bottomrule
\end{tabular}
\label{tab:grading-scale}
\end{table}

\begin{table}[ht]
    \centering
    \caption{Grade Weights}
    \begin{tabular}{ll}
        \toprule
    \textbf{Assignment} & \textbf{Percentage} \\
    \midrule
    Video Lectures and Quizzes & 30\% \\
    Discussions & 20\% \\
    Writing Assignments & 50\% \\
    \bottomrule
    \end{tabular}
    \label{tab:grade-weights}
    \end{table}

\subsection*{Grade Disputes}

If you have a question about a grade, please contact the professor via email or \emph{Canvas} message. Please include a detailed explanation of your question and a copy of the assignment in question. Please allow up to 48 hours for a response. 


\section*{Course Schedule}

\subsection*{Week 1: Introduction and Foundations (12/16 - 12/22)}

\begin{itemize}
    \item \textbf{Module 1:} Course Introduction and Overview
        \begin{itemize}
            \item Introduction video
            \item Overview of Public Administration
        \end{itemize}
    \item \textbf{Module 2:} The Foundations of Public Administration
        \begin{itemize}
            \item Video 1: Basic Concepts
            \item Video 2: Historical Context
        \end{itemize}
    \item \textbf{Discussion:} Importance of Public Administration
    \item \textbf{Writing Assignment:} Reflective essay on the role of public administration in government
\end{itemize}

\subsection*{Week 2: Government Functions and Organizational Theory (12/23 - 12/29)}
\begin{itemize}
    \item \textbf{Module 3:} Government Functions
        \begin{itemize}
            \item Video 1: What Government Does
            \item Video 2: How Government Functions
        \end{itemize}
    \item \textbf{Module 4:} Organizational Theory
        \begin{itemize}
            \item Video 1: Basics of Organizational Theory
            \item Video 2: Application in Public Administration
        \end{itemize}
    \item \textbf{Discussion:} Real-world examples of government functions
    \item \textbf{Writing Assignment:} Case study analysis on organizational theory
\end{itemize}

\subsection*{Week 3: Government Structure and Human Capital (12/30 - 01/05)}
\begin{itemize}
    \item \textbf{Module 5:} The Executive Branch and Organization Problems
        \begin{itemize}
            \item Video 1: Structure of the Executive Branch
            \item Video 2: Common Organizational Problems
        \end{itemize}
    \item \textbf{Module 6:} Human Capital in Government
        \begin{itemize}
            \item Video 1: The Civil Service
            \item Video 2: Human Capital Management
        \end{itemize}
    \item \textbf{Discussion:} Challenges in managing human capital in public sector
    \item \textbf{Writing Assignment:} Research paper on civil service reforms
\end{itemize}

\subsection*{Week 4: Decision Making, Budgeting, and Performance (01/06 - 01/12)}
\begin{itemize}
    \item \textbf{Module 7:} Decision Making and Budgeting
        \begin{itemize}
            \item Video 1: Decision Making in Public Administration
            \item Video 2: Budgeting Process
        \end{itemize}
    \item \textbf{Module 8:} Implementation and Performance
        \begin{itemize}
            \item Video 1: Strategies for Effective Implementation
            \item Video 2: Measuring Performance in Public Sector
        \end{itemize}
    \item \textbf{Discussion:} Budgeting challenges in public sector
    \item \textbf{Writing Assignment:} Analysis of a government budget
\end{itemize}

\subsection*{Week 5: Regulation, Accountability, and Course Wrap-Up (01/13 - 01/19)}
\begin{itemize}
    \item \textbf{Module 9:} Regulation and the Courts
        \begin{itemize}
            \item Video 1: Role of Regulation
            \item Video 2: Public Administration and the Judiciary
        \end{itemize}
    \item \textbf{Module 10:} Accountability and Oversight
        \begin{itemize}
            \item Video 1: Concepts of Accountability
            \item Video 2: Mechanisms for Oversight
        \end{itemize}
    \item \textbf{Discussion:} Balancing regulation and innovation
    \item \textbf{Writing Assignment:} Comprehensive essay integrating course concepts
    \item \textbf{Course Wrap-Up:} Summary video, feedback, and next steps
\end{itemize}


\end{document}