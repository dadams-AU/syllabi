\documentclass[11pt, letterpaper]{article}
\usepackage{fancyhdr}
\usepackage[english]{babel}
\usepackage[T1]{fontenc}
\usepackage[margin=1in]{geometry}
\usepackage[sfdefault]{roboto}
\usepackage{xcolor}
\usepackage{url}
\usepackage[utf8]{inputenc}
\usepackage{tabularx}
\usepackage{booktabs}
\frenchspacing
\usepackage{multicol}
\usepackage{eso-pic}
\usepackage{caption}
\usepackage{subcaption}
\usepackage{setspace}
\usepackage{paralist}
\usepackage{quoting}
\usepackage{comment}
\usepackage{enumitem}
\usepackage{fancyhdr}
\usepackage{graphicx}
\pagestyle{fancy}
\renewcommand{\headrulewidth}{0pt}
\fancyhead{}
\fancyfoot{}
\fancyfoot[C]{\thepage}
\usepackage{float}
\usepackage{bookmark}
\renewcommand{\thesection}{\arabic{section}.}
\renewcommand{\thesubsection}{\thesection\arabic{subsection}}
\renewcommand{\thesubsubsection}{\thesubsection.\arabic{subsubsection}}
\usepackage{etoolbox}
\patchcmd{\thebibliography}{\section*{\refname}}{\section{\refname}}{}{}
\usepackage{hyperref}
\hypersetup{
  colorlinks=true,
  linkcolor=blue,
  filecolor=magenta,
  urlcolor=blue,
}
\usepackage[scaled]{helvet} % Load the helvet package
\renewcommand*\familydefault{\sfdefault} % Set the default font to be sans-serif



\begin{document}

\title{\includegraphics[width=8cm]{/home/snags/Nextcloud/Syllabi/Images/stacked.png} \\\textbf{Introduction to Public Administration}}
\author{CRJU/POSC 320 — Summer Session 2024}
\date{\textsc{Asynchronous Online Course} \\ 
May 28th through August 2nd}
\maketitle


\subsection*{Professor: David P. Adams, Ph.D.}

\subsubsection*{Contact Information:}


\begin{itemize}
	\item Office: 516 Gordon Hall
	\item Phone/Text: \href{tel:+16572784770}{(657) 278-4770}
	\item Zoom: \href{https://fullerton.zoom.us/j/3347502369}{\texttt{https://fullerton.zoom.us/j/3347502369}}
	\item website: \href{https://dadams.io}{\texttt{https://dadams.io}}
	\item email: \href{dpadams@fullerton.edu}{\texttt{dpadams@fullerton.edu}}
	\item Office Hours:
        \begin{itemize}
            \item Monday: 10:00am-12:00pm on Zoom
            \item Schedule meetings throughout the week: \href{https://dadams.io/appt}{\texttt{https://dadams.io/appt}}
        \end{itemize}  
\end{itemize}


\section{Catalog Description}

Introduces public administration through current trends and problems of public sector agencies in such areas as organization behavior, public budgeting, personnel, planning and policy making. Examples and cases from the Criminal Justice field. (POSC 320 and CRJU 320 are the same course.)

\section{Course Description}

Public administration plays an important role in our everyday lives. What do public administrators do? What makes this important field of government work? How are decisions made and how does the political environment impact those decisions? Our public administrators have to respond to various demands from United States residents and deal with situations and demands from abroad. The values we share interact and compete for the way our administrators create and implement policy. The core values of public administration include accountability, efficiency, and equity. We'll explore these topics and more as we engage in our class together. 

\vspace*{1em}

\noindent This course is an introduction to the study and practice---the science and art---of public administration. Students will be acquainted with the theoretical and practical aspects of public administration in the American political setting. Topics include organizational theory and practice, decision making, systems analysis, performance evaluation, and administrative and managerial improvement, among others. The emphasis is placed on understanding the roles and responsibilities of public administrators in a democratic political system. 
	

\section{Student Learning Objectives}

\begin{enumerate}
    \item Display a broad understanding of public administration and its role in a democratic society. 

    \item Demonstrate knowledge of the concepts and theories in public administration. 
    
    \item Identify complex problems that face public organizations.

    \item Exhibit critical thinking by interpreting information, comparing ideas, and developing opinions. 
    
    \item Contrast public and private administration with their corresponding benefits and shortfalls. 

    \item Demonstrate effective written communication skills. 

\end{enumerate}

\section{Required Text}

\begin{itemize}
    \item Kettl, Daniel F. 2023. \emph{Politics of the Administrative Process} (9th ed.) Washington, D.C.: CQ Press.
    \item Additional readings posted to Canvas
\end{itemize}

\section{Course Prerequisites}

POSC 100 and completion of G.E. Category D.1.  If you have not already taken and passed this course or its equivalent, you should not be enrolled in POSC/CRJU 320.

\section{General Education Information}


\subsection*{Requirements Satisfied}

	This course satisfies General Education Explorations in Social Sciences subarea D.4 for those using Catalog Years 2018 and later. The writing assignments in this course, including the policy memo papers and current event summaries described below, meet the requirement of UPS 411.201: 
	\begin{quote}Writing assignments in General Education courses shall involve the organization and expression of complex data or ideas and careful and timely evaluations of writing so that deficiencies are identified, and suggestions for improvement and/or for means of remediation are offered. Evaluations of the student's writing competence shall determine the final course grade\ldots .\end{quote}

\subsection*{General Education Student Learning Goals}

	Students completing courses in this subarea shall encounter the following learning goals:

\begin{enumerate}
	\item Examine problems, issues, and themes in the social sciences in greater depth; in a variety of cultural, historical, and geographical contexts; and from different disciplinary and interdisciplinary perspectives.
	\item Analyze and critically evaluate the application of social science concepts and theories to particular historical, contemporary, and future problems or themes, such as economic and environmental sustainability, globalization, poverty, and social justice.
	\item Analyze and critically evaluate constructs of cultural differentiation, including ethnicity, gender, race, class, and sexual orientation, and their effects on the individual and society.
	\item Apply theories and concepts from the social sciences to address historical, contemporary, and future problems confronting communities at different geographical scales, from local to global.
\end{enumerate}

\section{Technical Requirements}

\subsection*{Pollak Library Resources}

The Pollak Library provides a wide range of resources and services to support your research and learning. These resources include books, journals, databases, and research guides. You can access the library's resources online through the \href{http://www.library.fullerton.edu/}{Pollak Library website}. The library also offers research assistance through the \href{http://www.library.fullerton.edu/research/}{Research Assistance Program}. You can also access the \href{http://www.library.fullerton.edu/about/guidelines/online-instruction-guidelines.php}{library's online instruction guidelines} for help with online learning.

\subsection*{Canvas}

This course will be conducted using \href{https://csufullerton.instructure.com/}{Canvas}. You are responsible for checking \emph{Canvas} regularly for announcements, assignments, and other course materials. You are also responsible for ensuring that your \emph{Canvas} notifications are set to receive messages from the course. 

\subsection*{Kritik}

This course will use \href{https://kritik.io/}{Kritik} for peer review assignments. You are responsible for familiarizing yourself with the \href{https://help.kritik.io/en/collections/3549681-student-guide}{Kritik student guide} and following the instructions provided by the professor. 

\subsection*{Zoom}
This course will use \href{https://fullerton.zoom.us/}{Zoom} for office hours and appointments. You are responsible for ensuring that you have the necessary equipment and internet connection to participate in Zoom meetings. You are also responsible for familiarizing yourself with the \href{https://support.zoom.us/hc/en-us/articles/201362193-Joining-a-Meeting}{Zoom meeting guidelines} and following the instructions provided by the professor.

\subsection*{Minimum Technical Requirements}

To participate in this course, you will need the following minimum technical requirements:
\begin{itemize}
    \item A computer or tablet with a reliable internet connection
    \item A webcam and microphone
    \item A modern web browser (Chrome, Firefox, Safari, or Edge)
    \item Microsoft Word or a compatible word processing program
    \item Adobe Acrobat Reader or a compatible PDF reader
\end{itemize}

\noindent Long- and short-term computer and internet access loans are available through the \href{http://www.fullerton.edu/it/students/sgc/index.php}{Student Genius Center}.

\section{Technical Problems}

\subsection*{University IT Help Desk}

Contact the instructor immediately to document the problem if you encounter any technical difficulties. Then contact the \href{http://www.fullerton.edu/it/students/helpdesk/index.php}{Student IT Help Desk} for assistance. You can also call the Student IT Help Desk at (657) 278-8888, \href{mailto:StudentITHelpDesk@fullerton.edu}{email}, visit them at the Pollak Library North \href{http://www.fullerton.edu/it/students/sgc/index.php}{Student Genius Center}, or log on to the \href{http://my.fullerton.edu/}{my.fullerton.edu} portal and click ``Online IT Help'' followed by ``Live Chat''.

\subsection*{Canvas Support}

If you encounter any technical difficulties with Canvas, call the Canvas Support Hotline at 855-302-7528, visit the \href{https://community.canvaslms.com/docs/DOC-10720-67952720329}{Canvas Community}, or click the ``Help'' button in the lower left corner of Canvas and select ``Report a Problem''. The \href{https://cases.canvaslms.com/liveagentchat?chattype=student&sfid=001A000000YzcwQIAR}{Student Support Live Chat} is available 24 hours a day, 7 days a week.



\section*{University Student Policies}

In accordance with UPS 300.004, students must be familiar with certain policies applicable to all courses. Please review these policies as needed and visit the Cal State Fullerton website for \href{https://fdc.fullerton.edu/teaching/student-info-syllabi.html}{syllabus policies} for links to the following information:

\begin{enumerate}
    \item   University learning goals and program learning outcomes.
    \item	Learning objectives for each General Education (GE) category.
    \item	Guidelines for appropriate online behavior (netiquette).
    \item	Students’ rights to accommodations for documented special needs.
    \item   Campus student support measures, including Counseling \& Psychological Services, Title IV and Gender Equity, Diversity Initiatives and Resource Centers, and Basic Needs Services.
    \item	Academic integrity (refer to UPS 300.021).
    \item	Actions to take during an emergency.
    \item	Library services information.
    \item	Student Information Technology Services, including details on technical competencies and resources required for all students.
    \item	Software privacy and accessibility statements.
\end{enumerate}

\section{Course Student Policies}

\subsection*{Course Communication}
All course announcements and communications will be sent via \emph{Canvas} and university email. Students are responsible for regularly checking their \emph{Canvas} notifications and email. Students are also responsible for ensuring that their \emph{Canvas} notifications are set to receive messages from the course. Students are expected to check \emph{Canvas} and their email at least once daily.

\subsubsection*{Response Time}I will strive to respond to all student emails and \emph{Canvas} messages within 24 hours, except on weekends and holidays. If you do not receive a response within 24 hours, please send a follow-up message. If you do not receive a response within 48 hours, please send another follow-up message and contact me via phone or SMS text at (657) 278-4770.

\subsection*{Due Dates}
All assignments are due by 11:59 p.m. on the date specified in the course schedule. Late assignments will only be accepted if prior arrangements have been made with the professor. Students must submit all assignments on time and in the correct format. Failure to submit an assignment on time may result in a grade penalty.

\subsection*{Alternative Procedures for Submitting Work}
Students are expected to submit all assignments via \emph{Canvas}. If you cannot submit an assignment via \emph{Canvas}, please contact the professor to discuss alternative submission procedures.

\subsection*{Extra Credit}
Extra credit opportunities will not be offered in this course. All students will be graded based on the same criteria and standards.

\subsection*{Retention of Student Work}
Students are responsible for retaining copies of all assignments submitted in this course. Students should keep copies of all assignments until the end of the semester and verify that their assignments have been graded and returned before discarding them.

\subsection*{Academic Integrity}
Students are expected to adhere to the highest standards of academic integrity. Any student found to have engaged in academic dishonesty will be subject to the sanctions described in the \href{https://www.fullerton.edu/senate/publications_policies_resolutions/ups/UPS%20300/UPS%20300.021.pdf}{Academic Dishonesty Policy} (UPS 300.021). Academic dishonesty includes, but is not limited to, cheating, plagiarism, fabrication, facilitating academic dishonesty, and submitting previously graded work without prior authorization. Students are expected to be familiar with the university's policy on academic dishonesty and to adhere to this policy in all aspects of this course. Any student who has questions about the policy should ask the professor for clarification.

\subsection*{Plagiarism}
Plagiarism is a serious violation of academic integrity and will not be tolerated in this course. Plagiarism includes, but is not limited to, copying and pasting text from sources without proper citation, paraphrasing text from sources without proper citation, and submitting work that is not your own. Students are expected to properly cite all sources used in their work and to submit original work. Failure to do so may result in a failing grade for the assignment and further disciplinary action.

\subsection*{Written Work}
All written work must be submitted in a professional format, including proper grammar, spelling, and punctuation. Written work must also be properly cited using the appropriate citation style. Students are expected to follow the guidelines for written work provided by the professor and to seek clarification if they have questions about the requirements.

\subsection*{AI Generated Text} 
Large language models, such as OpenAI's ChatGPT-4o, Anthropic's Claude Opus 3, and others, have made it easier to generate text that mimics human writing. While these models can be useful for generating ideas and content, they can also be misused to produce work that is not original. Students are expected to use AI-generated text responsibly and to ensure that all work submitted in this course is their own. Failure to do so may result in a failing grade for the assignment and further disciplinary action.

\subsection*{Participation}

Students are expected to participate in all course activities. This includes completing all assigned readings, watching all assigned videos, and participating in all discussions. Students are expected to participate in discussions in a professional and respectful manner. Students are expected to be familiar with the university policy on netiquette and to adhere to this policy in all aspects of this course. Any student who has questions about the policy should ask the professor for clarification. 

\begin{quote}Netiquette refers to a set of behaviors that are appropriate for online activity (e.g., social media, email, discussions, presentations). All personnel at Cal State Fullerton are expected to demonstrate appropriate online behavior at all times. A good summary of netiquette can be found in the \href{https://canvashelp.fullerton.edu/m/Student/l/1336786-student-what-is-netiquette}{CSUF Canvas self-help guides}, which adapts ten rules to the online course situation from the website for the book \href{http://www.albion.com/netiquette/corerules.html}{Netiquette by Virginia Shea} and other sources referenced at the bottom of the guide.\end{quote}

\noindent This course will be delivered asynchronously online via \emph{Canvas}. Students are expected to log on to \emph{Canvas} at least once daily to check for announcements and updates. Students are also expected to check their university email at least once daily.

\section{Course Structure}

This course is divided into 10 modules. Each module will include a video lecture, assigned readings, a discussion, and a writing assignment. With the exception of the first week, each module will run from Sunday to Saturday. Students are expected to complete all assigned readings, watch all video lectures, participate in all discussions, and complete all writing assignments.

\section*{Course Requirements}

\subsection*{Readings}

Each module will include assigned readings. Students are expected to complete all assigned readings. Readings consist of the textbook material covered in each module and additional readings posted to \emph{Canvas}. Complete the readings before watching the video lectures and participating in the discussions.

\subsection*{Video Lectures and Quizzes}
\begin{itemize}
    \item \textbf{Overview:} Each module includes a video lecture that covers the assigned readings. You are expected to watch the video lecture and complete the corresponding quiz for each module. The video lectures will be available on \emph{Canvas} and YouTube, while the quizzes will be administered through \emph{Canvas}. \textbf{The deadline for completing all quizzes is 11:59 p.m. every Thursday}, with the exception of July 4th, when the deadline will be extended to Friday, July 5th.
    \item \textbf{Grading:} There are a total of 10 quizzes throughout the course, each worth 10 points. Your quiz grade will be based on completion and accuracy. To maximize your learning and performance, it is essential to watch each video lecture attentively and complete each quiz to the best of your ability.
\end{itemize}

\subsection*{Discussion Assignments:}
    \begin{itemize}
        \item \textbf{Overview:} Throughout the course, you will participate in five discussion assignments. These discussions are designed to foster engagement with the course material, encourage critical thinking, and facilitate interaction among your peers. Each discussion assignment is worth 10 points.
        \item \textbf{Grading:} Each discussion assignment is worth 10 points. Your discussion posts will be graded using the following criteria:
            \begin{itemize}
                \item Relevance and insight
                \item Evidence and examples
                \item Clarity and coherence
                \item Engagement and interaction
            \end{itemize}
        \item \textbf{Due Dates:} There are five discussions throughout the term. Students are expected to participate in all discussions. Discussions will be administered via \emph{Canvas}. \textbf{All discussions must be completed by Saturday at 11:59 p.m. for the weeks they are assigned.}
    \end{itemize}

\subsection*{Term Paper: Organizational Profile}

\begin{itemize}
    \item \textbf{Overview:} In this term paper, you will have the opportunity to synthesize the key concepts covered in this course by conducting an in-depth examination of a public sector organization. You will explore the structure, functions, challenges, and strategies of a specific department within a city government or a department, agency, or bureau at the state or federal level.
    \item \textbf{Assignment Structure:} The term paper will be divided into four milestones, each focusing on different aspects of the organizational profile. You will submit your work for peer review at each milestone, allowing you to incorporate feedback and improve your paper throughout the process.
    \begin{itemize}
        \item Milestone 1: Introduction and Organizational Overview
        \item Milestone 2: Financial Management and Human Capital
        \item Milestone 3: Regulatory Role, Accountability, and Challenges
        \item Milestone 4: Conclusion and Final Draft
    \end{itemize}
    \item \textbf{Peer Review Process:} At each milestone, you will submit your work to Kritik for peer review. You will be required to review and provide constructive feedback on the submissions of your peers. The peer review process is designed to help you refine your analysis and improve the quality of your final term paper.
    \item \textbf{Learning Objectives:} By completing this term paper assignment, you will:
        \begin{itemize}
            \item Demonstrate an in-depth understanding of the structure, functions, and challenges of a public sector organization.
            \item Analyze the organization's budgeting process, financial management, human resource policies, and regulatory functions.
            \item Examine accountability measures and oversight mechanisms within the organization.
            \item Identify key challenges and opportunities faced by the organization and propose strategies for improvement.
            \item Develop your skills in research, critical analysis, and academic writing.
        \end{itemize}
    \item \textbf{Final Submission:} Your final term paper should be 8-10 pages long, excluding the title page and references, and formatted according to APA guidelines. It should incorporate feedback received during the peer review process and demonstrate a comprehensive understanding of the chosen organization's public administration practices.
    \item \textbf{Grading Criteria:} Your term paper will be graded based on the following criteria:
        \begin{itemize}
            \item Depth and accuracy of organizational analysis
            \item Clarity and coherence of arguments
            \item Use of evidence and examples to support points
            \item Compliance with APA style guidelines
            \item Quality of writing, including organization, grammar, and spelling
            \item Successful incorporation of peer feedback
        \end{itemize}
    \item \textbf{Detailed instructions for each milestone and the peer review process will be provided on Kritik. This assignment provides a unique opportunity to learn from your peers and improve your work through an iterative process. Embrace the feedback, engage actively in the peer review, and strive to produce a high-quality organizational profile that showcases your understanding of public administration concepts.}
    \item \textbf{Due Dates:} The term paper will be divided into four milestones, each with its own due date. The final submission will at the end of week nine (9) the course. The submission calendar for each milestone is as follows:
        \begin{itemize}
            \item Milestone 1: Introduction and Organizational Overview
                \begin{itemize}
                    \item Assigned in Week 2 
                    \item Due at the end of Week 3
                    \item Peer review in Week 4
                \end{itemize}
            \item Milestone 2: Financial Management and Human Capital 
                \begin{itemize}
                    \item Assigned in Week 4
                    \item Due at the end of Week 5
                    \item Peer review in Week 6
                \end{itemize}
            
            \item Milestone 3: Regulatory Role, Accountability, and Challenges
                \begin{itemize}
                    \item Assigned in Week 6
                    \item Due at the end of Week 7
                    \item Peer review in Week 8
                \end{itemize}
        
            \item Milestone 4: Conclusion and Final Draft
                \begin{itemize}
                    \item Assigned in Week 8
                    \item Due at the end of Week 9
                    \item Peer review in Week 10
                \end{itemize}
            \end{itemize}
    \item \textbf{Grading}: The term paper is worth 40\% of the final grade. The term paper will be graded based on the quality of the analysis, the depth of understanding of the organization, the clarity of the arguments, the use of evidence and examples, compliance with APA style guidelines, and the quality of writing. The peer review process will also be taken into account in the grading of the final submission.

\end{itemize}

\section{Grades}

\subsection*{Grading Scale and Grade Weights}  
The grading scale is shown in Table~\ref{tab:grading-scale}. Grades will be given based on Table~\ref{tab:grade-weights} weights.

\begin{table}[ht]
\centering
\caption{Grading Scale}
\begin{tabular}{lll}
\toprule
\textbf{Percentage Range} & \textbf{Grade} & \textbf{Grade Point} \\
\midrule
97-100 & A+ & 4.0 \\
93-96 & A & 4.0 \\
90-92 & A- & 3.7 \\
87-89 & B+ & 3.3 \\
83-86 & B & 3.0 \\
80-82 & B- & 2.7 \\
77-79 & C+ & 2.3 \\
73-76 & C & 2.0 \\
70-72 & C- & 1.7 \\
67-69 & D+ & 1.3 \\
63-66 & D & 1.0 \\
60-62 & D- & 0.7 \\
0-59 & F & 0.0 \\
\bottomrule
\end{tabular}
\label{tab:grading-scale}
\end{table}

\begin{table}[ht]
    \centering
    \caption{Grade Weights}
    \begin{tabular}{ll}
        \toprule
    \textbf{Assignment} & \textbf{Percentage} \\
    \midrule
    Quizzes & 40\% \\
    Discussions & 20\% \\
    Term Paper & 40\% \\
    \bottomrule
    \end{tabular}
    \label{tab:grade-weights}
    \end{table}

\subsection*{Grade Disputes}

If you have a question about a grade, please contact the professor via email or \emph{Canvas} message. Please include a detailed explanation of your question and a copy of the assignment in question. Please allow up to 48 hours for a response. 


\section{Course Schedule}

\subsection*{Week 1: Introduction and Overview (Tuesday, 5/28 - Saturday, 6/1)}
\begin{itemize}
    \item \textbf{Module 1:} Course Introduction and Overview
    \begin{itemize}
        \item Introduction video
        \item Overview of Public Administration
    \end{itemize}
    \item \textbf{Assignment:} \href{https://us.kritik.io/course/cltwics8d035m146piz0jn5t0/assignment/cltwics8j035p146pdgj8gj1t?resultsPerPage=50&page=1&filterBy=SHOW_ALL&searchString=&sortBy=NAME&sortOrder=1&viewType=Progress}{Kritik registration and introduction video}
\end{itemize}

\subsection*{Week 2: The Foundations of Public Administration (Sunday, 6/2 - Saturday, 6/8)}
\begin{itemize}
    \item \textbf{Module 2:} The Foundations of Public Administration
    \begin{itemize}
        \item Video 1: Basic Concepts
        \item Video 2: Historical Context
    \end{itemize}
    \item \textbf{Discussion:} Importance of Public Administration


\end{itemize}

\subsection*{Week 3: Government Functions (Sunday, 6/9 - Saturday, 6/15)}
\begin{itemize}
    \item \textbf{Module 3:} Government Functions
    \begin{itemize}
        \item Video 1: What Government Does
        \item Video 2: How Government Functions
    \end{itemize}
    \item \textbf{Discussion:} Real-world examples of government functions
    \item \textbf{Assignment Due:} \href{https://us.kritik.io/course/cltwics8d035m146piz0jn5t0/assignment/clwmrwdll013ufih40nlg0gar?resultsPerPage=50&page=1&filterBy=SHOW_ALL&searchString=&sortBy=NAME&sortOrder=1&viewType=Progress}{Term Paper Milestone 1: Introduction and Organizational Overview}
\end{itemize}

\subsection*{Week 4: Organizational Theory (Sunday, 6/16 - Saturday, 6/22)}
\begin{itemize}
    \item \textbf{Module 4:} Organizational Theory
    \begin{itemize}
        \item Video 1: Basics of Organizational Theory
        \item Video 2: Application in Public Administration
    \end{itemize}
\end{itemize}

\subsection*{Week 5: The Executive Branch and Organization Problems (Sunday, 6/23 - Saturday, 6/29)}
\begin{itemize}
    \item \textbf{Module 5:} The Executive Branch and Organization Problems
    \begin{itemize}
        \item Video 1: Structure of the Executive Branch
        \item Video 2: Common Organizational Problems
    \end{itemize}
    \item \textbf{Assignment Due:} \href{hhttps://us.kritik.io/course/cltwics8d035m146piz0jn5t0/assignment/clwms1l5k0125f8blzdaawh9x?resultsPerPage=50&page=1&filterBy=SHOW_ALL&searchString=&sortBy=NAME&sortOrder=1&viewType=Progress}{Term Paper Milestone 2: Financial Management and Human Capital}
\end{itemize}

\subsection*{Week 6: Human Capital in Government (Sunday, 6/30 - Saturday, 7/6)}
\begin{itemize}
    \item \textbf{Module 6:} Human Capital in Government
    \begin{itemize}
        \item Video 1: The Civil Service
        \item Video 2: Human Capital Management
    \end{itemize}
    \item \textbf{Discussion:} Challenges in managing human capital in public sector
\end{itemize}

\subsection*{Week 7: Decision Making and Budgeting (Sunday, 7/7 - Saturday, 7/13)}
\begin{itemize}
    \item \textbf{Module 7:} Decision Making and Budgeting
    \begin{itemize}
        \item Video 1: Decision Making in Public Administration
        \item Video 2: Budgeting Process
    \end{itemize}
    \item \textbf{Discussion:} Budgeting challenges in public sector
    \item \textbf{Assignment Due:} \href{https://us.kritik.io/course/cltwics8d035m146piz0jn5t0/assignment/clwms5jnp010712k819w1oes6?resultsPerPage=50&page=1&filterBy=SHOW_ALL&searchString=&sortBy=NAME&sortOrder=1&viewType=Progress}{Term Paper Milestone 3: Regulatory Role, Accountability, and Challenges}
\end{itemize}

\subsection*{Week 8: Implementation and Performance (Sunday, 7/14 - Saturday, 7/20)}
\begin{itemize}
    \item \textbf{Module 8:} Implementation and Performance
    \begin{itemize}
        \item Video 1: Strategies for Effective Implementation
        \item Video 2: Measuring Performance in Public Sector
    \end{itemize}
\end{itemize}

\subsection*{Week 9: Regulation and the Courts (Sunday, 7/21 - Saturday, 7/27)}
\begin{itemize}
    \item \textbf{Module 9:} Regulation and the Courts
    \begin{itemize}
        \item Video 1: Role of Regulation
        \item Video 2: Public Administration and the Judiciary
    \end{itemize}
    \item \textbf{Discussion:} Balancing regulation and innovation
    \item \textbf{Assignment Due:} \href{https://us.kritik.io/course/cltwics8d035m146piz0jn5t0/assignment/clwms87ek014pfih46sq1qe3p?resultsPerPage=50&page=1&filterBy=SHOW_ALL&searchString=&sortBy=NAME&sortOrder=1&viewType=Progress}{Term Paper Milestone 4: Conclusion and Final Draft} 
\end{itemize}

\subsection*{Week 10: Accountability and Oversight (Sunday, 7/28 - Friday, 8/2)}
\begin{itemize}
    \item \textbf{Module 10:} Accountability and Oversight
    \begin{itemize}
        \item Video 1: Concepts of Accountability
        \item Video 2: Mechanisms for Oversight
    \end{itemize}
    \item \textbf{Course Wrap-Up:} Summary video, feedback, and next steps
\end{itemize}


\end{document}