% CSUF Accessible Syllabus Shell
% This is a template for creating an accessible syllabus that meets
% CSU Fullerton requirements and ADA/PDF-UA accessibility standards
% 
% Instructions:
% 1. Replace all CAPS placeholders with your course information
% 2. Delete sections marked as optional if they don't apply
% 3. Compile with LuaLaTeX for best accessibility support
% 4. Check accessibility using Adobe Acrobat's accessibility checker
%
% ACCESSIBILITY NOTES:
% - This template addresses common PDF accessibility issues:
%   * Proper document title in PDF properties
%   * Tagged annotations for links
%   * Correct list structure tagging
%   * Logical reading order (tab order)
% - After compiling, run Adobe Acrobat's accessibility checker
% - Fix any remaining issues using Acrobat's accessibility tools
%
% Created by: David P. Adams, Ph.D.
% License: CC BY-NC-SA 4.0

\documentclass[12pt]{article}

% ========== PACKAGES FOR ACCESSIBILITY ==========
% These packages enable PDF tagging for screen readers
\usepackage[english]{babel}
\usepackage[tagged, highstructure]{accessibility}
% Load hyperref with proper options for accessibility
\usepackage[pdfusetitle, 
            pdflang=en-US,
            pdfstartview=FitH,
            pdfdisplaydoctitle=true]{hyperref}

% ========== LAYOUT AND FONTS ==========
\usepackage[T1]{fontenc}
\usepackage[utf8]{inputenc}
\usepackage{geometry}
\geometry{margin=1in}
\usepackage{lmodern}
\usepackage{setspace}
\onehalfspacing

% Use Helvetica font (similar to Calibri in Word)
\renewcommand{\rmdefault}{phv} % Helvetica
\renewcommand{\sfdefault}{phv} % Helvetica
\renewcommand{\ttdefault}{pcr} % Courier

% ========== OTHER PACKAGES ==========
\usepackage{enumitem}      % Better list formatting
\usepackage{graphicx}      % For including images
\usepackage{array, booktabs, longtable}  % For tables
\usepackage{caption}       % For table/figure captions

% ========== PDF METADATA ==========
% Update these fields with your course information
\hypersetup{
  unicode=true,
  colorlinks=true,
  linkcolor=blue,
  urlcolor=blue,
  citecolor=blue,
  pdfauthor={YOUR NAME},
  pdftitle={COURSE PREFIX \& NUMBER: COURSE TITLE},
  pdfsubject={California State University, Fullerton Course Syllabus},
  pdfkeywords={CSUF, Syllabus, YOUR DEPARTMENT}
}

% ========== DOCUMENT INFORMATION ==========
% Replace these with your course information
\title{PREFIX \& Number, \textit{Course Title}}
\author{}  % Leave empty - instructor info goes in Faculty Information section
\date{Term and Year}

% ========== ACCESSIBILITY FIXES ==========
% Set list depth for proper tagging
\setlistdepth{3}
% Enable proper link annotations
\hypersetup{pdfborderstyle={/S/U/W 1}}

\begin{document}

% Set the PDF document title properly
\pdfinfo{
  /Title (PREFIX & Number: Course Title - Term Year)
}

% ========== CSUF HEADER (OPTIONAL) ==========
% If you have the CSUF logo file, uncomment the following lines
% \begin{center}
% \includegraphics[width=2.75in, alt={Cal State Fullerton wordmark}]{csuf_logo.png}
% \end{center}

\maketitle

% ========== SECTION 1: FACULTY INFORMATION ==========
\section*{Faculty Information}
% Replace placeholders with your information
\noindent \textbf{Instructor:} YOUR NAME \\
\noindent \textbf{Office:} BUILDING AND ROOM \\
\noindent \textbf{Phone:} (657) 278-XXXX \\
\noindent \textbf{Email:} youremail@fullerton.edu \\
\noindent \textbf{Office hours:} DAYS AND TIMES, by appointment

% ========== SECTION 2: COURSE COMMUNICATION ==========
\section*{Course Communication}
% Describe how students should contact you and your response time
YOUR PREFERRED COMMUNICATION METHOD AND POLICY HERE

% ========== SECTION 3: TECHNICAL PROBLEMS ==========
\section*{Technical Problems}
If you encounter any technical difficulties, contact the instructor immediately to document the problem. Then, contact: \href{http://www.fullerton.edu/it/students/helpdesk/index.php}{student IT help desk}, \href{mailto:StudentITHelpDesk@fullerton.edu}{email}, phone (657) 278-8888, walk-in \href{http://www.fullerton.edu/it/students/sgc/index.php}{student genius center}, online chat - log into \href{http://my.fullerton.edu}{portal}; click ``Online IT Help''; click ``Live Chat.''

\vspace{0.5em}
\noindent \textbf{For issues with Canvas:} Canvas Support Hotline (855) 302-7528, \href{https://cases.canvaslms.com/liveagentchat?chattype=student&sfid=001A000000YzcwQIAR}{student support chat}

\vspace{0.5em}
\noindent \textbf{Alternative plan for submitting work:} DESCRIBE YOUR BACKUP PLAN

\vspace{0.5em}
\noindent \textbf{Response time:} YOUR RESPONSE TIME POLICY

% ========== SECTION 4: COURSE INFORMATION ==========
\section*{Course Information}
% Fill in all course details
\noindent \textbf{Prefix, number, title:} PREFIX \#\#\#, \textit{Course Title} \\
\noindent \textbf{Meeting times with modality, day(s), time(s), and location (if synchronous):} 
% Choose ONE modality from UPS 411.104:
% - In-Person (0-20% online)
% - Hybrid (mostly in-person, 21%-49% online)
% - Hybrid (mostly online, 50%-99% online)  
% - Fully online (100% online)
YOUR MODALITY, DAYS, TIMES, LOCATION

\vspace{0.5em}
\noindent \textbf{Zoom:} YOUR ZOOM LINK IF APPLICABLE \\
\noindent \textbf{Course requisite(s):} LIST PREREQUISITES/COREQUISITES OR "none" \\
\noindent \textbf{Catalog description:} COPY CATALOG DESCRIPTION VERBATIM (40 words max) \\
\noindent \textbf{Additional description:} OPTIONAL EXPANDED DESCRIPTION \\
\noindent \textbf{Policy regarding the use of generative AI:} YOUR AI POLICY \\
\noindent \textbf{Course materials and equipment:} ~ \\
\noindent \textbf{Required text(s):} LIST REQUIRED TEXTS OR "none" \\
\noindent \textbf{Recommended text(s):} LIST RECOMMENDED TEXTS OR DELETE THIS LINE \\
\noindent \textbf{Other course materials and equipment:} LIST OTHER MATERIALS OR "none" \\
\noindent \textbf{Zero cost:} IF APPLICABLE, NOTE THAT THIS IS A ZERO-COST COURSE

\vspace{1em}
\noindent \textbf{Student Learning Outcomes:}
% List your course learning outcomes
\begin{enumerate}
\item LEARNING OUTCOME 1
\item LEARNING OUTCOME 2
\item LEARNING OUTCOME 3
\item ADD MORE AS NEEDED
\end{enumerate}

% ========== SECTION 5: GRADING POLICIES AND STANDARDS ==========
\section*{Grading Policies and Standards}

% Part a: Grading Scale
\noindent \textbf{a. Grading scale:}

% Example grading scale - modify as needed
\begin{center}
\begin{tabular}{|l|c||l|c|}
\hline
\textbf{Grade} & \textbf{Percent} & \textbf{Grade} & \textbf{Percent} \\
\hline
A+ & 98.0--100.0 & C+ & 77.0--79.9 \\
A  & 93.0--97.9  & C  & 73.0--76.9 \\
A- & 90.0--92.9  & C- & 70.0--72.9 \\
B+ & 87.0--89.9  & D+ & 67.0--69.9 \\
B  & 83.0--86.9  & D  & 63.0--66.9 \\
B- & 80.0--82.9  & D- & 60.0--62.9 \\
   &             & F  & 0.0--59.9 \\
\hline
\end{tabular}
\end{center}

% Part b: Required Course Assignments
\vspace{1em}
\noindent \textbf{b. Required Course Assignments:}

% Option 1: Use a table
\begin{center}
\begin{tabular}{|l|c|}
\hline
\textbf{Assignment} & \textbf{Percentage} \\
\hline
ASSIGNMENT TYPE 1 & XX\% \\
ASSIGNMENT TYPE 2 & XX\% \\
ASSIGNMENT TYPE 3 & XX\% \\
ASSIGNMENT TYPE 4 & XX\% \\
\hline
\textbf{Total} & 100\% \\
\hline
\end{tabular}
\end{center}

% Option 2: Or list assignments individually (uncomment if preferred)
% \noindent \textbf{Assignment 1:} Description, XX\% of grade, due date \\
% \noindent \textbf{Assignment 2:} Description, XX\% of grade, due date \\
% \noindent \textbf{Assignment 3:} Description, XX\% of grade, due date

% Part c: Attendance and Participation
\vspace{1em}
\noindent \textbf{c. Attendance and Participation policy:}
YOUR ATTENDANCE AND PARTICIPATION EXPECTATIONS

% Part d: Examination dates
\vspace{1em}
\noindent \textbf{d. Examination dates:}
% Remember: No major exams in week 15 unless there's also a final
% Finals must be in week 16 only
LIST EXAM DATES OR "No exams in this course"

% Part e: Make-up and late submission policy
\vspace{1em}
\noindent \textbf{e. Make-up and late submission policy:}
YOUR POLICY FOR LATE WORK AND MAKE-UP EXAMS

% Part f: Authentication of student work
\vspace{1em}
\noindent \textbf{f. Authentication of student work:}
YOUR POLICY FOR VERIFYING STUDENT WORK

% Part g: Extra credit
\vspace{1em}
\noindent \textbf{g. Extra credit:}
YOUR EXTRA CREDIT POLICY OR "No extra credit offered"

% Part h: Retention of student work
\vspace{1em}
\noindent \textbf{h. Retention of student work:}
YOUR POLICY FOR KEEPING STUDENT WORK

% ========== GRADUATE CREDIT (DELETE IF NOT APPLICABLE) ==========
% \vspace{1em}
% \noindent \textbf{Additional assignments for graduate students:}
% If this is a 400-level course available for graduate credit,
% describe the additional assignment(s) required for graduate students

% ========== SECTION 7: ACADEMIC INTEGRITY ==========
\section*{Academic Integrity}
% Describe your expectations for academic honesty
YOUR ACADEMIC INTEGRITY POLICY AND CONSEQUENCES FOR VIOLATIONS

% ========== SECTION 8: TECHNICAL COMPETENCIES (DELETE IF NOT APPLICABLE) ==========
% \section*{Technical Competencies}
% List any technical skills beyond those expected of all students
% YOUR SPECIAL TECHNICAL REQUIREMENTS

% ========== SECTION 9: STUDENT RESOURCES WEBSITE ==========
\section*{Student Resources Website}
It is the student's responsibility to read and understand the required and important \href{https://fdc.fullerton.edu/teaching/student-info-syllabi.html}{student information for course syllabi}. Included is information about:

\begin{itemize}[noitemsep]
\item University learning goals
\item General Education learning objectives
\item Netiquette/appropriate online behavior
\item Students' rights to accommodations
\item Campus student support resources
\item Academic integrity
\item Emergency preparedness/what to do
\item Library services
\item Student IT services and competencies
\item Software privacy and accessibility
\item Accessibility statement
\item Diversity statement
\item Land acknowledgement
\item Final exam schedule
\item Semester calendar
\end{itemize}

% ========== SECTION 10: CLASSROOM MANAGEMENT ==========
\section*{Classroom Management}
% Add any additional classroom policies or "rules"
YOUR CLASSROOM POLICIES (cell phones, recording, etc.)

% ========== SECTION 11: GE REQUIREMENTS (DELETE IF NOT A GE COURSE) ==========
% \section*{General Education Requirements}
% \noindent \textbf{GE requirement(s) that this course meets:} GE CATEGORY
% 
% \noindent \textbf{How the GE writing requirement will be met and assessed:}
% DESCRIBE HOW WRITING WILL BE INCORPORATED AND ASSESSED
% 
% \noindent \textbf{GE grading standard:}
% % For Golden Four (A.1, A.2, A.3, B.4):
% A grade of ``C-'' (1.7) or higher is required to meet this General Education requirement. A grade of ``D+'' (1.3) or below will not satisfy this General Education requirement.
% 
% % For all other GE courses:
% % A grade of ``D'' (1.0) or higher is required to meet this General Education requirement. A grade of ``D-'' (0.7) or below will not satisfy this General Education requirement.

% ========== SECTION 12: UPPER-DIVISION WRITING (DELETE IF NOT APPLICABLE) ==========
% \section*{Upper-Division Writing Course Requirements}
% DESCRIBE HOW THIS COURSE MEETS UPPER-DIVISION WRITING REQUIREMENTS

% ========== SECTION 13: CALENDAR/SCHEDULE ==========
\section*{Calendar of Topics / Schedule of Classes}
% Remember: 16 weeks total (15 instruction + 1 finals)
% Don't number spring/fall break week

% Option 1: Weekly list format
\noindent \textbf{Week 1, MM/DD}\\
Topic(s): \\
Reading(s): \\
Assignment(s) Due: \\

\noindent \textbf{Week 2, MM/DD}\\
Topic(s): \\
Reading(s): \\
Assignment(s) Due: \\

% Continue for all 16 weeks...

% Option 2: Table format (uncomment if preferred)
% \begin{center}
% \begin{tabular}{|c|c|l|l|l|}
% \hline
% \textbf{Week} & \textbf{Date} & \textbf{Topic} & \textbf{Readings} & \textbf{Assignments Due} \\
% \hline
% 1 & MM/DD & & & \\
% \hline
% 2 & MM/DD & & & \\
% \hline
% % ... continue for all weeks
% \hline
% 16 & MM/DD & Final Exam & & \\
% \hline
% \end{tabular}
% \end{center}

\end{document}