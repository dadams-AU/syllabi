% CSUF Accessible Syllabus Shell
% This is a template for creating an accessible syllabus that meets
% CSU Fullerton requirements and ADA/PDF-UA accessibility standards
% 
% Instructions:
% 1. Replace all CAPS placeholders with your course information
% 2. Delete sections marked as optional if they don't apply
% 3. Compile with LuaLaTeX for best accessibility support
% 4. Check accessibility using Adobe Acrobat's accessibility checker
%
% ACCESSIBILITY NOTES:
% - This template addresses common PDF accessibility issues:
%   * Proper document title in PDF properties
%   * Tagged annotations for links
%   * Correct list structure tagging
%   * Logical reading order (tab order)
% - After compiling, run Adobe Acrobat's accessibility checker
% - Fix any remaining issues using Acrobat's accessibility tools
%
% Created by: David P. Adams, Ph.D.
% License: CC BY-NC-SA 4.0
% Document metadata for PDF accessibility and compliance
\DocumentMetadata{
  pdfstandard=UA-2,               % PDF/UA-2 standard for accessibility
  pdfversion=2.0,                 % PDF version 2.0
  lang=en-US                      % Document language set to US English
}

% Document class and font size
\documentclass[12pt]{article}     % Standard article class with 12pt font size

% Language settings
\usepackage[american]{babel}      % Set language to American English

% ================================
% FONT SETTINGS
% ================================

\usepackage{fontspec}

% === Recommended (Sans-serif body text) ===
\IfFontExistsTF{TeX Gyre Heros}{
  \setmainfont{TeX Gyre Heros} % Default to sans-serif
}{
  \setmainfont{Latin Modern Sans} % Fallback sans
}

% === Alternative (Serif body text) ===
% Uncomment below to switch back to serif body text:
% \IfFontExistsTF{TeX Gyre Termes}{
%   \setmainfont{TeX Gyre Termes} % Default to serif
% }{
%   \setmainfont{Latin Modern Roman} % Fallback serif
% }

% Monospaced font (for code, etc.)
\IfFontExistsTF{TeX Gyre Cursor}{
  \setmonofont{TeX Gyre Cursor}
}{
  \setmonofont{Latin Modern Mono}
}


% Page layout and spacing
\usepackage{geometry}             % Adjust page margins
\geometry{margin=1in}             % Set 1-inch margins
\usepackage{setspace}             % Control line spacing
\onehalfspacing                   % Set line spacing to 1.5

% List, graphics, table, and caption settings
\usepackage{enumitem}             % Enhanced list customization
\setlist[itemize]{                % Default settings for itemized lists
  itemsep=0pt,                    % No extra space between items
  parsep=0pt,                     % No extra space between paragraphs
  topsep=0.25\baselineskip        % Small space before the list
}
\newlist{flatlist}{itemize}{1}    % Define a bulletless, flush-left list
\setlist[flatlist]{               % Settings for flatlist
  label={},                       % No bullet or label
  leftmargin=0pt,                 % No left margin
  itemsep=0pt, parsep=0pt,        % No extra spacing
  topsep=0pt                      % No space before the list
}
\setlistdepth{3}                  % Ensure consistent list tagging depth
\usepackage{xcolor}               % Colors (used by hyperref and optional table shading)
\usepackage{graphicx}             % Include graphics
\usepackage{array, booktabs, longtable} % Enhanced table formatting
\usepackage{caption}              % Customize captions
\usepackage{tabularray}           % Modern table package
\UseTblrLibrary{booktabs}         % Use booktabs for better table rules

% Tagging for accessibility
% Check if the tagpdf package is available
\IfFileExists{tagpdf-base.sty}{
  % If available, load the tagpdf package for accessibility tagging
  \usepackage{tagpdf}
  % Activate all tagging features and enable interword spacing for better accessibility
  \tagpdfsetup{activate-all=true, interwordspace=true}
}{
  % If the tagpdf package is not installed, provide a fallback
  % Define a dummy \tagpdfsetup command to avoid compilation errors
  \newcommand\tagpdfsetup[1]{}
}


% Hyperlinks and PDF metadata
\usepackage[
  pdfusetitle,                    % Use \title and \author for PDF metadata
  pdflang=en-US,                  % Set PDF language to US English
  pdfstartview=FitH,              % Open PDF with horizontal fit
  pdfdisplaydoctitle=true         % Display document title in PDF viewer
]{hyperref}

\hypersetup{
  unicode=true,                   % Enable Unicode support
  colorlinks=true,                % Use colored links
  linkcolor=blue,                 % Color for internal links
  urlcolor=blue,                  % Color for URLs
  citecolor=blue,                 % Color for citations
  pdfsubject={California State University, Fullerton Course Syllabus}, % PDF subject
  % NOTE: Keep the PDF metadata below in sync with the course fields further down.
  % These values show up in the PDF's "Properties" and help screen readers.
  pdftitle={\CourseCode: \CourseTitle}, % PDF title
  pdfauthor={\InstructorName},           % PDF author metadata
  pdfkeywords={CSUF, Syllabus, \DepartmentKeywords, \CourseCode}, % PDF keywords
  pdfborderstyle={/S/U/W 1}       % Underline links in PDF
}
\urlstyle{same}                   % Use the same font style for URLs

% ================================
% COURSE FIELDS (edit these)
% ================================
% These fields are used for BOTH the visible title block and the PDF metadata.
\newcommand{\CourseCode}{PREFIX\ \#\#\#}
\newcommand{\CourseTitle}{Course Title}
\newcommand{\CourseTerm}{Term and Year}
\newcommand{\InstructorName}{YOUR NAME}
\newcommand{\DepartmentKeywords}{YOUR DEPARTMENT}

% Document title and metadata (visible)
\title{\CourseCode: \CourseTitle}
\author{}                         % Leave author blank (info in body)
\date{\CourseTerm}

% Optional logo configuration (must include alt text for accessibility)
\newcommand{\CSUFLogoPath}{csuf_logo.png}
\newcommand{\CSUFLogoAltText}{Cal State Fullerton wordmark}

% Begin the document
\begin{document}

% ========== CSUF HEADER (OPTIONAL) ==========
% If \CSUFLogoPath exists, it will be included with alt text.
\makeatletter
\renewcommand{\maketitle}{%
  \begin{center}
    \IfFileExists{\CSUFLogoPath}{%
      \includegraphics[width=2.75in, alt={\CSUFLogoAltText}]{\CSUFLogoPath}\par
      \vspace{0.75em}
    }{}
    {\LARGE \@title\par}
    \vspace{0.25em}
    {\large \@date\par}
  \end{center}
  \vspace{1em}
}
\makeatother

\maketitle

% ========== SECTION 1: FACULTY INFORMATION ==========
\section*{Faculty Information}
% Replace placeholders with your information
\noindent \textbf{Instructor:} \InstructorName \\
\noindent \textbf{Office:} BUILDING AND ROOM \\
\noindent \textbf{Phone:} (657) 278-XXXX \\
\noindent \textbf{Email:} youremail@fullerton.edu \\
\noindent \textbf{Office hours:} DAYS AND TIMES, by appointment

% ========== SECTION 2: COURSE COMMUNICATION ==========
\section*{Course Communication}
% Describe how students should contact you and your response time
YOUR PREFERRED COMMUNICATION METHOD AND POLICY HERE

% ========== SECTION 3: TECHNICAL PROBLEMS ==========
\section*{Technical Problems}
If you encounter any technical difficulties, contact the instructor immediately to document the problem. Then, contact: \href{http://www.fullerton.edu/it/students/helpdesk/index.php}{student IT help desk}, \href{mailto:StudentITHelpDesk@fullerton.edu}{email}, phone (657) 278-8888, walk-in \href{http://www.fullerton.edu/it/students/sgc/index.php}{student genius center}, online chat - log into \href{http://my.fullerton.edu}{portal}; click ``Online IT Help''; click ``Live Chat.''

\vspace{0.5em}
\noindent \textbf{\underline{For issues with Canvas}}: Canvas Support Hotline = (657) 278-8888, \href{https://canvashelp.fullerton.edu/}{search the CSUF Canvas Guides with AI Assistant}, or \href{https://titans.service-now.com/sp?id=sc_cat_item&sys_id=f88efe80ebea6a10fb7cfcffcad0cdc6&subject=Canvas}{report a problem.}

\vspace{0.5em}
\noindent \textbf{Alternative plan for submitting work:} DESCRIBE YOUR BACKUP PLAN

\vspace{0.5em}
\noindent \textbf{Response time:} YOUR RESPONSE TIME POLICY

% ========== SECTION 4: COURSE INFORMATION ==========
\section*{Course Information}
% Fill in all course details
\noindent \textbf{Prefix, number, title:} \CourseCode, \textit{\CourseTitle} \\
\noindent \textbf{Meeting times with modality, day(s), time(s), and location (if synchronous):} 
% Choose ONE modality from UPS 411.104:
% - In-Person (0-20% online)
% - Hybrid (mostly in-person, 21%-49% online)
% - Hybrid (mostly online, 50%-99% online)  
% - Fully online (100% online)
YOUR MODALITY, DAYS, TIMES, LOCATION

\vspace{0.5em}
\begin{flatlist}
\item \textbf{Zoom:} YOUR ZOOM LINK IF APPLICABLE
\item \textbf{Course requisite(s):} LIST PREREQUISITES/COREQUISITES OR ``none''
\item \textbf{Catalog description:} COPY CATALOG DESCRIPTION VERBATIM (40 words max)
\item \textbf{Additional description:} OPTIONAL EXPANDED DESCRIPTION
\item \textbf{Course materials and equipment:} ~
\item \textbf{Required text(s):} LIST REQUIRED TEXTS OR ``none''
\item \textbf{Recommended text(s):} LIST RECOMMENDED TEXTS OR DELETE THIS LINE
\item \textbf{Other course materials and equipment:} LIST OTHER MATERIALS OR ``none''
\item \textbf{Zero cost:} IF APPLICABLE, NOTE THAT THIS IS A ZERO-COST COURSE
\end{flatlist}

\vspace{1em}
\noindent \textbf{Student Learning Outcomes:}
% List your course learning outcomes
\begin{enumerate}
\item LEARNING OUTCOME 1
\item LEARNING OUTCOME 2
\item LEARNING OUTCOME 3
\item ADD MORE AS NEEDED
\end{enumerate}

% ========== SECTION 5: GRADING POLICIES AND STANDARDS ==========
\section*{Grading Policies and Standards}

% Part a: Grading Scale
\noindent \textbf{a. Grading scale:}

% Example grading scale - modify as needed
\begin{center}
\begin{table}[h]
  \caption{Grade scale}
  \centering
  \begin{tblr}{
    colspec = {l c l c},
    rowhead = 1,                 % ← marks first row as table header
    row{1} = {font=\bfseries},   % bold header text
  }
  Grade & Percent    & Grade & Percent \\
  A+    & 98.0--100.0& C+    & 77.0--79.9 \\
  A     & 93.0--97.9 & C     & 73.0--76.9 \\
  A-    & 90.0--92.9 & C-    & 70.0--72.9 \\
  B+    & 87.0--89.9 & D+    & 67.0--69.9 \\
  B     & 83.0--86.9 & D     & 63.0--66.9 \\
  B-    & 80.0--82.9 & D-    & 60.0--62.9 \\
        &            & F     & 0.0--59.9 \\
  \end{tblr}
\end{table}

\end{center}

% Part b: Required Course Assignments
\vspace{1em}
\noindent \textbf{b. Required Course Assignments:}

% Option 1: Use a table
\begin{center}
\begin{table}[h]
  \caption{Assignment weighting}
  \centering
  \begin{tblr}{
    colspec = {l c},
    rowhead = 1,                 % marks first row as table header
    row{1} = {font=\bfseries},   % bold header text
  }
  Assignment       & Percentage \\
  ASSIGNMENT TYPE 1 & XX\%       \\
  ASSIGNMENT TYPE 2 & XX\%       \\
  ASSIGNMENT TYPE 3 & XX\%       \\
  ASSIGNMENT TYPE 4 & XX\%       \\
  Total            & 100\%      \\
  \end{tblr}
\end{table}
\end{center}

% Option 2: Or list assignments individually (uncomment if preferred)
% \noindent \textbf{Assignment 1:} Description, XX\% of grade, due date \\
% \noindent \textbf{Assignment 2:} Description, XX\% of grade, due date \\
% \noindent \textbf{Assignment 3:} Description, XX\% of grade, due date

% Part c: Attendance and Participation
\vspace{1em}
\noindent \textbf{c. Attendance and Participation policy:}
YOUR ATTENDANCE AND PARTICIPATION EXPECTATIONS

% Part d: Examination dates
\vspace{1em}
\noindent \textbf{d. Examination dates:}
% Remember: No major exams in week 15 unless there's also a final
% Finals must be in week 16 only
LIST EXAM DATES OR "No exams in this course"

% Part e: Make-up and late submission policy
\vspace{1em}
\noindent \textbf{e. Make-up and late submission policy:}
YOUR POLICY FOR LATE WORK AND MAKE-UP EXAMS

% Part f: Authentication of student work
\vspace{1em}
\noindent \textbf{f. Authentication of student work:}
YOUR POLICY FOR VERIFYING STUDENT WORK

% Part g: Extra credit
\vspace{1em}
\noindent \textbf{g. Extra credit:}
YOUR EXTRA CREDIT POLICY OR "No extra credit offered"

% Part h: Retention of student work
\vspace{1em}
\noindent \textbf{h. Retention of student work:}
YOUR POLICY FOR KEEPING STUDENT WORK

% ========== GRADUATE CREDIT (DELETE IF NOT APPLICABLE) ==========
% \vspace{1em}
% \noindent \textbf{Additional assignments for graduate students:}
% If this is a 400-level course available for graduate credit,
% describe the additional assignment(s) required for graduate students

% ========== SECTION 7: ACADEMIC INTEGRITY ==========
\section*{Academic Integrity}
% Describe your expectations for academic honesty
YOUR ACADEMIC INTEGRITY POLICY AND CONSEQUENCES FOR VIOLATIONS

% ========== SECTION 8: TECHNICAL COMPETENCIES (DELETE IF NOT APPLICABLE) ==========
\section*{Technical Competencies}
% List any technical skills beyond those expected of all students
YOUR SPECIAL TECHNICAL REQUIREMENTS BEYOND THOSE ASSUMED OF ALL STUDENTS

% ========== SECTION 8A: POLICY ON THE USE OF GENERATIVE AI AND OTHER TECHNOLOGY ==========
\section*{Policy on the Use of Generative AI and Other Technology}
% Describe your policy on the use of AI tools like ChatGPT, etc.
INCLUDE COURSE POLICY AND EXPECTATIONS ABOUT STUDENTS' USE OF GENERATIVE ARTIFICIAL INTELLIGENCE TOOLS AND OTHER EMERGENT TECHNOLOGIES

% ========== SECTION 9: STUDENT RESOURCES WEBSITE ==========
\section*{Student Resources Website}
It is the student's responsibility to read and understand the required and important \href{https://fdc.fullerton.edu/teaching/student-info-syllabi.html}{student information for course syllabi}. Included is information about:

\begin{itemize}
\item University learning goals
\item General Education learning objectives
\item Netiquette/appropriate online behavior
\item Students' rights to accommodations
\item Campus student support resources
\item Academic integrity
\item Emergency preparedness/what to do
\item Library services
\item Student IT services and competencies
\item Software privacy and accessibility
\item Accessibility statement
\item Diversity statement
\item Land acknowledgement
\item Final exam schedule
\item Semester calendar
\end{itemize}

% ========== SECTION 10: CLASSROOM MANAGEMENT ==========
\section*{Classroom Management}
% Add any additional classroom policies or "rules"
YOUR CLASSROOM POLICIES (cell phones, recording, etc.)

% ========== SECTION 11: GE REQUIREMENTS (DELETE IF NOT A GE COURSE) ==========
% \section*{General Education Requirements}
% \noindent \textbf{GE requirement(s) that this course meets:} GE CATEGORY
% 
% \noindent \textbf{How the GE writing requirement will be met and assessed:}
% DESCRIBE HOW WRITING WILL BE INCORPORATED AND ASSESSED
% 
% \noindent \textbf{GE grading standard:}
% % For Golden Four (A.1, A.2, A.3, B.4):
% A grade of ``C-'' (1.7) or higher is required to meet this General Education requirement. A grade of ``D+'' (1.3) or below will not satisfy this General Education requirement.
% 
% % For all other GE courses:
% % A grade of ``D'' (1.0) or higher is required to meet this General Education requirement. A grade of ``D-'' (0.7) or below will not satisfy this General Education requirement.

% ========== SECTION 12: UPPER-DIVISION WRITING (DELETE IF NOT APPLICABLE) ==========
% \section*{Upper-Division Writing Course Requirements}
% DESCRIBE HOW THIS COURSE MEETS UPPER-DIVISION WRITING REQUIREMENTS

% ========== SECTION 13: CALENDAR/SCHEDULE ==========
\section*{Calendar of Topics / Schedule of Classes}
% Remember: 16 weeks total (15 instruction + 1 finals)
% Don't number spring/fall break week

% Option 1: Weekly list format
\noindent \textbf{Week 1, MM/DD}\\
Topic(s): \\
Reading(s): \\
Assignment(s) Due: \\

\noindent \textbf{Week 2, MM/DD}\\
Topic(s): \\
Reading(s): \\
Assignment(s) Due: \\

% Continue for all 16 weeks...

% Option 2: Table format (uncomment if preferred)
% \begin{center}
% \begin{tabular}{|c|c|l|l|l|}
% \hline
% \textbf{Week} & \textbf{Date} & \textbf{Topic} & \textbf{Readings} & \textbf{Assignments Due} \\
% \hline
% 1 & MM/DD & & & \\
% \hline
% 2 & MM/DD & & & \\
% \hline
% % ... continue for all weeks
% \hline
% 16 & MM/DD & Final Exam & & \\
% \hline
% \end{tabular}
% \end{center}

\end{document}
