\documentclass[12pt, letterpaper]{article}
\usepackage[english]{babel}
\usepackage[T1]{fontenc}
\usepackage[margin=1.15in]{geometry}
\usepackage{xcolor}
\usepackage{url}
\usepackage[utf8]{inputenc}
\usepackage{tabularx}
\usepackage{booktabs}
\frenchspacing
\usepackage{multicol}
\usepackage{eso-pic}
\usepackage[longnamesfirst]{natbib}
\bibpunct{(}{)}{;}{a}{}{,}
\usepackage{caption}
\usepackage{subcaption}
\usepackage{setspace}
\usepackage{paralist}
\usepackage{quoting}
\usepackage{comment}
\usepackage{enumitem}
\usepackage{graphicx}
\usepackage{float}
\usepackage{bookmark}
\renewcommand{\thesection}{\arabic{section}.}
\renewcommand{\thesubsection}{\thesection\arabic{subsection}}
\renewcommand{\thesubsubsection}{\thesubsection.\arabic{subsubsection}}
\usepackage{hyperref}
\hypersetup{
    colorlinks=true,
        linkcolor=blue,
        filecolor=magenta,      
        urlcolor=cyan,
        citecolor=purple,
}
\usepackage[scaled]{helvet} % Load the helvet package
\renewcommand*\familydefault{\sfdefault} % Set the default font to be sans-serif
\newenvironment{boldnumlist}
{\begin{enumerate}[label=\textbf{\arabic*.}]}
{\end{enumerate}}

\begin{document}
\title{\includegraphics[width=8cm]{Images/stacked.png} \\ \textbf{Introduction to Public Administration}}

\title{MPA Capstone Seminar: \\ Public Administration Theory}
\author{POSC 521 — Fall 2024}
\date{Thursdays at 7:00 in GH 248}
    \maketitle
    \subsection*{Session Schedule:}
    
    \begin{multicols}{2}
    \subsubsection*{In-Person Sessions}
    \begin{itemize}[leftmargin=*]
      \item August 29
      \item September 5, 12, 19, 26
      \item October 3, 10, 17
      \item November 14
      \item December 12
    \end{itemize}
    
    \columnbreak
    
    \subsubsection*{Asynchronous Online Sessions}
    \begin{itemize}[leftmargin=*]
      \item October 24, 31
      \item November 7, 28
      \item December 5, 19
    \end{itemize}
    \end{multicols}


    \subsection*{Professor: David P. Adams, Ph.D.}

    \subsubsection*{Contact Information:}
    
    \begin{itemize}
        \item Office: 516 Gordon Hall
        \item Phone/SMS: (657) 278-4770
        \item Zoom Meeting ID: 334 750 2369 or \href{https://fullerton.zoom.us/j/3347502369}{\texttt{fullerton.zoom.us/j/3347502369}} 
        \item website: \href{https://dadams.io}{\texttt{dadams.io}}
        \item email: \href{dpadams@fullerton.edu}{\texttt{dpadams@fullerton.edu}}
        \item Office hours: Tuesdays \& Thursdays from 9:30 to 11:00, Thursdays from 5:30 to 6:30, and by \href{https://dadams.io/appointments}{appointment}.
        \item Schedule meetings throughout the week: \href{https://dadams.io/appointments}{\texttt{dadams.io/appointments}}
    \end{itemize}
    
    \section{Catalog Description}
    Concepts, models and ideologies of public administration within the larger political system. Course restricted to students in their final six units of graduate work.
    
    \section{Course Description}
    The capstone seminar in the Master of Public Administration program at Cal State Fullerton examines concepts, models, and ideologies of public administration within the larger political system.

    \section{Course Prerequisites}
    This course is restricted to students in their final six units of graduate work in the MPA program. Students must have completed all other required courses in the MPA program before enrolling in this course.
    
    \section{Course Objectives}
    This course is designed to accomplish five interrelated objectives:
    
    \begin{enumerate}
        \item \textbf{Theory Examination}: We will delve into the most important theories and literature in public administration, fostering a deep understanding of the field.
        \item \textbf{Literature Review}: You will complete a literature review in your concentration area, allowing you to specialize and delve deeper into a specific aspect of public administration. This preparation will be crucial for the general concentration portion of the comprehensive exams.
        \item \textbf{Writing Skills}: This course will enhance your writing skills, focusing on clear, concise, and effective communication. This preparation will be crucial for the general theory portion of the comprehensive exams.
        \item \textbf{Critical Thinking}: You will develop your critical thinking skills, learning to analyze and evaluate complex arguments and theories. This preparation will be crucial for the general theory portion of the comprehensive exams.
        \item \textbf{Professional Development}: This course will help you develop the skills and knowledge necessary for a successful career in public administration. You will learn about the latest trends and issues in the field and how to navigate the challenges of public service.
    
    \end{enumerate}
    
\section{University-wide Student Learning Outcomes}
    As a capstone course in the MPA program, this course is designed to help students achieve the following university-wide student learning outcomes:
    \begin{enumerate}
        \item Knowledge, skills, and professional dispositions including higher order competence in disciplinary perspectives and interdisciplinary points of view;
        \item The ability to access, analyze, synthesize, and evaluate complex information from multiple sources and in new situations and settings;
        \item Advanced communication skills;
        \item The ability to work independently and in collaboration with others as artists, practitioners, researchers, and/or scholars;
        \item The ability to determine and apply appropriate methods and technologies to address problems that affect their communities; 
        \item A commitment to social justice and ethical leadership within diverse communities and an interdependent global community.
    \end{enumerate}

    \section{Course Materials}
    \subsection*{Required Texts}
    \begin{itemize}
        \item \textbf{Denhardt and Denhardt}. \textit{The New Public Service: Serving, Not Steering}. 4th ed. Routledge, 2015.
        \item \textbf{Lipsky, Michael}. \textit{Street-Level Bureaucracy: Dilemmas of the Individual in Public Services}. Russell Sage Foundation, 2010.
        \item \textbf{Stivers, Camilla}. \textit{Bureau Men, Settlement Women: Constructing Public Administration in the Progressive Era}. University of Kansas Press, 2000.
    \end{itemize}

\subsection*{Additional Readings}

\noindent Additional readings are indicated on the course schedule and in the references section of this document. These readings and a \texttt{.bib} file are located in a shared DropBox folder. Please contact the professor if you need access to the folder.

\section{Technical Requirements}

\subsection*{Pollak Library Resources}

The Pollak Library provides a wide range of resources and services to support your research and learning. These resources include books, journals, databases, and research guides. You can access the library's resources online through the \href{http://www.library.fullerton.edu/}{Pollak Library website}. The library also offers research assistance through the \href{http://www.library.fullerton.edu/research/}{Research Assistance Program}. You can also access the \href{http://www.library.fullerton.edu/about/guidelines/online-instruction-guidelines.php}{library's online instruction guidelines} for help with online learning.

\subsection*{Canvas}

This course will use \href{https://csufullerton.instructure.com/}{Canvas} as a learning management system. You will use \emph{Canvas} to access course materials, submit assignments, participate in discussions, and communicate with the professor and your classmate. You are responsible for checking \emph{Canvas} regularly for announcements, assignments, and other course materials. You are also responsible for ensuring that your \emph{Canvas} notifications are set to receive messages from the course. 

\subsection*{Zoom}
This course may include synchronous online sessions using \href{https://fullerton.zoom.us/}{Zoom}. You are responsible for ensuring that you have the necessary equipment and internet connection to participate in these sessions. 

\subsection*{Minimum Technical Requirements}

To participate in this course, you will need the following minimum technical requirements:
\begin{itemize}
    \item A computer or tablet with a reliable internet connection
    \item A webcam and microphone
    \item A modern web browser (Chrome, Firefox, Safari, or Edge)
    \item Microsoft Word or a compatible word processing program
    \item Adobe Acrobat Reader or a compatible PDF reader
\end{itemize}


\noindent Long- and short-term computer and internet access loans are available through the \href{http://www.fullerton.edu/it/students/sgc/index.php}{Student Genius Center}.

\section{Technical Problems}

\subsection*{University IT Help Desk}

Contact the instructor immediately to document the problem if you encounter any technical difficulties. Then contact the \href{http://www.fullerton.edu/it/students/helpdesk/index.php}{Student IT Help Desk} for assistance. You can also call the Student IT Help Desk at (657) 278-8888, \href{mailto:StudentITHelpDesk@fullerton.edu}{email}, visit them at the Pollak Library North \href{http://www.fullerton.edu/it/students/sgc/index.php}{Student Genius Center}, or log on to the \href{http://my.fullerton.edu/}{my.fullerton.edu} portal and click ``Online IT Help'' followed by ``Live Chat''.

\subsection*{Canvas Support}

If you encounter any technical difficulties with Canvas, call the Canvas Support Hotline at 855-302-7528, visit the \href{https://community.canvaslms.com/docs/DOC-10720-67952720329}{Canvas Community}, or click the ``Help'' button in the lower left corner of Canvas and select ``Report a Problem''. The \href{https://cases.canvaslms.com/liveagentchat?chattype=student&sfid=001A000000YzcwQIAR}{Student Support Live Chat} is available 24 hours a day, 7 days a week.


\section{University Student Policies}

In accordance with UPS 300.00, students must be familiar with certain policies applicable to all courses. Please review these policies as needed and visit this Cal State Fullerton website \texttt{\href{https://fdc.fullerton.edu/teaching/student-info-syllabi.html}{https://fdc.fullerton.edu/teaching/student-info-syllabi.html}} for links to the following information:

\begin{enumerate}
    \item   University learning goals and program learning outcomes.
    \item	Learning objectives for each General Education (GE) category.
    \item	Guidelines for appropriate online behavior (netiquette).
    \item	Students' rights to accommodations for documented special needs.
    \item   Campus student support measures, including Counseling \& Psychological Services, Title IV and Gender Equity, Diversity Initiatives and Resource Centers, and Basic Needs Services.
    \item	Academic integrity (refer to UPS 300.021).
    \item	Actions to take during an emergency.
    \item	Library services information.
    \item	Student Information Technology Services, including details on technical competencies and resources required for all students.
    \item	Software privacy and accessibility statements.
\end{enumerate}

\section{Course Student Policies}

\subsection*{Course Communication}
All course announcements and communications will be sent via \emph{Canvas} and university email. Students are responsible for regularly checking their \emph{Canvas} notifications and email. Students are also responsible for ensuring that their \emph{Canvas} notifications are set to receive messages from the course. Students are expected to check \emph{Canvas} and their email at least once daily.

\subsubsection*{Response Time}I will strive to respond to all student emails and \emph{Canvas} messages within 24 hours, except on weekends and holidays. If you do not receive a response within 24 hours, please send a follow-up message. If you do not receive a response within 48 hours, please send another follow-up message and contact me via phone or SMS text at (657) 278-4770.

\subsection*{Due Dates}
All assignments are due on the date specified in the course schedule. Late assignments will only be accepted if prior arrangements have been made with the professor. Students must submit all assignments on time and in the correct format. Failure to submit an assignment on time may result in a grade penalty.

\subsection*{Alternative Procedures for Submitting Work}
Students are expected to submit all assignments via \emph{Canvas}. If you cannot submit an assignment via \emph{Canvas}, please contact the professor to discuss alternative submission procedures.

\subsection*{Extra Credit}
Extra credit opportunities will not be offered in this course. All students will be graded based on the same criteria and standards.

\subsection*{Attendance}
Students are expected to attend all in-person sessions. If you are unable to attend a session, please notify the professor in advance. If you miss a session, you are responsible for obtaining the information and materials covered in the session.

\subsection*{Retention of Student Work}
Students are responsible for retaining copies of all assignments submitted in this course. Students should keep copies of all assignments until the end of the semester and verify that their assignments have been graded and returned before discarding them.

\subsection*{Academic Integrity}
Students are expected to adhere to the highest standards of academic integrity. Any student found to have engaged in academic dishonesty will be subject to the sanctions described in the \href{https://www.fullerton.edu/senate/publications_policies_resolutions/ups/UPS%20300/UPS%20300.021.pdf}{Academic Dishonesty Policy} (UPS 300.021). Academic dishonesty includes, but is not limited to, cheating, plagiarism, fabrication, facilitating academic dishonesty, and submitting previously graded work without prior authorization. Students are expected to be familiar with the university's policy on academic dishonesty and to adhere to this policy in all aspects of this course. Any student who has questions about the policy should ask the professor for clarification.

\subsection*{Plagiarism}
Plagiarism is a serious violation of academic integrity and will not be tolerated in this course. Plagiarism includes, but is not limited to, copying and pasting text from sources without proper citation, paraphrasing text from sources without proper citation, and submitting work that is not your own. Students are expected to properly cite all sources used in their work and to submit original work. Failure to do so may result in a failing grade for the assignment and further disciplinary action.

\subsection*{Written Work}
All written work must be submitted in a professional format, including proper grammar, spelling, and punctuation. Written work must also be properly cited using the appropriate citation style. Students are expected to follow the guidelines for written work provided by the professor and to seek clarification if they have questions about the requirements.

\subsection*{Artificial Intelligence Policy}
\subsubsection*{Definitions of Generative AI}

\noindent For the purposes of this course, generative AI refers to artificial intelligence systems capable of producing human-like text, images, or other content. This includes, but is not limited to:

\begin{itemize}
    \item Large language models (e.g., GPT-3, GPT-4)
    \item Text-to-image generators (e.g., DALL-E, Midjourney)
    \item AI-powered writing assistants (e.g., ChatGPT, Claude)
    \item Automated content generators
\end{itemize}

\subsubsection*{AI Use Policy}

In this course, the use of AI is permitted and encouraged as a learning tool, with the following guidelines:

\begin{itemize}
    \item \textbf{Allowed}: Using AI for feedback on annotated bibliographies and syntheses, as outlined in the assignment.
    \item \textbf{Allowed}: Using AI for the optional reflection activity.
    \item \textbf{Not Allowed}: Submitting AI-generated content as your own work without substantial human input and critical analysis.
    \item \textbf{Not Mandatory}: The use of AI is optional. Students are not required to use AI tools if they prefer not to.
\end{itemize}

\subsubsection*{Rationale for AI Policy}

This AI policy is designed to align with our course objectives:

\begin{enumerate}
    \item It supports the examination of public administration theories by encouraging critical engagement with AI feedback.
    \item It enhances literature review skills by allowing AI to provide additional perspectives on annotations and syntheses.
    \item It improves writing skills by providing an additional source of feedback and encouraging revision.
    \item It fosters critical thinking by requiring students to evaluate and incorporate AI feedback thoughtfully.
    \item It contributes to professional development by familiarizing students with AI tools they may encounter in their public administration careers.
\end{enumerate}

\subsubsection*{Guidance for Students on AI Use}

\begin{itemize}
    \item When using AI for feedback, always cite it as a source. For example: ``ChatGPT (Version X.X) [AI model]. OpenAI. Feedback received on [Date].''
    \item Be critical of AI-generated content. AI can make mistakes or provide biased information.
    \item Use AI as a tool to enhance your learning, not as a substitute for your own critical thinking and analysis.
    \item We will discuss ethical AI use and its implications for public administration during our course.
\end{itemize}

\subsubsection*{Repercussions for Policy Breaches}

\begin{itemize}
    \item Suspected misuse of AI (e.g., submitting AI-generated content as your own without proper attribution) will be addressed on a case-by-case basis.
    \item Detection methods may include inconsistencies in writing style, unusual phrasing, or content that doesn't align with the student's demonstrated knowledge and skills.
    \item Consequences for policy breaches may range from a required revision of the assignment to more severe academic integrity penalties, such as dismissal from the MPA program, depending on the nature and extent of the misuse.
\end{itemize}

\noindent Remember, the goal of this policy is to enhance your learning experience and prepare you for a future where AI will likely play a significant role in public administration. Use these tools responsibly and ethically to support your academic and professional growth.

\subsubsection*{Goals and Assessment of AI Use}

\noindent The goals for AI use in this course are:

\begin{enumerate}
    \item To teach students to use AI as a complementary tool for enhancing critical thinking and writing skills.
    \item To prepare students for professional environments where AI tools are increasingly common in public administration.
    \item To encourage critical evaluation of AI-generated feedback and content.
    \item To foster ethical and responsible use of AI in academic and professional settings.
\end{enumerate}

\noindent Assessment of these goals and overall learning objectives will be conducted through:

\begin{itemize}
    \item Comparison of initial drafts with AI-assisted revisions
    \item Analysis of student reflections on the learning process
    \item Evaluation of class participation and discussions
    \item Periodic comparative assignments (with and without AI assistance)
    \item Review of AI usage logs
    \item Assessment of peer review quality
    \item Comprehensive final project or exam
    \item Professional development reflection on future AI use in public administration
\end{itemize}

\noindent These assessment methods will help ensure that AI use is enhancing, rather than replacing, critical thinking and analytical skills essential to public administration.

 

\section*{Kritik: Peer Review Platform}

\noindent This course will use the peer review platform \href{https://kritik.io/}{Kritik} for peer review assignments. Kritik is an online platform that allows students to provide feedback on their classmates' work. Students will be assigned to review the work of their peers and provide constructive feedback. Students will also receive feedback from their peers on their own work. The professor will use Kritik to monitor the peer review process and provide guidance as needed.
\vspace{0.5em}
\noindent The Kritik platform will be used for the weekly syntheses and the concentration area literature review assignments. Students are expected to familiarize themselves with the platform and use it to complete the peer review assignments. The professor will provide guidance and support as needed to ensure that students can use Kritik effectively.

\subsection*{Kritik Overview}

\begin{itemize}

\item \textbf{A Three-Stage Learning Process:}

\begin{enumerate}
    \item Craft Your Analysis: Follow the provided rubric and delve into a public policy challenge. This could involve, for example, evaluating the ethical implications of a proposed environmental regulation or assessing the effectiveness of a social welfare program.
    \item Provide Constructive Critique: Anonymously evaluate your peers' work using the rubric. Offer actionable feedback that focuses on the strengths and weaknesses of their analysis, drawing connections to relevant public administration concepts.
    \item Reflect and Improve: Receive anonymous feedback on the quality and impact of your comments. Learn to deliver clear, concise, and impactful feedback—a crucial skill for public servants collaborating on complex issues.
\end{enumerate}

\item \textbf{Grading and Participation:} You'll earn four scores for each Kritik activity: Creation, Evaluation, Feedback, and Overall. These scores, along with active participation, will contribute to your course grade. Participating thoughtfully in Kritik activities will not only improve your own skills but also enrich the learning experience for your peers.

\item \textbf{Registration and Support:} We'll thoroughly introduce Kritik in class, and a dedicated email invitation will guide you through registration and course enrollment. The Kritik Help Center offers additional resources, and I'm always available to address any questions or concerns.
\end{itemize}
\section{Course Requirements}
\subsection*{1) Weekly Annotated Bibliography, Synthesis, and Reflection}

To facilitate class discussions and prepare you for comprehensive exams, your weekly writing assignment comprises two main components: an annotated bibliography of each weekly reading and a 3- to 5-page synthesis of the week's readings. You may also submit a personal reflection for bonus points. 

\subsubsection*{Definitions}
\begin{itemize}
\item \textbf{Annotated Bibliography}: An annotated bibliography is an organized list of sources, similar to a reference list. Each source is followed by a brief (approximately 150 words) descriptive and evaluative paragraph—the annotation. For this course:

\begin{itemize}
    \item \textbf{Citation}: Provide a full citation in APA or Chicago author-date style.
    \item \textbf{Summary}: Summarize the central theme and key points in your own words.
    \item \textbf{Relevance}: Note the source's relevance to the week's topic and its implications for public administration.
\end{itemize}

\noindent The purpose is to review the literature on the week's topic, aiding in class discussions and developing your ability to distill and synthesize complex information.

\item \textbf{Synthesis}: A synthesis combines information from multiple sources to present a comprehensive understanding or argument. In this context, your sources are the weekly readings. The synthesis should:

\begin{itemize}
    \item Identify patterns and themes across readings;
    \item Draw connections between sources; and
    \item Address contradictions among sources or within individual readings.
\end{itemize}

\noindent This critical analysis encourages deeper thinking about the readings and their implications for public administration theory and practice.

\item \textbf{Reflection} (Optional Bonus Activity): Reflection is a metacognitive practice that allows you to analyze your learning process, feedback received, and the evolution of your understanding of the week's topic. This activity is optional and offers bonus points. It involves self-assessment and promotes continuous improvement.

\end{itemize}
\subsubsection*{Assignment Details}

\textbf{Objectives}: Engage deeply with course readings, reflect on their implications for public administration, and collaborate with peers and AI to enhance analytical and writing skills.
\vspace{0.5em}

\noindent \textbf{Incorporating AI}: We'll use AI tools like ChatGPT to provide feedback on your assignments, aiming to refine your comprehension and writing skills. Remember, AI is a complementary tool to your critical thinking.

\subsubsection*{Assignment Components}

\begin{boldnumlist}
\item \textbf{Annotated Bibliography}

\begin{itemize}
    \item \textbf{Purpose}: Review literature on the week's topic to prepare for synthesis and class discussions.
    \item \textbf{Process}:
    \begin{itemize}
        \item Write an annotation (150 words) for each weekly reading, including:
        \begin{itemize}
            \item Full citation in APA or Chicago author-date style
            \item Summary of central themes and key points.
            \item Relevance to the week's topic.
        \end{itemize}
        \item \textbf{Optional}: Submit your annotations to AI for feedback using the prompt:
        \textit{"I've completed an annotated bibliography for an article. The citation is [insert citation]. The summary I wrote is [insert summary]. I've noted the relevance as [insert relevance]. Could you provide feedback on my summary and relevance note?"}
        \item Revise based on AI feedback as desired. This activity is ungraded and for your benefit.
    \end{itemize}
\end{itemize}

\item \textbf{Synthesis}

\begin{itemize}
    \item \textbf{Purpose}: Integrate main themes and insights from weekly readings, discussing their implications for public administration. \textbf{This portion of the assignment conducted on the Kritik platform. All papers should be submitted anonymously to ensure unbiased feedback.}
   
    \item \textbf{Process}:
    \begin{itemize}
        \item Write a 3- to 5-page synthesis, focusing on:
        \begin{itemize}
            \item Critical analysis rather than mere summaries.
            \item Identifying patterns, connections, and contradictions.
        \end{itemize}
        \item Submit your draft to Canvas for peer review. Each student will be assigned two peers.
        \item Review your peers' syntheses using the Canvas rubric. Provide kind, constructive, and specific feedback. The rubric is a guide; feel free to add additional comments.
        \item Revise your synthesis based on peer feedback.
        \item Submit the revised synthesis to AI for feedback using the prompt:
        \textit{"I've synthesized information from [number] articles on [topic]. Here's my draft of the synthesis [insert draft]. Could you provide feedback and suggest any connections or contrasts I missed?"}
        \item \emph{Revising a second time based on AI feedback is optional and ungraded.}
        \item Submit your final synthesis \emph{and} the AI's feedback to Canvas for grading.
    \end{itemize}
    \item \textbf{Grading}:
    \begin{itemize}
        \item \textbf{Completeness}: Addresses all required components.
        \item \textbf{Critical Analysis}: Offers insightful and thoughtful analysis.
        \item \textbf{Clarity}: Communicates ideas clearly and effectively.
        \item \textbf{Citation}: Properly cites sources in APA or Chicago author-date style.
        \item \textbf{Peer Review}: Provides constructive feedback to peers.
        \item \textbf{Revision}: Incorporates peer and (optional) AI feedback effectively.
    \end{itemize}
    The synthesis, peer review, and AI feedback will be graded as a single assignment. Your revised synthesis will be evaluated using the Canvas rubric.
\end{itemize}

\item \textbf{Bonus Point Reflection Activity} (Optional)

\begin{itemize}
    \item \textbf{Purpose}: Engage in metacognitive practice to analyze your learning process and feedback.
    \item \textbf{Process}:
    \begin{itemize}
        \item After completing your revised synthesis and reviewing AI feedback, reflect on the entire process, including annotated bibliographies and peer reviews.
        \item Consider the following questions (optional):
        \begin{itemize}
            \item How did your understanding of the readings evolve?
            \item Which feedback (from peers or AI) was most beneficial, and why?
            \item How did the peer review process impact your perspective or approach?
            \item What strategies were most effective in distilling and presenting information?
            \item How can you apply lessons from this assignment to future tasks in public administration or other professional settings?
        \end{itemize}
        \item Write a brief reflection encapsulating your thoughts.
        \item Submit your reflection to AI for feedback using the prompt:
        \textit{"I've completed a reflection on my learning process for this week's synthesis assignment. Here's my reflection [insert reflection]. Could you provide feedback or ask questions to provoke further thought?"}
        \item Submit both your reflection and the AI's feedback to Canvas for bonus point consideration.
    \end{itemize}
\end{itemize}
\end{boldnumlist}
\subsubsection*{Skip Week!}

Students may skip one week of the annotated bibliography, synthesis, and reflection assignment during the semester. This option is intended to accommodate unexpected events or personal commitments. If you choose to skip a week, please notify the professor in advance. In place of the final synthesis during the skip week place a note in the Canvas assignment indicating that you are using your skip week.

\subsection*{2) Reading Discussion Facilitation}
A group of students will be responsible for facilitating the discussion of the assigned readings each week. The facilitators will lead the class in a discussion of the readings, pose questions to stimulate critical thinking and analysis, and encourage active participation from all students. The facilitators will also provide a brief summary of the readings and highlight the key points for discussion. The facilitators will be selected at the beginning of the semester, and the schedule will be posted on \emph{Canvas}. The facilitators will be graded based on the following criteria:

\begin{itemize}
    \item Preparation: The facilitators are well-prepared and have a thorough understanding of the readings.
    \item Engagement: The facilitators actively engage the class in discussion and encourage participation from all students.
    \item Critical Thinking: The facilitators pose thought-provoking questions and stimulate critical thinking and analysis.
    \item Communication: The facilitators communicate clearly and effectively, ensuring that all students understand the key points of the readings.
    \item Leadership: The facilitators lead the discussion in a professional and respectful manner, creating a positive learning environment.
    \item Participation: The facilitators actively participate in the discussion and contribute to the class's understanding of the readings.
\end{itemize}

\subsection*{3) MPA Comprehensive General Area Essay Exam}

Students will complete a comprehensive general area essay exam as part of the MPA program's comprehensive exam requirement. The exam will consist of two questions from which students will choose one to answer. The questions will be based on the course readings and discussions and will require students to demonstrate their understanding of public administration's key concepts, theories, and debates. The exam will allow students to synthesize their learning in the course and demonstrate their ability to think critically and write clearly about complex issues in public administration. Students who do not pass the exam on the first attempt will have the opportunity to retake the exam once during finals week. The grade for the exam is pass/fail. Students who do not pass on the second attempt will be required to retake the course.

\subsection*{4) Concentration Area Literature Review}

\subsubsection*{Literature Review}

Students will complete a literature review in their concentration area as part of the MPA program's comprehensive exam requirement. The literature review will examine the practical and theoretical issues related to a specific public administration topic and synthesize the essential findings and debates in the literature. The literature review will allow students to deepen their knowledge of their concentration area and develop their research and writing skills. The length of the literature review will be approximately 15 pages, double-spaced, in APA or Chicago author-date style.

A peer review process will be used to provide feedback on each component of the literature review, including the topic selection, annotated bibliography, draft, and final draft. The peer review process will help students refine their work and improve their writing and analytical skills. The final draft of the literature review will be graded based on the following criteria. All submissions should be anonymous to ensure unbiased feedback.


\paragraph*{Objectives:}
\begin{itemize}
    \item Deepen knowledge in a specific concentration area.
    \item Develop research and writing skills.
    \item Synthesize practical and theoretical issues in public administration.
\end{itemize}

\paragraph*{Requirements:}
\begin{enumerate}
    \item \textbf{Topic Selection:}
    \begin{itemize}
        \item Choose a topic within your concentration area in consultation with the professor.
        \item Ensure the topic addresses both practical and theoretical dimensions.
    \end{itemize}
    \item \textbf{Theoretical Book Requirement:}
    \begin{itemize}
        \item Select a theoretical book related to your concentration area as part of your literature review.
        \item Include an analysis of the book's contribution to the field and how it relates to your topic.
    \end{itemize}
    \item \textbf{Literature Sources:}
    \begin{itemize}
        \item Utilize a minimum of 15 peer-reviewed articles or books.
        \item Incorporate recent studies to ensure up-to-date analysis.
    \end{itemize}
    \item \textbf{Structure:}
    \begin{itemize}
        \item Introduction: State the research question and its significance.
        \item Literature Review: Summarize and synthesize key findings and debates.
        \item Analysis: Critically evaluate the literature, identifying gaps and future research directions.
        \item Conclusion: Summarize the main findings and their implications for public administration.
    \end{itemize}
    \item \textbf{Peer Review:}
    \begin{itemize}
        \item Submit a draft for peer review on Canvas.
        \item Provide constructive feedback to at least two classmates.
        \item Revise your literature review based on the feedback received.
    \end{itemize}
    \item \textbf{Final Draft Deliverable:}
    \begin{itemize}
        \item Submit a final draft of your literature review on Canvas.
        \item Ensure the final draft meets the requirements outlined in the assignment guidelines.
    \end{itemize}
\end{enumerate}

\paragraph*{Submission:}
\begin{itemize}
    \item Topic Outline: Week 12
    \item Annotated Bibliography: Week 13
    \item First Draft: Week 14
    \item Peer Review: Week 15
    \item Final Draft: Week 16
\end{itemize}

\section{Course Requirements Due Dates}

The due dates for the course requirements are as follows:
    \begin{itemize}
        \item Annotated Bibliography and Synthesis: 
        \begin{itemize}
            \item Annotated Bibliographies and Syntheses: Due each week by 11:59 p.m. on Wednesday
            \item Peer Review: Due each week by 11:59 p.m. on Friday
            \item Final Synthesis: Due each week by 11:59 p.m. on Saturday
            \item Personal Reflection: Due each week by 11:59 p.m. on Saturday
        \end{itemize}
        \item MPA Comprehensive General Area Essay Exam:
        \begin{itemize}
            \item Distributed on 10/31
            \item Due on 11/7
        \end{itemize}
        \item Concentration Area Literature Review:
        \begin{itemize}
            \item Topic Selection: Due on 11/14
            \item Annotated Bibliography: Due on 11/21
            \item Draft: Due on 12/5
            \item Peer Review: Due on 12/12
            \item Final Draft: Due on 12/19
        \end{itemize}
    \end{itemize}

\section{Grades}


\subsection*{Grading Scale and Grade Weights}  

The grading scale is shown in Table~\ref{tab:grading-scale}. Grades will be given based on the weights in Table~\ref{tab:grade-weights}.

\begin{table}[h]
\centering
\caption{Grading Scale}
\begin{tabular}{llll}
\toprule
\textbf{Grade} & \textbf{Percentage} & \textbf{Grade} & \textbf{Percentage} \\
\midrule
A+ & 98.0 -- 100 & B- & 80.0 -- 81.9\\
A & 92.0 -- 97.9 & C+ & 78.0 -- 79.9\\
A- & 90.0 -- 91.9 & C & 72.0 -- 77.9\\
B+ & 88.0 -- 89.9 & C- & 70.0 -- 71.9\\
B & 82.0 -- 87.9 & D & 60.0 -- 69.9\\
D- & 50.0 -- 59.9 & F & 0 -- 49.9\\

\bottomrule
\end{tabular}
\label{tab:grading-scale}
\end{table}

\begin{table}[h!]
    \centering
    \caption{Grade Weights}
    \begin{tabular}{ll}
        \toprule
    \textbf{Assignment} & \textbf{Percentage} \\
    \midrule
    Weekly Annotated Bibliographies & 10\% \\
    Weekly Syntheses & 15\% \\
    Reading Discussion Facilitation & 5\% \\
    MPA Comprehensive General Area Essay Exam & 35\% \\
    Concentration Area Literature Review Annotated Bibliography & 5\% \\
    Concentration Area Literature Review Draft & 10\% \\
    Concentration Area Literature Review Peer Review & 5\% \\
    Concentration Area Literature Review Final Draft & 15\% \\
    \bottomrule
    \end{tabular}
    \label{tab:grade-weights}
    \end{table}
    
    \subsubsection*{Bonus Points}
    \begin{itemize}
        \item Bonus points will be awarded for exceptional work, including insightful comments in class discussions and high-quality written assignments. Bonus points will be added to your final grade at the end of the semester.
        \item Bonus points will be awarded for personal reflections on the weekly assignments. The reflections are an opportunity for you to engage in self-assessment and continuous improvement, enhancing your learning and professional development.
    \end{itemize}
    
\section{Course Schedule}

\subsection*{Week 1 -- 8/29: Public Administration Theory I}
\begin{itemize}
    \item \textbf{In-person Session}: Introduction to the Course
    \item Readings:
        \begin{itemize}
            \item \citet{Wilson1887}, ``The Study of Administration''
            \item \citet{Weber1946}, ``Bureaucracy'' 
            \item \citet{gulick1937}, ``Notes on the Theory of Organization'' 
            \item \citet{Follett1926}, ``The Giving of Orders'' 
            \item \citet{simon1946}, ``Proverbs of Administration'' 
        \end{itemize}
    \item Due Wednesday: Annotated Bibliography and Synthesis
    \item Due Friday: Peer Review
    \item Due Saturday: Final Synthesis and Personal Reflection
\end{itemize}


\subsection*{Week 2 -- 9/5: Public Administration Theory II}
\begin{itemize}
    \item \textbf{In-person Session}: Public Administration in the U.S. Context
    \item Readings:
        \begin{itemize}
            \item \citet{Stivers2000}, \emph{Bureau Men, Settlement Women} 
            \item \citet{Denhardt2015}, \emph{The New Public Service}, Chapters 1--4 
        \end{itemize}
        \item Due Wednesday: Annotated Bibliography and Synthesis
        \item Due Friday: Peer Review
        \item Due Saturday: Final Synthesis and Personal Reflection
\end{itemize}
\subsection*{Week 3 -- 9/12: Ethics and Values in Public Administration}
\begin{itemize}
    \item \textbf{In-person Session}: Public Service Values and Ethics
    \item Readings:
        \begin{itemize}
            \item \citet{friedrich1935}, ``Responsible Government Service Under the American Constitution'' 
            \item \citet{FINER1941}, ``Administrative Responsibility in Democratic Government'' 
            \item \citet{goss1996}, ``A Distinct Public Administration Ethics?'' 
            \item \citet{Adams2009}, ``Unmasking Administrative Evil'' 
            \item \citet{Denhardt2015}, \emph{The New Public Service}, Chapter 7 
        \end{itemize}
        \item Due Wednesday: Annotated Bibliography and Synthesis
        \item Due Friday: Peer Review
        \item Due Saturday: Final Synthesis and Personal Reflection
\end{itemize}

\subsection*{Week 4 -- 9/19: Leadership and Motivation}
\begin{itemize}
    \item \textbf{In-person Session}: Leadership and Motivation
    \item Readings:
        \begin{itemize}
            \item \citet{Christensen2017}, ``Public Service Motivation Research'' 
            \item \citet{Denhardt2015}, \emph{The New Public Service}, Chapter 8 
            \item \citet{Lachance2017}, ``Public Service Motivation'' 
            \item \citet{maslow1943}, ``A Theory of Human Motivation'' 
            \item \citet{Fairholm2004}, ``Different Perspectives on the Practice of Leadership'' 
        \end{itemize}
        \item Due Wednesday: Annotated Bibliography and Synthesis
        \item Due Friday: Peer Review
        \item Due Saturday: Final Synthesis and Personal Reflection
\end{itemize}

\subsection*{Week 5 -- 9/26: Performance Management}
\begin{itemize}
    \item \textbf{In-person Session}: Performance Management
    \item Readings:
        \begin{itemize}
            \item \citet{Behn2003}, ``Why Measure Performance?'' 
            \item \citet{Denhardt2015}, \emph{The New Public Service}, Chapter 9 
            \item \citet{douglas2021}, ``Getting a Grip on Performance of Collaborations'' 
            \item \citet{marvel2015}, ``Unconscious Bias in Citizens' Evaluations\dots'' 
            %\item \citet{nicholson-crotty2004}, ``Public Management and Organizational Performance'' 
        \end{itemize}
        \item Due Wednesday: Annotated Bibliography and Synthesis
        \item Due Friday: Peer Review
        \item Due Saturday: Final Synthesis and Personal Reflection
\end{itemize}

\subsection*{Week 6 -- 10/3: Street-Level Bureaucrats}
\begin{itemize}
    \item \textbf{In-person Session}: Street-Level Bureaucrats
    \item Readings:
        \begin{itemize}
            \item \citet{Lipsky2010}, \emph{Street-Level Bureaucracy}, Chapters 1--10 
            \item \citet{Denhardt2015}, \emph{The New Public Service}, Chapters 5--6 
        \end{itemize}
        \item Due Wednesday: Annotated Bibliography and Synthesis
        \item Due Friday: Peer Review
        \item Due Saturday: Final Synthesis and Personal Reflection
\end{itemize}

\subsection*{Week 7 -- 10/10: Privatization and Contracting}
\begin{itemize}
    \item \textbf{In-person Session}: Privatization and Contracting
    \item Readings:
        \begin{itemize}
            \item \citet{MILWARD2000a}, ``Governing the Hollow State''
            \item \cite{hood1991}, ``A Public Management for All Seasons?''
            \item \citet{brown2006}, ``Managing Public Service Contracts''
            \item \citet{jos2009}, ``Keeping it Public''
            \item \citet{rainey2000}, ``Comparing Public and Private Organizations''
        \end{itemize}
        \item Due Wednesday: Annotated Bibliography and Synthesis
        \item Due Friday: Peer Review
        \item Due Saturday: Final Synthesis and Personal Reflection
\end{itemize}

\subsection*{Week 8 -- 10/17: 21st Century Challenges}
\begin{itemize}
    \item \textbf{In-person Session}: 21st Century Challenges
    \item Readings:
        \begin{itemize}
            \item \citet{maynard-moody2012}, ``Social Equities and Inequities in Practice'' 
            \item \citet{GOODEN2017}, ``Social Equity and Evidence'' 
            \item \citet{mccandless2022}, ``A Long Road''
            \item \citet{Denhardt2015}, \emph{The New Public Service}, Chapters 10--12 
        \end{itemize}
        \item Due Wednesday: Annotated Bibliography and Synthesis
        \item Due Friday: Peer Review
        \item Due Saturday: Final Synthesis and Personal Reflection
\end{itemize}

\subsection*{Week 9 -- 10/24: Comprehensive General Area Essay}
\begin{itemize}
    \item \textbf{Asynchronous Session}: Comprehensive General Area Essay Study Break
\end{itemize}


\subsection*{Week 10 -- 10/31: Comprehensive General Area Essay Exam}
\begin{itemize}
    \item \textbf{Asynchronous Session}: Comprehensive General Area Essay Exam Distributed
\end{itemize}   

\subsection*{Week 11 -- 11/7: Comprehensive General Area Essay Exam}
\begin{itemize}
    \item \textbf{Asynchronous Session}: \textbf{Comprehensive General Area Essay Exam Due}
\end{itemize}

\subsection*{Week 12 -- 11/14: Concentration Area Literature Review}
\begin{itemize}
    \item \textbf{In-Person Session}: Concentration Area Literature Review
    \item Literature review expectations and guidelines
    \item Due: Literature Review Topic outline
\end{itemize}

\subsection*{Week 13 -- 11/21: Concentration Area Literature Review}
\begin{itemize}
    \item \textbf{Asynchronous Session}: Concentration Area Literature Review
    \item Due: Literature Review Annotated Bibliography
\end{itemize}

\subsection*{Week 14 -- 12/5: Concentration Area Literature Review}
\begin{itemize}
    \item \textbf{Asynchronous Session}: Concentration Area Literature Review
    \item Due: Literature Review Draft
\end{itemize}

\subsection*{Week 15 -- 12/12: Concentration Area Literature Review}
\begin{itemize}
    \item \textbf{In-Person Session}: Concentration Area Literature Review
    \item Due: Literature Review Peer Review
    \item Course Wrap-Up and Final Reflections
    \item Potluck Celebration
\end{itemize}

\subsection*{Week 16 -- 12/19: Course Conclusion}
\begin{itemize}
    \item \textbf{Asynchronous Session}: Course Conclusion
    \item Due: Concentration Area Literature Review Final Draft
\end{itemize}


            \singlespace
            \bibliographystyle{apsr}
            \bibliography{521_Fall2024}

% updated date and time
\begin{flushright}
Updated: \today
\end{flushright}

            
\end{document}