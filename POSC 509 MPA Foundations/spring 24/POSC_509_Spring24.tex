\documentclass[10pt, letterpaper]{article}
\usepackage[english]{babel}
\usepackage[T1]{fontenc}
\usepackage[margin=1.5in]{geometry}
\usepackage{xcolor}
\usepackage{url}
\usepackage[utf8]{inputenc}
\usepackage[sfdefault]{roboto}
\usepackage{tabularx}
\usepackage{booktabs}
\frenchspacing
\usepackage{multicol}
\usepackage{eso-pic}
\usepackage[longnamesfirst]{natbib}
\bibpunct{(}{)}{;}{a}{}{,}
\usepackage{caption}
\usepackage{subcaption}
\usepackage{setspace}
\usepackage{paralist}
\usepackage{quoting}
\usepackage{comment}
\usepackage{enumitem}
\usepackage{graphicx}
\usepackage{float}
\usepackage{bookmark}
\renewcommand{\thesection}{\arabic{section}.}
\renewcommand{\thesubsection}{\thesection\arabic{subsection}}
\renewcommand{\thesubsubsection}{\thesubsection.\arabic{subsubsection}}
\usepackage{hyperref}
\hypersetup{
    colorlinks=true,
    linkcolor=blue,
    citecolor=magenta,      
    urlcolor=blue
}

\usepackage{fancyhdr}
\usepackage{graphicx}
\pagestyle{fancy}
\renewcommand{\headrulewidth}{0pt}
\fancyhead{}
\fancyfoot{}
\fancyfoot[C]{\thepage}

\fancypagestyle{plain}{
  \fancyhead{}
  \fancyhead[C]{\includegraphics[width=8cm]{csuf_logo.png}}
  \fancyfoot{}
  \renewcommand{\headrulewidth}{0pt}
}


\usepackage{titlesec}
\titleformat{\section}{\normalfont\fontsize{14}{15}\bfseries}{\thesection}{1em}{}
\titleformat{\subsection}{\normalfont\fontsize{12}{15}\bfseries}{\thesubsection}{1em}{}
\titleformat{\subsubsection}{\normalfont\fontsize{12}{15}\bfseries}{\thesubsubsection}{1em}{}
\titleformat{\paragraph}{\normalfont\fontsize{12}{15}\bfseries}{\theparagraph}{1em}{}
\titleformat{\subparagraph}{\normalfont\fontsize{12}{15}\bfseries}{\thesubparagraph}{1em}{}
\titlespacing*{\section}{0pt}{0.5\baselineskip}{0.5\baselineskip}
\titlespacing*{\subsection}{0pt}{0.5\baselineskip}{0.5\baselineskip}
\titlespacing*{\subsubsection}{0pt}{0.5\baselineskip}{0.5\baselineskip}
\titlespacing*{\paragraph}{0pt}{0.5\baselineskip}{0.5\baselineskip}
\titlespacing*{\subparagraph}{0pt}{0.5\baselineskip}{0.5\baselineskip}
 
    
    
    \begin{document}
    \title{Foundations of Public Administration}
    
    \author{POSC 509 -- Spring 2024}
    \date{Thursdays at 7:00 in Langsdorf Hall 401}
    
        \maketitle

\subsection*{Professor: David P. Adams, Ph.D.}

\subsubsection*{Contact Information:}

\begin{itemize}
	\item Office: 516 Gordon Hall
	\item Phone/Text: (657) 278-4770
	\item website: \href{https://dadams.io}{\texttt{dadams.io}}
	\item email: \href{dpadams@fullerton.edu}{\texttt{dpadams@fullerton.edu}}
	\item Office hours: Tuesdays \& Thursdays from 9:30 to 11:00, Thursdays from 5:30 to 6:30, and by \href{https://dadams.io/appt}{appointment}.
	\item Schedule meetings throughout the week: \href{https://dadams.io/appt}{\texttt{dadams.io/appt}}
\end{itemize}

\section*{Catalog Description}

Introduction to the field of public administration. Topics include the history of the field, the legal and political environment of public administration, organizational theory and practice, decision-making, systems analysis, performance evaluation and administrative improvement. Non-Public Administration/Political Science graduate students must obtain department permission to enroll.

\section*{Course Description}

This course is a graduate-level introduction to the study and practice---the science and art---of public administration. It is for graduate students in public administration who have yet to have an introductory course in public administration. It is meant to acquaint the student with the theoretical and practical aspects of public administration in the American political setting. Topics include organizational theory and practice, decision-making, systems analysis, performance evaluation, and administrative improvement. Emphasis is also placed on understanding the roles and responsibilities of public administrators in a democratic political system.
\parskip=0.5\baselineskip
\noindent The course begins with an introduction to the field of public administration, a discussion of the rise of the administrative state, and the role of federalism and intergovernmental politics. Numerous classic articles from early public administration scholarship will be introduced, and the student will gain an understanding of the relevance of public administration in theory through examination of a wide variety recent literature. The course will focus on the evaluation of public administration in theory and practice.

\section*{Course Objectives}

The student should gain both a broad understanding of public administration as an academic discipline, as well as a better understanding the elements of public administration in practice. At the end of this course, students should be \underline{introduced} to concepts related to several of programmatic student learning outcome goals. In particular, at an introductory level, students will be able to 
	\begin{enumerate}
		\item understand the nature, context, and foundations of public administration;
		\item understand the importance of citizen engagement;
		\item learn the skills necessary to be an effective public administrator;
		\item synthesize relevant information to address public problems using major theories underlying the field of public administration, including those related to governance structures, federalism and intergovernmental relations, and intersectoral relations; 
		\item articulate and apply a public service perspective in the demonstration of knowledge related to the structures, components, goals, and objectives of the public sector; 
		\item effectively communicate and productively interact with diverse teams and diverse communities; and 
		\item articulate and appreciate the value of diversity in the public sector and the communities it serves.
	\end{enumerate}

\section*{Required Texts}

\begin{enumerate}
    \item Ashworth, Kenneth. 2001. \emph{Caught Between the Dog and the Fireplug, or How to Survive Public Service}. Washington, D.C.: Georgetown Press. (\textit{Ashworth})
    \item Guy, Mary E. and Todd L. Ely. 2022. \emph{Essentials of Public Service: An Introduction to Contemporary Public Administration}. 2nd ed. Irvine, CA: Melvin \& Leigh. (\textit{Essentials})
    \item Doucette, Meriem and David Adams, eds. 2019. \emph{Ethics in Public Administration}. San Diego: Cognella. (\textit{Ethics})
\end{enumerate}

\subsection*{Additional Readings}

In addition to the above texts, several additional readings, including articles, book chapters, and case studies, are posted on Canvas and are noted in the course schedule at the end of this document. 


\section*{University Student Policies}

In accordance with UPS 300.00, students must be familiar with certain policies applicable to all courses. Please review these policies as needed and visit this Cal State Fullerton website \href{https://t.ly/csuf-syllabus}{https://t.ly/csuf-syllabus} for links to the following information:

\begin{enumerate}
    \item   University learning goals and program learning outcomes.
    \item	Learning objectives for each General Education (GE) category.
    \item	Guidelines for appropriate online behavior (netiquette).
    \item	Students’ rights to accommodations for documented special needs.
    \item   Campus student support measures, including Counseling \& Psychological Services, Title IV and Gender Equity, Diversity Initiatives and Resource Centers, and Basic Needs Services.
    \item	Academic integrity (refer to UPS 300.021).
    \item	Actions to take during an emergency.
    \item	Library services information.
    \item	Student Information Technology Services, including details on technical competencies and resources required for all students.
    \item	Software privacy and accessibility statements.
\end{enumerate}

\section*{Course Student Policies}

\subsection*{Course Communication}
All course announcements and communications will be sent via \emph{Canvas} and university email. Students are responsible for regularly checking their \emph{Canvas} notifications and email. Students are also responsible for ensuring that their \emph{Canvas} notifications are set to receive messages from the course. Students are expected to check \emph{Canvas} and their email at least once daily.

\subsection*{Due Dates}
If you have concerns about meeting assignment deadlines, please get in touch with the professor in advance to discuss potential accommodation. Late work is not accepted without prior approval from the professor.

\subsection*{Alternative Procedures for Submitting Work}
Students are expected to submit all assignments via \emph{Canvas}. If you cannot submit an assignment via \emph{Canvas}, please get in touch with the professor to discuss alternative submission procedures.

\subsection*{Extra Credit}
There are no extra credit assignments in this course. 

\subsection*{Academic Integrity}
Students are expected to adhere to the highest standards of academic integrity. Any student found to have engaged in academic dishonesty will be subject to the sanctions described in the \href{https://www.fullerton.edu/senate/publications_policies_resolutions/ups/UPS%20300/UPS%20300.021.pdf}{Academic Dishonesty Policy} (UPS 300.021). Academic dishonesty includes, but is not limited to, cheating, plagiarism, fabrication, facilitating academic dishonesty, and submitting previously graded work without prior authorization. Students are expected to be familiar with the university's policy on academic dishonesty and to adhere to this policy in all aspects of this course. Any student who has questions about the policy should ask the professor for clarification.


\subsection*{Response Time} I will strive to respond to all student emails and \emph{Canvas} messages within 24 hours, except on weekends and holidays. If you are still awaiting a response within 24 hours, please send a follow-up message. If you are still waiting to receive a response within 48 hours, please send another follow-up message and contact me via phone or SMS at (657) 278-4770.


\section*{Kritik: Sharpening Your Peer-Review Skills}

This term, we'll leverage Kritik, a dynamic peer learning platform, to hone your critical thinking and communication skills—essential tools for any aspiring public administrator. Through Kritik, you'll analyze real-world policy scenarios, provide constructive feedback to peers, and receive valuable insights on your own work.

\begin{itemize}

\item \textbf{A Three-Stage Learning Journey:}

    \begin{enumerate}
        \item Craft Your Analysis: Follow the provided rubric and delve into a public policy challenge. This could involve, for example, evaluating the ethical implications of a proposed environmental regulation or assessing the effectiveness of a social welfare program.
        \item Provide Constructive Critique: Anonymously evaluate your peers' work using the rubric. Offer actionable feedback that focuses on the strengths and weaknesses of their analysis, drawing connections to relevant public administration concepts.
        \item Reflect and Improve: Receive anonymous feedback on the quality and impact of your comments. Learn to deliver clear, concise, and impactful feedback—a crucial skill for public servants collaborating on complex issues.
    \end{enumerate}

\item \textbf{Grading and Participation:} You'll earn four scores for each Kritik activity: Creation, Evaluation, Feedback, and Overall. These scores, along with active participation, will contribute to your course grade. Participating thoughtfully in Kritik activities will not only improve your own skills but also enrich the learning experience for your peers.

\item \textbf{Registration and Support:} We'll thoroughly introduce Kritik in class, and a dedicated email invitation will guide you through registration and course enrollment. The Kritik Help Center offers additional resources, and I'm always available to address any questions or concerns.

\end{itemize}

\section*{Course Requirements}

The sections below outline assignments for this course and the ways in which they are graded. It is up to the student to carefully read each of the assignments, as well as to notice their due dates. Kritik assignments are due in three stages: (1) the initial submission, (2) the peer review, and (3) the reflection. Initial submissions are due as indicated below. Peer reviews are due 72 hours after the initial submission deadline. Reflections are due 48 hours after the peer review deadline.
	\begin{enumerate}

        \item \textbf{Kritik Orientation:} This assignment is designed to familiarize you with Kritik.
		
		\item \textbf{Memorandum:} Following the advice in ``Skillbox: How to Write A Memorandum'' at the end of chapter 1 in \emph{Essentials} \citep{GUY2022} located on pages 24--25, write a no more than 500--600 word page memo based on the following prompt: \emph{Why am I here? Why do I want a career in public administration? What do I hope to get out of this course?} This assignment is worth 10\% of the student's grade. Turn in a hard copy of this assignment at the beginning of the week 5 class meeting.
		
		\item \textbf{Reading Presentation:} Each student will present a chapter from \emph{Ethics} \citep{DOUCETTE2020} or an another reading posted to \emph{Canvas} and lead a class discussion. Each class member should read the material. Students tasked with presentation and discussion should present the reading to the class in approximately 20 minutes. With the assistance of fellow classmates and the professor, the student will then be responsible for leading a class discussion and answering questions on the case for approximately 10 minutes. Students should prepare a short memo on the reading that they will distribute to the class; or, alternatively, students can create a PowerPoint presentation and distribute copies to the class. The memo or PowerPoint should briefly summarize and overview the chapter and then connect it to other readings from the from class. This will be 10\% of the student's overall grade. It will be assessed for thoroughness, familiarity with the reading, presentation style, and engagement with classmates.
		
	
		\item \textbf{Papers}
			\begin{itemize}
				\item Papers are to be submitted on \emph{Canvas} according to the schedule below. Find the link on \emph{Canvas} and \underline{upload} a paper for each of the following questions:
				\begin{enumerate}
					\item \underline{Paper 1}: In about 500 words, describe your vision of the ideal public administrator. What are this person's goals? What characteristics and skills does this person possess that will make it possible to achieve those goals? Is there a public administrator you believe fits your vision? If so, who and how. \emph{Use the Essentials textbook and/or other sources to support your ideas.}
					\item \underline{Paper 2}: In about 750--1,000 words, describe a real-life example of ethical misconduct in public contracting. How was it unethical? How was the misconduct discovered? What was the outcome? What type of solution is appropriate here? \emph{NOTE: for this paper, you must use a minimum of 3 sources (Essentials textbook and/or others).}
				\end{enumerate}
			\end{itemize}
		\item \textbf{Budgeting Assignment:} This assignment is related to Chapter 6 in the \emph{Essentials} textbook. For this assignment you are tasked with 1) developing an inventory of nonprofits in the state that provide housing services, 2) limiting that pool of potential collaborators to those with substantial assets and experience, 3) prioritizing the potential partners based on income, and 4) presenting an aggregated overview of the identified partners. Specific instructions and details, as well as the data, are available on the Canvas module for this topic. 
		
		\item \textbf{Program Evaluation Assignment:} This assignment is located at the end of Chapter 14 in the \emph{Essentials} textbook. Read and follow the instructions in ``Skillbox: Essentials of Program Evaluation'' located on pages 368--372. For this assignment you are tasked with 1) writing a two-page memo that proposes an evaluation plan for a neighborhood policing program using the information in the Skillbox reading and the chapter and 2) find an existing program evaluation report online and write a one-page memo to the evaluated organization. 
		
		\item \textbf{Canvas Discussion on \emph{Caught Between the Dog and the Fireplug}:} This course includes a dynamic, 10-week structured discussion on \cite{ASHWORTH2001}, \emph{Caught Between the Dog and the Fireplug, or How to Survive Public Service.} Each week, as indicated on Canvas, we'll explore different themes from the book, ranging from the intricacies of public service and bureaucracy to ethical dilemmas and leadership challenges in public administration. These discussions are designed to not only deepen your understanding of key concepts in public administration but also to connect these ideas with real-world applications and your personal experiences. Through active participation, you'll gain valuable insights into the practical aspects of public service and develop skills essential for your future career in this field.
		
		\item \textbf{Final Response:} The final response essay will consist of six question prompts related to course readings and class discussion from the semester. Students will answer two of the six prompts, writing approximately 1,000 words for each response. The question prompts will be distributed in class in Week 15 and the students submission should be uploaded to \emph{Canvas} no later than 10:00 p.m. the following week. 
	\end{enumerate}

    \section*{Grades}

    \subsection*{Grading Scale and Grade Weights}  
    The grading scale is shown in Table~\ref{tab:grading-scale}. Grades will be given based on Table~\ref{tab:grade-weights} weights. Assignment due dates are indicated on Canvas and Kritik. 
    
    \begin{table}[ht]
    \centering
    \caption{Grading Scale}
    \begin{tabular}{llll}
    \toprule
    \textbf{Grade} & \textbf{Percentage} & \textbf{Grade} & \textbf{Percentage} \\
    \midrule
    A+ & 98.0 – 100 & B- & 80.0 – 81.9\\
    A & 92.0 – 97.9 & C+ & 78.0 – 79.9\\
    A- & 90.0 – 91.9 & C & 72.0 – 77.9\\
    B+ & 88.0 – 89.9 & C- & 70.0 – 71.9\\
    B & 82.0 – 87.9 & & \\
    \bottomrule
    \end{tabular}
    \label{tab:grading-scale}
    \end{table}
    
    \begin{table}[ht]
        \centering
        \caption{Graded Items and Points}
        \begin{tabular}{ll}
            \toprule
        \textbf{Assignment} & \textbf{Points} \\
        \midrule
        1. Hostile Architecture Podcast \& Discussion & 2 \\
        2. Kritik Orientation & 2 \\
        3. Memorandum & 3 \\
        4. Student-led Reading Discussion & 5 \\
        5. Paper 1 & 15 \\
        6. Paper 2 & 20 \\
        7. Budgeting Assignment & 18 \\
        8. Program Evaluation Assignment & 10 \\
        9. \emph{Ashworth} discussion & 10 \\
        10. Final Response & 25 \\ \bottomrule
        \emph{Total} & \emph{100} \\
        
        \end{tabular}
        \label{tab:grade-weights}
        \end{table}


\section*{Course Schedule}

\noindent Note: On the schedule below, each particular reading is indicated for the class day on which it will be discussed.  
    
    \subsection*{Week 1: January 25}
        \begin{itemize}
            \item Podcast and Discussion
            \item Kritik Orientation
        \end{itemize}

    \subsection*{Week 2: February 1} 
    \begin{itemize}
        \item Introduction and Overview
        \item Student Presentations Overview
        \item Basics about Graduate School and Using Library Website
        \item Eras of Public Administration
    \end{itemize}

    \subsection*{Week 3: February 8} 
    \begin{itemize}
        \item Read: ``Running a Constitution'' ch. 1 in \emph{Essentials}
        \item Read: ``Citizen Engagement'' ch. 2 in \emph{Essentials}
        \item Due: Memorandum on Kritik
    \end{itemize}

    \subsection*{Week 4: February 15} 
    \begin{itemize}
        \item Read: \cite{WILSON1887b}
        \item \underline{Student Presentation 1}
        \begin{itemize}
            \item \cite{DESLATTE2020}
        \end{itemize}	
        \item Read: ``The Savvy Administrator'' ch. 3 in \emph{Essentials}
        \item Read: ``Organizing Principles'' ch. 4 in \emph{Essentials}
    \end{itemize}

    \subsection*{Week 5: February 22} 
    \begin{itemize}
        \item Organizing Principles Continued
        \item Read: \cite{SIMON1946}
        \item Watch: YouTube videos linked on \emph{Canvas}
        \item Due: Hard copy of Memorandum
        \item Due: Paper 1 on \emph{Kritik}
    \end{itemize}

    \subsection*{Week 6: February 29} 
    \begin{itemize}
        \item Read: ``Human Resource Functions and Processes'' ch. 5 in \emph{Essentials} 
        \item Read: ``Budgeting'' ch. 6 in \emph{Essentials}
        \item \underline{Student Presentations 2 \& 3}
        \begin{itemize}
            \item Student Presentation 2
            \begin{itemize}
                \item \cite{PERLMAN2016}
            \end{itemize}
            \item Student Presentation 3
            \begin{itemize}
                \item \cite{AFONSO2021}
            \end{itemize}
        \end{itemize}
    \end{itemize}

    \subsection*{Week 7: March 7} 
    \begin{itemize}
        \item Read: ``Digital Democracy'' ch. 7 in \emph{Essentials}
        \item \underline{Student Presentations 4 \& 5}
        \begin{itemize}
            \item Student Presentation 4
            \begin{itemize}
                \item \cite{YOUNG2020a}
            \end{itemize}
            \item Student Presentation 5
            \begin{itemize}
                \item \cite{KIM2021}
            \end{itemize}
        \end{itemize}	
        \item Due: Budgeting Assignment on \emph{Kritik}
    \end{itemize}

    \subsection*{Week 8: March 14} 
    \begin{itemize}
        \item Read: ``Foundations of Ethics in Public Administration'' ch. 1 in \emph{Ethics}
        \item Read: ``What is Ethics'' ch. 2 in \emph{Ethics}
        \item \cite{OLEARY2010}
        \item \underline{Student Presentation 6}
        \begin{itemize}
            \item \cite{HOLLIBAUGHJR.2020a}
        \end{itemize}
    \end{itemize}

    \subsection*{Week 9:  March 21} 
    \begin{itemize}
        \item Read: ``Public Economics and Policy'' ch. 8 in \emph{Essentials}
        \item Read: ``Transparency'' ch. 10 in \emph{Essentials}
    \end{itemize}

    \subsection*{Week 10: March 28}
    \begin{itemize}
        \item Read: ``Public Integrity'' ch. 13 in \emph{Essentials}
        \item Read: \underline{Student Presentations 7, 8, \& 9}
        \begin{itemize}
            \item Student Presentation 7: Chapter 7 in \emph{Ethics}
            \item Student Presentation 8: Chapter 8 in \emph{Ethics}
            \item Student Presentation 9: Chapter 9 in \emph{Ethics}
        \end{itemize}
    \end{itemize}

    \subsection*{Week 11: April 11}
    \begin{itemize}
        \item Decision Making
        \begin{itemize}
            \item Read: \cite{SIMON1965}
            \item Read: \cite{Lindblom1959}
            \item Read: \cite{OSTROM1971}
            \item TBD: ``\textit{The Abilene Paradox}'': Video and Discussion
        \end{itemize}
    \end{itemize}

    \subsection*{Week 12: April 18}
    \begin{itemize}
        \item Read: ``Contracting'' ch. 11 in \emph{Essentials}
        \item Read: ``Legal Dimensions of Public Administration'' ch. 12 in \emph{Essentials}
        \item \underline{Student Presentations 10 \& 11}
        \begin{itemize}
            \item Student Presentation 10
            \begin{itemize}
                \item \cite{ZEEMERING2018}
            \end{itemize}
            \item Student Presentation 11
            \begin{itemize}
                \item \cite{CHRISTENSEN2011}
            \end{itemize}
        \end{itemize}
        \item Due: Paper 2 on \emph{Kritik}
    \end{itemize}

    \subsection*{Week 13: April 25}
    \begin{itemize}
        \item Read: ``Measuring and Managing for Performance'' ch. 14 in \emph{Essentials}
        \item Read: ``Public Services, Well Delivered'' ch. 15 in \emph{Essentials}
        \item Read: \cite{YOUNG2020b}
        
    \end{itemize}           

    \subsection*{Week 14: May 2}
    \begin{itemize}
        \item Contemporary Public Administration: Representative Democracy and Intersectionality in Public Administration
        \item \underline{Student Presentations 12, 13, \& 14}
        \begin{itemize}
            \item Student Presentation 12
            \begin{itemize}
                \item \cite{Headly2021}
            \end{itemize}
            \item Student Presentation 13
            \item \begin{itemize}
                \item \cite{mccandless2022}
            \end{itemize}
            \item Student Presentation 14
            \item \begin{itemize}
                \item \cite{Zavattaro2022}
            \end{itemize}
        \end{itemize}
        \item Due: Program Evaluation Assignment on \emph{Kritik}
    \end{itemize}

    \subsection*{Week 15: May 9}
    \begin{itemize}
        \item \textbf{Movie Night}
        \item \textbf{Pot-Luck Dinner}
        \item \emph{Canvas} Brief Reading and Discussion Board for Movie
    \end{itemize}

    \subsection*{Week 16: May 16}
    \begin{itemize}
        \item \textbf{Final Response Due by 10:00 p.m.}
    \end{itemize}
    \section*{Technical Problems}
    
    \subsection*{University IT Help Desk}
    
    Contact the instructor immediately to document the problem if you encounter any technical difficulties. Then contact the \href{http://www.fullerton.edu/it/students/helpdesk/index.php}{Student IT Help Desk} for assistance. You can also call the Student IT Help Desk at (657) 278-8888, \href{mailto:StudentITHelpDesk@fullerton.edu}{email}, visit them at the Pollak Library North \href{http://www.fullerton.edu/it/students/sgc/index.php}{Student Genius Center}, or log on to the \href{http://my.fullerton.edu/}{my.fullerton.edu} portal and click ``Online IT Help'' followed by ``Live Chat''.
    
    \subsection*{Canvas Support}
    
    If you encounter any technical difficulties with Canvas, call the Canvas Support Hotline at 855-302-7528, visit the \href{https://community.canvaslms.com/docs/DOC-10720-67952720329}{Canvas Community}, or click the ``Help'' button in the lower left corner of Canvas and select ``Report a Problem''. The \href{https://cases.canvaslms.com/liveagentchat?chattype=student&sfid=001A000000YzcwQIAR}{Student Support Live Chat} is available 24 hours a day, 7 days a week.
    
    \subsection*{Kritik Support}
    
    If you have any questions about Kritik, please use their live chat. A human agent will respond promptly within a few minutes from 9am-5pm eastern time Monday to Friday. Outside of these hours, you’ll receive a reply that Kritik will be back the next business day. They monitor around the clock and will still respond to urgent requests within a few hours. You can also visit \href{https://help.kritik.io}{\texttt{https://help.kritik.io}} to view their help articles.
    
    \section*{Important Scheduling Note}

    Please be aware the California Faculty Association---the labor union of Lecturers, Professors, Coaches, Counselors, and Librarians across the 23 CSU campuses---is navigating challenging contract negotiations with CSU management, and a strike or work stoppage may occur this term. Our working conditions are your learning conditions; we seek to protect both. For updates, visit \href{www.CFAbargaining.org}{\texttt{www.CFAbargaining.org}}.
    

\singlespace
\bibliographystyle{apsr}
\bibliography{509}


\end{document}
    
