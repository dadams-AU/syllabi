\documentclass[11pt]{article} 
\usepackage[margin=1in]{geometry} 
\usepackage{graphicx} 
\usepackage{url} 
\usepackage[hidelinks]{hyperref} 
\hypersetup{ colorlinks = true, linkcolor = black, citecolor = gray, urlcolor = blue } 
\usepackage{enumitem} 
\usepackage{hanging} 
\frenchspacing 
\usepackage{eso-pic} 
\usepackage{multicol} 
\usepackage{amsmath}
\begin{document}

\AddToShipoutPicture*{\put(500,700){\includegraphics[width=2.5cm,height=2.5cm]{meeting_qr}}}

\title{\textbf{Foundations of Public Administration} \\
POSC 509 ~\textperiodcentered~ Spring 2023}

\author{ \Large{Instructor: \textsc{David P. Adams, Ph.D. }} \\
\\
\textbf{Office:} 516 University Hall \\
\textbf{Phone:} (657) 278-4770 ~\textperiodcentered~ \textbf{Email:} \url{dpadams@fullerton.edu} \\\\
\textbf{Office Hours:}  Tuesdays and Thursdays 9:30 - 11:30, \\  Thursdays 5:30 - 6:30, and by appointment at \url{https://t.ly/4dpa}}

\date{ \textbf{Dates:} January 26 -- May 18 ~\textperiodcentered~ \textbf{Time:} Thursdays 7:00--9:45 \textsc{p.m.} \\\vspace{1ex} \textbf{Place:} GH 305}

\maketitle
\begin{description}
	
	\item[Course Description:] This course is a graduate-level introduction to the study and practice---the science and art---of public administration. It is for graduate students in public administration who have yet to have an introductory course in public administration. It is meant to acquaint the student with the theoretical and practical aspects of public administration in the American political setting. Topics include organizational theory and practice, decision-making, systems analysis, performance evaluation, and administrative improvement. Emphasis is also placed on understanding the roles and responsibilities of public administrators in a democratic political system.
	
	The course begins with an introduction to the field of public administration, a discussion of the rise of the administrative state, and the role of federalism and intergovernmental politics. Numerous classic articles from early public administration scholarship will be introduced, and the student will gain an understanding of the relevance of public administration in theory through examination of a wide variety recent literature. The course will focus on the evaluation of public administration in theory and practice.
	
	\item[Course Objectives:] The student should gain both a broad understanding of public administration as an academic discipline, as well as a better understanding the elements of public administration in practice. At the end of this course, students should be \underline{introduced} to concepts related to several of programmatic student learning outcome goals. In particular, at an introductory level, students will be able to 
	\begin{enumerate}
		\item understand the nature, context, and foundations of public administration;
		\item understand the importance of citizen engagement;
		\item learn the skills necessary to be an effective public administrator;
		\item synthesize relevant information to address public problems using major theories underlying the field of public administration, including those related to governance structures, federalism and intergovernmental relations, and intersectoral relations; 
		\item articulate and apply a public service perspective in the demonstration of knowledge related to the structures, components, goals, and objectives of the public sector; 
		\item effectively communicate and productively interact with diverse teams and diverse communities; and 
		\item articulate and appreciate the value of diversity in the public sector and the communities it serves.
	\end{enumerate}
	
	\item[Required Texts:] There are three textbooks for this course: 
	\begin{enumerate}
		\item Ashworth, Kenneth. 2001. \emph{Caught Between the Dog and the Fireplug, or How to Survive Public Service}. Washington, D.C.: Georgetown Press. (\textit{Ashworth})
		\item Guy, Mary E. and Todd L. Ely. 2022. \emph{Essentials of Public Service: An Introduction to Contemporary Public Administration}. 2nd ed. Irvine, CA: Melvin \& Leigh. (\textit{Essentials})
		\item Doucette, Meriem and David Adams, eds. 2019. \emph{Ethics in Public Administration}. San Diego: Cognella. (\textit{Ethics})
	\end{enumerate}
	
	\item[Expectations:] To accomplish the course objectives above, the student must be in the classroom on time, completing the reading materials and assignments before class begins. Assignments are always due at the beginning of class. I expect students to attend all class meetings. You have two unexcused no-questions-asked absences, but please let me know in advance or shortly after so, I can provide you with any notes or additional materials. Participation is essential in this course; everyone must be present and ready to discuss the material. The instructor will call on volunteers and non-volunteers to answer questions and evaluate the readings. Class participation, discussion with your peers, and individual and team responsibilities are important parts of professional development. Failure to attend class, answer questions, participate in team assignments, evaluate material, ask questions, and contribute meaningfully to the class discussion will significantly reduce the student's grade. If the student wishes to learn from and enjoy this course, it is up to them to make it happen.
	
	One important part of the instructor's job is to help the student develop professionally in the MPA program. Students must regularly attend MPA events, like socials and guest lectures. More importantly, students must regularly visit the instructor's office for this course to ask questions or share concerns, insights, and experiences. The most important part of the instructor's job is to help the student succeed in this course, in the discipline, and the profession. 
	
	\item[Online Resources:] All course documentation will be placed on Canvas. The student is responsible for checking Canvas for course announcements and documents. Email messages will be sent to the class using this system. The student should regularly check his or her email account that is connected to Canvas. 
	\vspace{3ex}
	\begin{center}
		\begin{Large}
			The Important Stuff: 
		\end{Large}
	\end{center}
	\vspace{3ex}
	 
	\item[Course Assignments and Grading:] The sections below outline assignments for this course and the ways in which they are graded. It is up to the student to carefully read each of the assignments, as well as to notice their due dates.
	\begin{enumerate}
		
		\item \textbf{Memorandum:} Following the advice in ``Skillbox: How to Write A Memorandum'' at the end of chapter 1 in \emph{Essentials} located on pages 24--25, write a no more than 500--600 word page memo based on the following prompt: \emph{Why am I here? Why do I want a career in public administration? What do I hope to get out of this course?} This assignment is worth 10\% of the student's grade. Turn in a hard copy of this assignment at the beginning of the second class meeting.
		
		\item \textbf{Reading Presentation:} Each student will present a chapter from \emph{Ethics} or an another reading posted to \emph{Canvas} and lead a class discussion. This will be designed as a student-led ``forum''; it is worth 10\% of the student's grade. Each class member should read the material. Students tasked with presentation and discussion should present the reading to the class in approximately 20 minutes. With the assistance of fellow classmates and the professor, the student will then be responsible for leading a class discussion and answering questions on the case for approximately 10 minutes. Students should prepare a short memo on the reading that they will distribute to the class; or, alternatively, students can create a PowerPoint presentation and distribute copies to the class. The memo or PowerPoint should briefly summarize and overview the chapter and then connect it to other readings from the from class. This will be 10\% of the student's overall grade. It will be assessed for thoroughness, familiarity with the reading, presentation style, and engagement with classmates.
		
	
		\item \textbf{Papers}
			\begin{itemize}
				\item Papers are to be submitted on \emph{Canvas} according to the schedule below. Each of the two papers are worth 10\% of the student's grade. Find the link on \emph{Canvas} and \underline{upload} a paper for each of the following questions:
				\begin{enumerate}
					\item \underline{Paper 1}: In about 500 words, describe your vision of the ideal public administrator. What are this person's goals? What characteristics and skills does this person possess that will make it possible to achieve those goals? Is there a public administrator you believe fits your vision? If so, who and how. \emph{Use the Essentials textbook and/or other sources to support your ideas.}
					\item \underline{Paper 2}: In about 750--1,000 words, describe a real-life example of ethical misconduct in public contracting. How was it unethical? How was the misconduct discovered? What was the outcome? What type of solution is appropriate here? \emph{NOTE: for this paper, you must use a minimum of 3 sources (Essentials textbook and/or others).}
				\end{enumerate}
			\end{itemize}
		\item \textbf{Budgeting Assignment:} This assignment is related to Chapter 6 in the \emph{Essentials} textbook. For this assignment you are tasked with 1) developing an inventory of nonprofits in the state that provide housing services, 2) limiting that pool of potential collaborators to those with substantial assets and experience, 3) prioritizing the potential partners based on income, and 4) presenting an aggregated overview of the identified partners. Specific instructions and details, as well as the data, are available on the Canvas module for this topic. This assignment is worth 10\% of the student's grade.
		
		\item \textbf{Program Evaluation Assignment:} This assignment is located at the end of Chapter 14 in the \emph{Essentials} textbook. Read and follow the instructions in ``Skillbox: Essentials of Program Evaluation'' located on pages 368--372. For this assignment you are tasked with 1) writing a two-page memo that proposes an evaluation plan for a neighborhood policing program using the information in the Skillbox reading and the chapter and 2) find an existing program evaluation report online and write a one-page memo to the evaluated organization. This assignment is worth 10\% of the student's grade. 
		
		\item \textbf{Ashworth Assignment:} From the beginning of the semester, students should read \emph{Ashworth's} book. Briefly summarize each chapter and discuss the advice or points that stand out to you as especially important or relevant to your current or future career. This will be a discussion post that you will update with each chapter you read. Throughout the semester we will discuss the book and it's importance to our careers. Discussion details are on Canvas.
		
		\item \textbf{Final Response:} The final response essay will consist of six question prompts related to course readings and class discussion from the semester. Students will answer three of the six prompts, writing approximately 1,000 words for each response. The question prompts will be distributed on December 9th and the students submission should be uploaded to \emph{Canvas} no later than 10:00 p.m. on the assigned day of the final. 
	\end{enumerate}
	
	\item [] \textbf{Summary of Graded Items:}
	\begin{multicols}{2}
	\begin{enumerate}
		\item Memorandum: 10\%
		\item Student-led Reading Discussion: 10\%
		\item Paper 1: 10\%
		\item Paper 2: 10\%
		\item Budgeting Assignment: 10\%
		\item Program Evaluation Assignment: 10\%
		\item \emph{Ashworth} assignment: 10\%
		\item Final Response: 30\%
	\end{enumerate} 
	\end{multicols}
	
	\item[Grading Policy:] All graded items, including the final course grade, are assigned according to the following scale: 
	\begin{center}
		
		A+ = 98--100\%, A = 93--97\%, A- = 90--92\%, B+ = 87--89\%, B = 83--86\%, \\ B- = 80--82\%, C+ = 77--79\%, C = 73--76\%, C- = 70--72\%, D+ = 67--69\%, \\ D = 63--66\%, D- = 60--62\%, F = 0--59\%
	\end{center}
	
	\item[University Information:] There are numerous policy statements and regulations, as well as helpful resources, from the University. Below are just a few that are apply to this course and may be helpful for the student's graduate education.
	\begin{itemize}
		
		\item \underline{Academic Honesty:} Academic dishonesty includes such things as cheating, inventing false information or citations, plagiarism, and helping someone else commit an act of academic dishonesty. It usually involves an attempt by a student ot show a possession of a level of knowledge, which, in fact, the student does not possess. Cheating is defined as the act of obtaining or attempting to obtain credit for work by the use of any dishonest, deceptive, fraudulent, or unauthorized means. Plagiarism is defined as the act of taking the work of another and offering it as one's own without giving credit to that source. An instructor who believes that an act of academic dishonesty has occurred is (1) obliged to discuss the matter with the student(s) involved; (2) should possess reasonable evidence such as documents or personal observation; and (3) may take whatever action deemed appropriate, ranging from an oral reprimand to an F in the course. Additional information on this policy is available from the \href{http://www.fullerton.edu/senate/publications_policies_resolutions/ups/UPS 300/UPS 300.021.pdf}{University Policy Statement 300.021.}
		
		\item \underline{Accommodations for students with special needs:} Please inform the instructor during the first week of classes about any disability or special needs that you may have that may require specific arrangements related to attending class sessions, carrying out class assignments, or writing papers or examinations. Please do so by emailing the instructor to make an appointment to discuss your specific needs. According to California State University policy, students with disabilities must document their disabilities at the Disability Support Services (DSS) Office to be accommodated in their courses. Additional information can be found at the \href{http://www.fullerton.edu/dss}{DSS Website}, by calling 657-278-3112, or by emailing  \href{mailto:dsservices@fullerton.edu}{DSS}. 
		
		\item \underline{Emergency Preparedness:} To be able to respond effectively in an emergency, be sure to note (1) fire alarm pull station locations, (2) evacuation map including the class's outside meeting area, (3) emergency procedures for fire, medical emergency, hazardous materials release, earthquake and dangerous situations, and (4) the location of the nearest emergency phone. Any person with special needs is encouraged to privately speak with the instructor. All campus personnel are required to participate in all campus-wide emergency drills. Emergency preparedness information can be found at the \href{http://prepare.fullerton.edu/campuspreparedness/ClassroomPreparedness.asp}{Classroom Preparedness website}. 
		
		If an emergency disrupts normal campus operations or causes the University to close for a prolonged period of time (such as more than three days), students are expected to complete the course assignments listed on this syllabus as soon as reasonably possible to do so. At the instructors discretion, the syllabus may be amended or updated to reflect changing circumstances related to an emergency.
		\begin{itemize}
			\item[] Before and Emergency Occurs: 
			\begin{enumerate}
				\item Know the safe evacuation routs for your specific building and floor. 
				\item Know the evacuation assembly areas for your building. 
			\end{enumerate}
			\item[] If an Emergency Occurs: 
			\begin{enumerate}
				\item Keep calm; do not run or panic. It is best to maintain a clear head in an emergency situation. 
				\item Evacuation is not always the safest course of action. If directed to evacuate, take your belongings and proceed safely to the nearest evacuation route. 
				\item Do not leave the area. Remember that faculty and other staff members need to account for your whereabouts. 
				\item Do not re-enter the building until informed it is save by a building marshal or other campus authority. 
				\item If directed to evacuate the campus, please follow the evacuation routs established by either parking or police officers. 
			\end{enumerate}
		\end{itemize}
		
		\item \underline{Writing Center:} The Writing Center offers 30-minute one-on-one peer tutoring sessions and workshops, aimed at providing assistance for all written assignments and student writing concerns. Writing Center services are available to students from all academic disciplines. Registration and appointment schedules are available at the \href{http://fullerton.mywconline.com/}{Writing Center Appointment Scheduling System}. Walk-in appointments are also available on a first come, first serve basis to students who have registered online. More information can be found at the \href{http://www.fullerton.edu/learningassistance/tutoring_centers/writing.asp}{Writing Center's webpage}. The Writing Center is located on the first floor of the Pollak Library; their phone number is 657-278-3650.
		
		\item \underline{Graduate Student Success Center:} A new enhanced Graduate Student Success Center is now open on the third floor of Pollak Libary South, located in PLS 365. The center is open to all graduate students; it is equipped with computers and a printer, and it has plenty of room for studying and preparing assignments. 
	\end{itemize}
	
	\item[Technical Requirements:] Students are expected to: 
	\begin{enumerate}
		\item Have basic computer competency, which includes: 
		\begin{enumerate}
			\item the ability to use a personal computer to locate, create, move, copy, delete, name, rename, and save files and folders on hard drives, secondary storage devices, and clouds such as Dropbox and Google Drive; 
			\item the ability to use a word processing program to create, edit, format, store, retrieve, and print documents; 
			\item the ability to use their CSUF email accounts to receive, create, edit, print, save, and send an email message with and without an attached file(s); 
			\item the ability to use an Internet browser such as Chrome, Safari, Firefox, or Internet Explorer to search and access web sites. 
		\end{enumerate}
		\item Have ongoing reliable access to a computer with Internet connectivity for regular course assignments. 
		\item Utilize a word processing program such as Microsoft Word and a .pdf viewer such as Adobe Acrobat; and have the ability to regularly print assignments. 
		\item Maintain and access three times weekly their CSUF student email account. 
		\item Utilize \emph{Canvas} to access course materials and complete assignments. 
	\end{enumerate}
\end{description}

\vspace{1cm}

\begin{center}
	\begin{Large}
		\textbf{Tentative  Schedule} 
	\end{Large}
\end{center}

\noindent Note: On the schedule below, each particular reading is indicated for the class day on which it will be discussed. 
\begin{itemize}
	\item[] \textbf{Week 1: January 26} 
	\begin{itemize}
		\item Introduction and Overview
		\item Student Presentations Overview
		\item Basics about Graduate School and Using Library Website
		\item Eras of Public Administration
	\end{itemize}
	
	\item[] \textbf{Week 2: February 2} 
	\begin{itemize}
		\item Read: ``Running a Constitution'' ch. 1 in \emph{Essentials}
		\item Read: ``Citizen Engagement'' ch. 2 in \emph{Essentials}
		\item Memorandum Due
		
	\end{itemize}
	
	\item[] \textbf{Week 3: February 9} 
	\begin{itemize}
		\item Read: Wilson, Woodrow. (1887) ``The Study of Administration'' \emph{Political Science Quarterly}, 2(2), 197--222.
		\item \underline{Student Presentation 1}
		\begin{itemize}
			\item Deslatte, Aaron. (2020) ``The Erosion of Trust During a Global Pandemic and How Public Administrators Should Counter It.'' \emph{American Review of Public Administration}, 50(6-7), 489--496. 
		\end{itemize}	
		\item Read: ``The Savvy Administrator'' ch. 3 in \emph{Essentials}
		\item Read: ``Organizing Principles'' ch. 4 in \emph{Essentials}
		%\item Decision Making
	\end{itemize}
	
	\item[] \textbf{Week 4: February 16} 
	\begin{itemize}
		\item Organizing Principles Continued
		\item Read: Simon, Herbert. (1946) ``The Proverbs of Administration.'' \textit{Public Administration Review}, 6(1), 53--67. 
		\item Watch: YouTube videos linked on \emph{Canvas}
	\end{itemize}
	
	\item[] \textbf{Week 5: February 23} 
	\begin{itemize}
		\item Read: ``Human Resource Functions and Processes'' ch. 5 in \emph{Essentials} 
		\item Read: ``Budgeting'' ch. 6 in \emph{Essentials}
		\item \underline{Student Presentations 2 \& 3}
			\begin{itemize}
				\item Student Presentation 2
					\begin{itemize}
						\item Perlman, Bruce J. (2016) ``Human Resource Management at the Local Level: Strategic Thinking and Tactical Action.'' \emph{State and Local Government Review}, 48(2), 114--120.
					\end{itemize}
				\item Student Presentation 3
					\begin{itemize}
						\item Alfonso, Whitney. (2021) ``Planning for the Unknown: Local Government Strategies from the Fiscal Year 2021 Budget Season in Response to the COVID-19 Pandemic.'' \emph{State and Local Government Review}, 53(2), 159--171.
					\end{itemize}
			\end{itemize}
	\end{itemize}

	\item[] \textbf{Week 6: March 2} 
	\begin{itemize}
		\item Read: ``Digital Democracy'' ch. 7 in \emph{Essentials}
		\item \underline{Student Presentations 4 \& 5}
			\begin{itemize}
				\item Student Presentation 4
					\begin{itemize}
						\item Young, Matthew M. (2020) ``Implementation of Digital-Era Governance: The Case of Open Data in U.S. Cities.'' \emph{Public Administration Review}, 80(2), 3015--315.
					\end{itemize}
				\item Student Presentation 5
					\begin{itemize}
						\item Kim, Taehee, Lauren Bock Mullins, and Taweon Yoon. (2021) ``Supervision of Telework: A Key to Organizational Performance.'' \emph{American Review of Public Administration}, 51(4), 263--277.
					\end{itemize}
			\end{itemize}	
	\end{itemize}
		
	\item[] \textbf{Week 7: March 9} 
		\begin{itemize}
				\item Read: ``Foundations of Ethics in Public Administration'' ch. 1 in \emph{Ethics}
				\item Read: ``What is Ethics'' ch. 2 in \emph{Ethics}
				\item \underline{Student Presentation 6}
					\begin{itemize}
						\item Hollibaugh, Gary E., Matthew R. Miles, and Chad B. Newswander. (2019) ``Why Public Employees Rebel: Guerrilla Government in the Public Sector.'' \emph{Public Administration Review}, 80(1), 64--74.
					\end{itemize}
		\end{itemize}

	
	\item[] \textbf{Week 8:  March 16} 
	\begin{itemize}
		\item Read: ``Public Economics and Policy'' ch. 8 in \emph{Essentials}
		\item Read: ``Transparency'' ch. 10 in \emph{Essentials}
	\end{itemize}
	
	\item[] \textbf{Week 9: March 23} 
	\begin{itemize}
		\item Decision Making
			\begin{itemize}
				\item Read: Chapter 1 of March, James. (1994) \textit{A Primer on Decision Making: How Decisions Happen} New York: The Free Press.
				\item Read: Lindblom, Charles E. (1959) ``The Science of Muddling Through.'' \textit{Public Administration Review}, 19(2), 79-88.  
				\item TBD: ``\textit{The Abilene Paradox}'': Video and Discussion
			\end{itemize}
	\end{itemize}
	
	\item[] \textbf{Week 10: April 6} 
	\begin{itemize}
		\item Read: ``Contracting'' ch. 11 in \emph{Essentials}
		\item Read: ``Legal Dimensions of Public Administration'' ch. 12 in \emph{Essentials}
		\item \underline{Student Presentations 7 \& 8}
			\begin{itemize}
				\item Student Presentation 7
					\begin{itemize}
						\item Zeemering, Eric S. (2017) ``Why Terminate? Exploring the End of Interlocal Contracts for Police Service in California Cities.'' \emph{American Review of Public Administration}, 48(6), 596--609.
					\end{itemize}
				\item Student Presentation 8
					\begin{itemize}
						\item Christensen, Robert K., Holley T. Goerdel, and Sean Nicholson-Crotty. (2011) ``Management, Law, and the Pursuit of the Public Good in Public Administration.'' \emph{Journal of Public Administration Research and Theory}, 21(suppl-1), i125--i140.
					\end{itemize}
			\end{itemize}
	\end{itemize}
	
	\item[] \textbf{Week 11: April 13}
		\begin{itemize}
			\item Read: ``Public Integrity'' ch. 13 in \emph{Essentials}
			\item Read: \underline{Student Presentations 9, 10, 11}
				\begin{itemize}
					\item 9: Chapter 7 in \emph{Ethics}
					\item 10: Chapter 8 in \emph{Ethics}
					\item 11: Chapter 9 in \emph{Ethics}
					
				\end{itemize}
		\end{itemize}
	
	\item[] \textbf{Week 12: April 20}
	\begin{itemize}
		\item \textbf{Movie Night}
		\item \textbf{Pot-Luck Dinner}
		\item \emph{Canvas} Brief Reading and Discussion Board for Movie
	\end{itemize}

	
	\item[] \textbf{Week 13: April 27} 
		\begin{itemize}
			\item Contemporary Public Administration: Issue Networks, Collaboration, Collaborative Governance
			\item Read: TBD
			\item \underline{Student Presentations 12 \& 13}
				\begin{itemize}
					\item Student Presentation 12
						\begin{itemize}
							\item Cain, Bruce E., Elisabeth R. Gerber, and Iris Hui. (2020) ``The Challenge of Externally Generated Collaborative Governance: California's Attempt at Regional Water Management.'' \emph{American Review of Public Administration}, 50(4-5), 428--437.
						\end{itemize}
					\item Student Presentation 13
						\begin{itemize}
							\item Agranoff, Robert. (2006) ``Inside Collaborative Networks: Ten Lessons for Public Managers.'' \emph{Public Administration Review}, 66(s1), 56--65.
						\end{itemize}
				\end{itemize}
		\end{itemize}
	
	\item[] \textbf{Week 14: May 4} 
	\begin{itemize}
		\item Read: ``Measuring and Managing for Performance'' ch. 14 in \emph{Essentials}
		\item Read: ``Public Services, Well Delivered'' ch. 15 in \emph{Essentials}
		\item Read: Young, Sara L, Kimberly K. Wiley, and Elizabeth A. M. Searing. (2020) ```Squandered in Real Time': How Public Management Theory Underestimated the Public Administration--Politics Dichotomy.'' \emph{American Review of Public Administration}, 50(6-7), 480--488.
	\end{itemize}
	
	\item[] \textbf{Week 15: May 11} 
	\begin{itemize}
		\item Representative Bureaucracy and Diversity in Public Administration
			\begin{itemize} 
				\item Read: TBD
				\item Read: TBD
				\item Read: Jung, Jihye and Jochn C. Ronquillo. (2021) ``Racial Representation and Socialization in Bureaucratic Organizational Structures.'' \emph{American Review of Public Administration}, 51(3), 213--226.
			\end{itemize}
		
		\item Final Exam Distributed
	\end{itemize}

		
	\item[] \textbf{Final Response - May 18 submitted online}
\end{itemize}

\end{document}
